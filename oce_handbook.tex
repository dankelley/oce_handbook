\documentclass{article}
\usepackage{fullpage}
\usepackage{times}
\usepackage{ifthen}
\usepackage{microtype}
\usepackage{color}
\usepackage{xspace}
\usepackage{makeidx}
\usepackage{graphicx}
\usepackage[plainpages=false,debug]{hyperref} % should be last package

\newcommand\elink[1]{\url{#1}}

\definecolor{diColor}{rgb}{0.8,0,0} % hide them as black

\newcommand{\di}[2][EMPTY]{\ifthenelse{\equal{#1}{EMPTY}}{\color{diColor}#2\color{black}\index{#2}}{\color{diColor}#2\color{black}\index{#1}}}
\newcommand{\fixme}[1]{\color{red}$<$#1$>$\color{black}\index{$>>>>$FIXME$<<<<$}}

\newcommand{\curcom}{\index{Curriculum Committee}\index{Committee!Curriculum}Curriculum Committee}

\newcommand{\advcom}{\index{Advisory Committee}\index{Committee!Advisory}Advisory Committee}

\newcommand{\goc}{\index{Graduate Oversight Committee}\index{Committee!Graduate Oversight}Graduate Oversight Committee}

\makeindex

\begin{document}

% A.	GENERAL  PROGRAM REQUIREMENTS
\section{General Program Requirements}

%A1	REGISTRATION
\subsection{Registration}

Graduate students must maintain their registration in all three terms until
their program is completed, except in cases where a formal \di{Leave of
Absence} has been approved by the Faculty of Graduate Studies.

Registration consists of:
1.	REGN 9999
2.	Thesis Code
3.	Courses (if applicable)

Students who fail to register within the approved deadlines will be considered
to have \di{lapsed registration}. Such students will not be permitted to submit
a thesis, nor will they receive any services from the University during that
academic term. Students who allow their registration to lapse will be
considered to have withdrawn and will be required to apply for
readmission\footnote{FGS Regulations, section 5.2.3 Failure to Register
\elink{http://dalgrad.dal.ca/regulations/v5.2.3}}

%A1.1	REGN 9999
\subsection{REGN 9999}

Students must register at least one month prior to the beginning of each term.

Students must register for the \di{Fee Generating
Course}\index{course!fee-generating} (REGN 9999) in all three terms for the
duration of their program. If REGN 9999 is not added for each term, graduate
students are not considered to be registered. Failure to register at least one
month prior to the beginning of each term will result in non- payment of
scholarships and stipends.

The CRN for each term can be found in the Academic Timetable on the Registrar’s
Office website\footnote{Academic Timetable
\elink{http://www.dal.ca/academics.academic_timetable.html}.} as ``Registration
Course--Graduate''.

%A1.2	THESIS CODE
\subsection{Thesis Code}

Students must register at least one month prior to the beginning of each term.

Students must register for the Thesis Code in all three terms for the duration
of their program. Failure to do so in a term during which formal classes are
not being taken will result in a blank term on the student’s transcript, which
means that there will be no documentation demonstrating that work was done on
the thesis.

The CRN for each term can be found in the Academic Timetable under the
Oceanography course listings (MSc – OCEA 9000, PhD – OCEA 9530).



% A2	COURSES
\section{Courses}
 

%A2.1	CORE COURSES
\subsection{Core courses}

Core courses are broadly based, advanced-level courses that are intended to
help students gain a wide understanding of the major areas of oceanography, and
they are to be taken outside, as well as within, the student’s research area.
They are offered every year.

Depending on the degree program, students usually take some combination of the
\di{core course}s\index{course!core}, OCEA 5110, 5120, 5130, and 5140, which
cover Marine Geology and Geophysics, Physical Oceanography, Chemical
Oceanography and Biological Oceanography, respectively. One or more of these
sometimes may be waived with the consent of the thesis advisory committee, if
the student has sufficient background.


%A2.2	COURSE REQUIREMENTS
\subsection{Course requirements}

A Graduate Student Program form must be submitted to the Graduate Secretary
within one month of the start of your program.

Students must submit a Graduate Student Program Update Form to the Graduate
Secretary when a decision to make changes to course requirements is
made\footnote{Graduate Student Program Forms can be downloaded from the FGS
website \elink{http://dalgrad.dal.ca/currentstudents/forms}.}.  Failure to do
so can result in the assessment of additional fees for courses not included in
your program requirements, or affect the student’s ability to graduate.

In addition to the core classes as described above, there are several classes
designed to provide students with the tools and background knowledge
appropriate to their particular area of research. As explained in sections B
and C, the detailed requirements differ for MSc and PhD and between the
sub-disciplines of Oceanography. The choice of classes is made by the student
in consultation with the thesis supervisor, with the approval of the advisory
committee. Students must register for the 5000 level courses (4000 level is for
undergraduates only) prior to the registration deadline\footnote{Course
registration deadlines can be found in the Graduate Studies Academic Calendar
\elink{http://www.dal.ca/academics/academic_calendars.html}.}.

%A2.3	GRADING
\subsection{Grading}

FGS Regulations\footnote{FGS Regulations, section 7.6 Classes and Grades
\elink{http://dalgrad.dal.ca/regulations/vii/ - 7.6}.} stipulate that graduate
students must achieve a grade of B- or higher in all classes required as part
of their degree program.  The grading scheme follows the FGS regulations.

Upon Failure of any course (grade less than B-), a student will be withdrawn
from the Graduate Program and must apply for re-admission. Such a student may
apply, in writing, to the department for immediate reinstatement\footnote{FGS
Regulations, section 5.2.6 Readmission of Students
\elink{http://dalgrad.dal.ca/regulations/v/ - 5.2.6}.}.  Reinstatement to a
program after a failing grade must be supported by the Graduate Coordinator,
and must be approved in writing by the Faculty of Graduate Studies. If
readmitted, any subsequent 'F' will result in a final program dismissal. Any
academic withdrawal and reinstatement will be recorded on the student's
official transcript.

%A2.4	REGISTERING FOR COURSES AT ANOTHER INSTITUTION
\subsection{Registering for courses at another institution}

Classes approved by the Department and Faculty of Graduate Studies (after
examination of class descriptions) can be taken at other universities as part
of the graduate degree program, provided the class is not available at
Dalhousie\footnote{FGS Regulations, section 7.6.6 Letters of Permission
\elink{http://dalgrad.dal.ca/regulations/vii/ - 7.6.6}.}.  The Letter of
Permission Form\footnote{Letter of Permission Form and Guidelines
\elink{http://dalgrad.dal.ca/currentstudents/forms/ - lop}.} must be completed
and submitted to the Graduate Secretary in advance. Such approval will not be
given retroactively.


%A3	THESIS PROPOSAL
\subsection{Thesis proposal}

When a particular research area has been identified and an advisory committee
appointed, students are required to provide their advisory committees with
detailed thesis proposals. This document, typically developed in collaboration
with the supervisor, should demonstrate the student's

\begin{itemize}
\item Background of appropriate literature
\item Awareness of current research activity
\item Ability to formulate pertinent scientific hypotheses
\item Appreciation of the time, effort, and resources necessary to achieve the thesis objectives
\end{itemize}

The acceptance of a thesis proposal is a critical step in a student's program.
The detailed requirements differ for MSc and PhD proposals (see sections B and
C respectively).\fixme{fix section refs} 

%A4	SEA TIME
\subsection{Sea time}


Graduate students in Oceanography are required to spend time at sea on
oceanographic ships to familiarize themselves with oceanographic techniques,
even if their research does not require measurements at sea. Students are
required to submit a \di{Sea Time} Form to the Graduate Secretary. The \curcom
must approve completed forms. Requests for special consideration by the \curcom
for \di{fieldwork} in lieu of sea time may be made in extenuating
circumstances.


% A5	SEMINARS
\subsection{Seminars}

Students are required to attend the general \di{departmental
seminars}\index{seminars!departmental} and to attend, and participate in, the
specialty seminars in their field of interest.  Students who are unable to
attend seminars regularly must have the specific agreement of their \advcom
that this requirement is waived and this must be communicated to the
departmental office in a written memo signed by the supervisor. It is important
to note that materials presented in these seminars may form part of the
questions for the PhD Qualifying Exam and the PhD Thesis Proposal Defence.



 

%A6	CHANGE IN STATUS
\subsection{Change in status}

Any change in status, such as transfer from MSc to PhD (see section
B5\fixme{cross-ref}), a leave of absence, and entrance to another degree
program must be recommended by the advisory committee to the Graduate Oversight
Committee. Final approval for a change in status will be made by the Graduate
Oversight Committee, which will meet as required to review these requests.
Students should consult the Graduate Secretary regarding the required
documentation.


%A7	ANNUAL PROGRESS REPORT
\subsection{Annual progress report}

Students must submit an Annual Progress Report each year, one month prior to
the anniversary of their admission date. Failure to submit this report will
result in delays in registration and funding.

This report is made with the Graduate Student Information System (GSIS) through
DalOnline\footnote{DalOnline \elink{https://dalonline.dal.ca}.}.  The
supervisor and the Graduate Coordinator must approve this report
electronically, and students should bear this in mind so that the deadline can
be met.

A hard copy of the Oceanography Supplementary Program Report Form (available
from the Graduate Secretary) must also be approved by your supervisor and the
Graduate Coordinator, and submitted to the Graduate Secretary annually,
following the same deadline as the Annual Progress Report.

The Supplementary Program Report Form is intended to be saved electronically
and added to on a regular basis, similar to a CV. An up-to-date copy of the
Supplementary Program Report Form is to be brought to all meetings with the
Graduate Coordinator.


%A8	TIME LIMITS FOR COMPLETION OF DEGREES
\subsection{Time limits for completion of degrees}

Graduate students have a maximum period of time within which to complete their
graduate program, 4 years for MSc and 6 years for PhD.

Extensions may be granted by FGS on the recommendation of the department, along
with a satisfactory Progress Report\footnote{FGS Regulations, section 7.3
Maximum Time for Degree Completion and Extensions
\elink{http://dalgrad.dal.ca/regulations/vii/ - 7.3}.}.  Under no circumstances
can a student be registered in a program beyond 10 years from their initial
registration in the program.


%A9	THESIS ADVISORY COMMITTEE
\subsection{Thesis advisory committee}

The Thesis Advisory Committee must be formed within one month of the start of
the student’s program.

The supervisor will select a provisional committee, in consultation with the
student. A Graduate Student Program Form listing the Advisory Committee Members
must be submitted to the Graduate Secretary.

Membership on an Advisory Committee is flexible and may change during the
course of a student’s program. Students must complete a Graduate Student
Program Update Form when a decision to make changes to committee membership is
made. The Advisory Committee structure differs for MSc and PhD, as explained in
sections B and C\fixme{cross links}.

Advisory committees are formally sub-committees of the Department appointed
both to provide expert advice to students and to evaluate and report on student
progress.

All committee members must have Faculty of Graduate Studies membership (most
commonly as an External Scholar). Memberships should be applied for through the
Graduate Secretary.

Committees are changed by mutual consent within the committee or by the Chair
of the committee and student in consultation with the Graduate Coordinator or
Chair of the Department. Limits to changes are set by the availability of
particular people as supervisors and of research facilities and funds (noting
that most research grants are awarded for specific projects).

%A9.1	ADVISORY COMMITTEE MEETINGS
\subsection{Advisory committee meetings}

There must be no less than two advisory committee meetings held per academic
year\footnote{FGS Regulations, section 9.3 Supervisory Committees
\elink{http://dalgrad.dal.ca/regulations/ix/ - 9.3}.}.

Advisory committee meetings may be held as often as required by the student
and/or supervisor. It is the responsibility of the student to ensure that
committee meetings are held. The supervisor will ensure that brief minutes of
these meetings are recorded and are filed with the Graduate Secretary.

Following approval of the research proposal by the advisory committee, the
student should have several meetings in addition to the regular committee
meetings as follows:
\begin{itemize}

\item An interim progress meeting to verify that the research is on track.

\item A final progress meeting at which the committee agrees that sufficient
research has been conducted, and that thesis writing may proceed.

\item Individual meetings with advisory committee members to receive input on drafts or chapters.

\item A meeting at which the advisory committee agrees that the thesis is
defensible and that a defence may be scheduled. Alternatively, the advisory
committee may request further revision prior to approving a defence.

\end{itemize}





%A9.2	COMMITTEE APPROVAL
\subsection{Committee approval}


At several points in a student's program the approval of the advisory committee is essential:
\begin{itemize}
\item Initial discussion of research plan and direction; this should occur as early as possible in the student's program.
\item Approval of research plan and direction for thesis proposal.
\item Acceptance of a thesis proposal.
\item Change of status, such as advancement to the PhD program.
\item Approval to write up sections of a thesis.
\item Approval of draft of thesis.
\item The acceptance of the thesis, in final form, as being ready for defence.
\end{itemize}


%A9.3	SUPERVISOR
\subsection{Supervisor}

The Supervisor provides direct supervision of the student's research and is expected to
\begin{itemize}
\item Advise the student on the choice of a research topic, on the possible directions to emphasize during the work, and the point at which it should be concluded.
\item Supervise the research, as well as the preparation of the proposal, progress reports and thesis.
\item Ensure that the resources necessary to the thesis project are made available and that any necessary skills are acquired by the student.
\item Assess the student's progress, evaluating strengths and weaknesses.
\item Provide current updates of progress to the student's file.
\item Ensure that the student is funded
\item Select thesis examining committee members (following the guidelines for composition of the various examining committees).
\end{itemize}

A supervisor is changed only with the approval of the Department. This may
arise by mutual agreement within the advisory committee or at the request of
the student (to either the Graduate Coordinator or the Chair of the
Department).

In order to take advantage of the special expertise and facilities available,
the role of supervisor is not restricted to permanent faculty members of the
Department. However, supervisors must be Adjunct Professors or Full Professors
with Faculty of Graduate Studies membership in Oceanography.

In the case where a student has a supervisor who is not a full time faculty
member in the Department of Oceanography, there must be an internal supervisor
who is a faculty member in the Department.

%A9.4	INTERNAL SUPERVISOR
\subsection{Internal supervisor}

The responsibilities of the internal supervisor in part are
\begin{itemize}
\item To advise on academic requirements, course load, waiver of courses, general departmental affairs, etc.
\item To ensure that the student has suitable office and laboratory space.
\item In consultation with the student, to select a supervisor and advisory committee appropriate to the student's research interests (In many cases the chair will also be the supervisor).
\item To ensure that adequate reports and records of committee meetings are kept and submitted to the appropriate departmental files, and that necessary decisions as to the academic program, (e.g.. thesis proposal), are made on an appropriate time-scale.
\item To assess the progress of the student both in academic work and research.
\end {itemize}
 

%B.	MSc  PROGRAM REQUIREMENTS

\section{MSc program requirements}

% B1	MSc COURSE REQUIREMENTS

\subsection{MSc course requirements}

MSc students must complete a minimum of 5 half-credits at the 5000-level or
higher, at least three of which must be chosen from the introductory core
courses (refer to section A2.1)\fixme{cross link}.

Any student who anticipates a transfer to the PhD program should complete the
course-work and other requirements as listed in the PhD program requirements.

%B2	MSc COMMITTEE STRUCTURE
\subsection{MSc committee structure}

The Thesis Advisory Committee must be formed within one month of the start of
the student’s program.

The MSc Advisory Committee consists of at least three members. There must be
two full-time faculty members from the Oceanography Department (not Adjunct
Professors). If a regular faculty member serving on the committee leaves the
department, a replacement with another faculty member must be made. One member
will be from another sub-discipline, and it is desirable that at least one
member of the committee be from outside the Department.


%B3	MSc THESIS PROPOSAL
\subsection{MSc thesis proposal}

MSc students are expected to produce an approved proposal within one year of
enrolling in the program.

The thesis proposal should be developed in consultation with the supervisor and
advisory committee. The scope of the research should be such that it can be
accomplished and the thesis written within one year. Students should consult
with their supervisors on the content of a proposal.

The Supervisor will notify the Oceanography Office that the proposal
requirement has been satisfied and a copy of the approved proposal must be
placed on file with the Graduate Secretary. MSc students may be required to
defend the research proposal at the discretion of their committee.

Because a large portion of a student’s first year is consumed by course work,
the thesis proposal serves primarily as a well-reasoned course of action, and
not as an exhaustive literature review nor as a deeply detailed description of
methods. Fifteen pages of text should provide ample space.


%B4	MSc THESIS
\subsection{MSc thesis}

Graduate students should pay particular attention to the Faculty of Graduate
Studies deadlines for submission of a thesis for graduation and fee payment
schedules.
 
The examination for the degree of Master of Science is subject to detailed
regulations of the Faculty of Graduate Studies\footnote{FGS Regulations,
section 10.3 Master’s Theses \elink{http://dalgrad.dal.ca/regulations/x/ -
10.3}.}.

The MSc thesis should report original research of such value as to merit
publication and be in a satisfactory and consistent literary form. Faculty of
Graduate Studies thesis formatting requirements are available on their
website\footnote{FGS Common Thesis Format Review Comments
\elink{http://dalgrad.dal.ca/currentstudents/thesesanddefences/formattingcomments/}.}.


The oral defence of the thesis is open to all members of the department and to
other interested persons. Notices of the thesis defence, including an abstract,
will be distributed to all members of the Oceanography Department and will be
posted.

%B4.1	MSc EXAMINING COMMITTEE
\subsection{MSc examining committee}

It is the supervisor’s responsibility to select the examining committee
members, following the committee structure guidelines. The defence may NOT
proceed in violation of the below composition of the examining committee due to
short-term absence of committee members.

The examining committee will consist of a minimum of 5 members:
\begin{itemize}
\item 3 members of the advisory committee
\begin{itemize}
\item Supervisor(s)
    \item Full-time department faculty member
    \item One other member
    \end{itemize}

    \item Departmental Representative/Chair (from outside the
committee)\footnote{The Graduate Secretary will arrange for the Departmental
Representative (normally the Graduate Coordinator or a designate). The Chair of
the MSc thesis defence is independent of the examining committee. The
Departmental Representative normally performs this duty.}

    \item External examiner from outside the advisory
committee\footnote{adjunct professors may fulfill the requirement of an
external examiner.}

\end{itemize}


%B4.2	MSc THESIS DEFENCE TIMELINE
\subsection{MSc thesis defence timeline}

Students with immediate commitments subsequent to the defence must anticipate
the deadlines, rather than attempting to rush the defence process in a
compressed time frame.

Prior to proceeding with scheduling the defence, the thesis (in essentially
final form) should be distributed to the supervisor and committee to be
reviewed for suitability and final approval.

Six Weeks Prior to Defence:
\begin{itemize}
\item Have a format review done by FGS before the thesis is distributed to your examining committee
\item Submit an MSc Examination Information Form to the Graduate Secretary
\item Submit a Thesis Binding Submission Form and payment to the Graduate Secretary
\item Submit an electronic copy of the abstract to the Graduate Secretary
\item Provide copies of the thesis to the supervisor for distribution to the examining committee
\end{itemize}

Four Weeks Prior to Defence:
\begin{itemize}
\item Notify the Graduate Secretary of any Audiovisual requirements
\end{itemize}

Two Weeks Prior to Defence:
\begin{itemize}
\item Submit a printed copy of the thesis to the Graduate Secretary
\end{itemize}

Following a successful Defence:

\begin{itemize}
\item Have Examining Committee Members sign the signature page
\item Submit required changes to the supervisor within the specified timeframe
\item Have a final format check of the thesis done by FGS

\item Electronically submit the final version of the thesis via
DalSpace\footnote{DalSpace \elink{http://dalspace.library.dal.ca/}.}



\item Submit original completed forms to FGS (National Library of Canada Form,
Title Page of Thesis, Signature page with original signatures, Copyright Page,
Ethics Pages (if applicable), Student Contribution to Manuscripts (if
applicable)

\item Submit final copies of thesis to the Graduate Secretary for binding
(include copies of the signed signature page, and the Copyright Page with
original signatures)

\item Complete Exit Survey
\end{itemize}

%B4.3	MSc THESIS DEFENCE OUTCOME
\subsection{MSc thesis defence outcome}

Outcomes: All theses are either approved or not approved. The categories are:
\begin{itemize}
\item Approve as submitted
\item Approved pending corrections and a clear timetable for completion (normally within one month)
\item Rejected but with permission to re-submit a revised thesis for re-examination with a clear timetable for completion (within one year)
\item Rejected outright
\end{itemize}

A simple majority determines the outcome, based on all the examiners except the
Departmental Representative, who will vote only in the event of a tie.

Following the defence, the candidate will receive a letter from the Chair of
the examining committee indicating the nature of any corrections to be made and
the time frame within which they are to be completed. In the event of an
unsuccessful defence, an explanation of the negative outcome is provided. A
copy of the report will be provided to the Graduate Secretary.

\subsection{Transfer to PhD}

Request for transfer to the PhD program should be completed no later than one
month prior to the anniversary of the student’s admission date.

An MSc student may transfer to a PhD program without completing a MSc. Students
wishing to transfer should provide a letter to their advisory committee stating
the reason for the transfer, and justification that they are qualified for the
PhD program. A one-page explanation is sufficient. The student’s \advcom is
required to meet, and the committee may recommend advancement to the PhD
program. A letter from the supervisor, justifying the advancement of the
student to the PhD program, should be submitted to the Graduate Coordinator
along with the student’s letter no later than one month prior to the
anniversary of the student’s admission date. The \goc will then review the
request.

The decision categories for the transfer to the PhD are:
\begin{itemize}
\item Approved
\item Not approved
\item Deferred
\end{itemize}

Students anticipating this change should take the required core courses in the
first year and take the qualifying examinations (see section C3)\fixme{cross
ref} in the first year following completion of the required core classes.

%C.	PhD  PROGRAM REQUIREMENTS
\section{PhD program requirements}

%C1	PhD COURSE REQUIREMENTS
\subsection{PhD course requirements}

PhD students must complete a minimum of 6 half-credits at the 5000-level or
higher, at least two of which must be chosen from the introductory core courses
outside the student’s sub-discipline (refer to section A2.1)\fixme{cross link}.

Courses are to be chosen in consultation with the advisory committee. The
various sub-disciplines have individual course requirements, some of which
exceed the minimum requirements, as explained below.

%C1.1	BIOLOGICAL OCEANOGRAPHY GUIDELINES
\subsubsection{Biological Oceanography guidelines}

The normal course requirements for a PhD in Biological Oceanography are:
\begin{itemize}
\item Core Classes: Biological Oceanography (5140), Chemical Oceanography (5130), and Physical Oceanography (5120)
\item A minimum of one other half credit drawn from Geological Oceanography (5110) or a suitable Atmospheric Science course
\item Other Classes, as required by the Advisory Committee
\end{itemize}

%C1.2	CHEMICAL OCEANOGRAPHY GUIDELINES
\subsubsection{Chemical Oceanography guidelines}

The normal course requirements for a PhD in Chemical Oceanography are:

\begin{itemize}
\item Core Classes: Chemical Oceanography (5130), Geological Oceanography (5110), Physical Oceanography (5120), and Biological Oceanography (5140)
\item Advanced Chemical Oceanography (modular) (5290)
\item Other Classes, as required by the Advisory Committee
\end{itemize}

%C1.3	GEOLOGICAL OCEANOGRAPHY GUIDELINES
\subsubsection{Geological Oceanography guidelines}

The normal course requirements for a PhD in Geological Oceanography are:

\begin{itemize}
\item Core Classes: Geological Oceanography (5110) and two other Introductory Classes.
\item Other Classes, as required by the Advisory Committee
\end{itemize}

%C1.4	PHYSICAL OCEANOGRAPHY GUIDELINES
\subsubsection{Physical Oceanography guidelines}

The normal course requirements for a PhD in Physical Oceanography are:

\begin{itemize}
\item Core Classes: Physical Oceanography (5120) and two other Introductory Classes.
\item Advanced Classes: Fluid Dynamics (5311), Time Series Analysis (5210),
Ocean Dynamics (5221), Estuary, Coast and Shelf Dynamics (5222).
\item Other Classes, as required by the Advisory Committee e.g. Numerical
Modelling (5220), Ocean Waves (5223), Introduction to Acoustical Oceanography
(5250), Advanced Marine Particles (5293).
\end{itemize}

%C2	PhD COMMITTEE STRUCTURE
\subsection{PhD committee structure}

The Thesis Advisory Committee must be formed by October 1st of the first year
of study.

The PhD Advisory Committee consists of at least four members. There must be two
full-time faculty members from the Oceanography Department (not Adjunct
Professors) on the advisory committee. If a regular faculty member serving on
the committee leaves the department, a replacement with another faculty member
must be made.

One member will be from another sub-discipline, and it is desirable that at
least one member of the committee be from outside the Department.


%C3	PhD QUALIFYING EXAMINATION
\subsection{PhD Qualifying Examination}

The \di{qualifying examination} should be completed between months 9 and 12 of
the program.  Students transferring from the MSc program should take the
qualifying examinations within 12 months of their transfer.

This examination is designed to assess the student's background knowledge, with
the primary aim of identifying weaknesses that need to be addressed in order
for a student to undertake research at the Ph.D. level in the chosen field. The
format varies with sub-discipline. In each case there is an oral component, but
in some cases there can also be a written assignment before or after the oral
examination.

The \curcom  may grant extensions up to month 15 if the student and supervisor
document a compelling conflict, e.g. \di{fieldwork}.
 

%C3.1	QUALIFYING EXAMINATION COMMITTEE
\subsubsection{Qualifying Examination committee}

The examination committee has at least 4 members, including a faculty member
from outside the sub- discipline, and a departmental representative who acts as
chair. The Graduate Secretary will arrange for the Departmental Representative
(normally the Graduate Coordinator or a designate). The composition of the rest
of the examining committee is set by the sub-discipline.

The topics of the examination are student-specific, and may include items of a
general oceanographic nature (tailored to the courses taken) as well as items
that are more tightly focused on the sub-discipline and the intended area of
research.

%C3.2	FORMAT AND GUIDELINES
\subsubsection{Format and guidelines}

%C3.2.1  BIOLOGICAL OCEANOGRAPHY QE GUIDELINES
\subsubsection{Biological Oceanography QE guidelines}

The core of the qualification process is an oral examination based on a reading
list tailored to the student’s research area. In some cases this may be
followed by a written examination. The details and the timing are as follows.

1)	The PhD candidate prepares a brief (2-3 page) summary of his or her research interests and sends it to the BO faculty. The student’s summary of research interests guides faculty members in the selection of papers; it will not be used as source material during the exam, and neither the student nor the committee members bring copies of the research summary to the exam.

2)	Based on this document, at least 3 of the available BO faculty members (including the supervisor) each contribute one or two papers or book chapters to a reading list. These papers are intended to serve as guides for the exam and to provide entry points for discussions of oceanographic topics during questioning. The process is organized by the research supervisor, who vets the papers to ensure appropriateness, and then communicates the list to the student and the BO faculty within 2 weeks of receipt of the student’s research summary.

3)	It is the responsibility of the candidate’s supervisor to: a) form the Examining Committee; and b) schedule the exam. The committee will include a Chair, at least three members of the BO faculty, and one Oceanography faculty member from out of the BO sub-discipline, preferably in the sub-discipline most closely related to the student’s research. The Chair (a member of the Oceanography faculty not in BO) will serve as moderator of the exam and will not ask questions. 3 members of the BO group must approve the form of the examining committee no later than 3 weeks prior to the date of the exam. The composition of the committee will be revealed to the student and the student will see that each member has copies of the readings.

4)	Within 6 weeks after the reading list is given to the student, an oral examination is held.

5)	The exam begins with a 20-minute presentation by the student, highlighting aspects of the papers or chapters that are relevant to his or her research interests.

6)	During the next 1 - 1.5 h, the committee asks questions about oceanography and topics related to the students research interests. Entry points into questions and discussions are to be based on the reading
 

material that was provided to the student and by the student’s presentation.

7)	After the exam, the committee meets in camera to agree on comments and recommendations to the student and to the students committee, as appropriate.

8)	The possible outcomes of the oral examination are:
-	The candidate passes without extra conditions.
-	The candidate passes, but is informed of weaknesses that should be addressed during the PhD work,
e.g. in courses or in directed studies.
-	The candidate is required to take a written examination, at a date determined during the oral examination meeting. This examination will be based on the topics that arose during the oral examination and will not exceed three hours. The committee can then pass the student with no extra conditions, or be informed of weaknesses that should be addressed during the PhD work, e.g. in courses or in directed studies.

9)	The Chair is to report in writing to the Chair of the Curriculum Committee on the outcome of the exam, with copies to candidate and examining committee.


%C3.2.2  CHEMICAL OCEANOGRAPHY QE GUIDELINES
\subsubsection{Chemical Oceanography QE guidelines}

No discipline-specific guidelines – follow general Departmental guidelines
C3.2.3  GEOLOGICAL OCEANOGRAPHY QE GUIDELINES

No discipline-specific guidelines – follow general Departmental guidelines
C3.2.4  PHYSICAL OCEANOGRAPHY QE GUIDELINES
The core of the qualification process is an oral examination based on a reading list tailored to the student’s research area. In some cases this may be followed by a written examination. The details and the timing are as follows.

The PhD candidate prepares a brief (1 page) description of the intended area of research and sends it to the physical oceanography faculty members 5 weeks prior to the oral examination.

Based on this document, each available faculty member in the sub-discipline contributes a paper to a reading list. This process is organized by the research supervisor, who then communicates the list to the student and the Physical Oceanography faculty members within 1 week of receipt of the student’s research summary.

Within 4 to 6 weeks, an oral examination is held, to test the student’s comprehension of the reading list and its relevance to the intended research. The examination committee comprises all available faculty in the sub-discipline, a faculty member from another sub-discipline, and a Departmental representative.

The possible outcomes of the oral examination are:
-	The candidate passes without extra conditions
-	The candidate passes, but is informed of weaknesses that should be addressed during the PhD work,
 

e.g. to be addressed in courses or in directed studies
-	The candidate is required to sit written examination, at a date determined during the oral examination meeting. This examination will be based on the topics arising during the oral defence.
C3.3	QUALIFYING EXAMINATION OUTCOME

There are several possible outcomes of the qualifying examination:

-	Continuation in the PhD program with additional course work or directed studies
-	Continuation in the PhD program without additional course work or directed studies
-	Transfer to the MSc program
-	Transfer to the PhD program with additional course work or directed studies
-	Transfer to the PhD program without additional course work or directed studies
-	Continuation in the MSc program
C4	PhD THESIS PROPOSAL AND DEFENCE

PhD students are expected to complete the proposal and examination of the proposal within 20 months of enrolling in the graduate program.

The advisory committee sets the exact timing.

In order to proceed with the proposal, the student will have completed the required first year courses, have a named supervisor and advisory committee, and received approval from the supervisor.

The requirement consists of a written thesis proposal and a public oral defence.  The written proposal will include an overview of the research topic and relevant background material, and a plan for carrying out the PhD research. The oral portion is a defence of the proposal, including other relevant areas of Oceanography. The topics covered in the defence may relate not just to the planned research, but also to wider oceanographic issues, as deemed appropriate by the defence committee.

The defence is open to attendance by all interested persons, except for the in-camera portion in which the decision is made by the examining committee. Notices of the thesis proposal defence, including an abstract will be distributed to all members of the Oceanography Department and will be posted.
C4.1	EXAMINING COMMITTEE

The examining committee is selected by the supervisory committee, and consists of members from that committee, external members, and a departmental representative. The Graduate Secretary will arrange for the Departmental Representative (normally the Graduate Coordinator or a designate).

The Chair of the PhD proposal defence is independent of the examining committee. The Departmental Representative normally performs this duty.

Students must make the proposal available in the Department of Oceanography office at least 10 working days prior to the oral defence.  The proposal typically receives input from the research
 

supervisor or others prior to its submission. Approval from the advisory committee is required to proceed with the oral defence.
C4.2	PROPOSAL TIMELINE

It is the student's responsibility to initiate the proposal process and develop the proposal in coordination with the supervisor and committee. Two terms are more than adequate to complete this requirement. The scheduling of the proposal defence should be planned well in advance so that it can be carried out prior to the 20th month of the program.

There is no restriction on early completion of the proposal and oral exam (e.g. in the summer between first and second years or fall term of second year).

Because the proposal is a formal requirement of the Department, it carries the same weight as courses, thesis defence, etc. Failure to complete the proposal within the required time frame reflects poorly on the student and can endanger standing in the program.
C4.3	PROPOSAL OUTCOME

Based on both the written proposal and oral defence, the supervisor will notify the Graduate Secretary, Graduate Coordinator, student, and examining committee in writing of one of the four possible outcomes:
-	Continuation in the PhD program
-	Permission to re-defend
-	Transfer to the MSc program
-	Withdrawal from the program

Revisions, and/or a re-defence must be completed prior to the end of the exam period in the winter term. Following the defence, a copy of the approved proposal must be placed on file in the Oceanography office.
C5	PhD THESIS

The PhD thesis should report original research of high caliber carried out by the student. It should be of such value as to merit publication and be in satisfactory literary form, with clearly presented figures and other supporting material. The thesis should be presented first, in draft form, to the supervisor who will provide initial comments and approve distribution to the rest of the committee. Typically, the writing phase is very intensive, and involves working closely with the supervisor and committee members. In many cases, the student writes journal articles that link closely with the thesis chapters, and the Faculty of Graduate Studies has conventions on how this is to be handled. As the thesis reaches completion, the advisory committee will meet to discuss any final changes to the thesis and sign the PhD Thesis Submission Form; this is just one of a series of steps that the student must keep in mind (see section C5.2).
C5.1	PhD EXAMINING COMMITTEE
 

It is the supervisor’s responsibility to select the examining committee members, following the committee structure guidelines.

The examining committee will consist of a minimum of 5 members:
-	3 members of the advisory committee
o	Supervisor(s)
o	Full-time department faculty member
o	One other member
-	Departmental Representative (from outside the committee)*
-	External examiner (from outside the University)
-	Chair **

*The Graduate Secretary will arrange for the Departmental Representative (normally the Graduate Coordinator or a designate).

**The Chair of the PhD thesis defence is independent of the examining committee. The Faculty of Graduate Studies will appoint the Chair.

C5.2	PhD THESIS DEFENCE TIMELINE

Students should be aware of the time scales required by FGS and Departmental regulations, which stretch over a period of more than eight months before the defence.

Prior to proceeding with scheduling the defence, the thesis (in essentially final form) should be distributed to the supervisor and committee to be reviewed for suitability and final approval.

Twelve Weeks Prior to Defence:
-	Submit Request to Arrange an Oral Defence Form, listing 3 choices for external examiners along with CV for external examiner of first choice to the Graduate Secretary.

Six Weeks Prior to Defence:
-	Have a format review of the thesis done by FGS before it is distributed to the examining committee
-	Submit a PhD Thesis Submission From and copy of the thesis for the External Examiner to the Graduate Secretary
-	Submit a PhD Examination Information Form to the Graduate Secretary
-	Submit a copy of the student’s CV to the Graduate Secretary
-	Submit a Thesis Binding Submission Form and payment to the Graduate Secretary
-	Submit an electronic copy of the abstract to the Graduate Secretary
-	Provide copies of the thesis to the supervisor for distribution to the internal examining committee members

Four Weeks Prior to Defence:
-	Notify the Graduate Secretary of any Audiovisual requirements Two Weeks Prior to Defence:
 

-	Submit a printed copy of the thesis to the Graduate Secretary

Following Defence:
-	Have Examining Committee Members sign the signature page
-	Submit required changes to the thesis within the specified timeframe
-	Have a final format check done by FGS
-	Electronically submit the final version of the thesis via DalSpace15
-	Submit original completed forms to FGS (National Library of Canada Form, Title Page of Thesis, Signature page with original signatures, Copyright Page, Ethics Pages (if applicable), Student Contribution to Manuscripts (if applicable)
-	Submit final copies of thesis to the Graduate Secretary for binding (include copies of the signed signature page, and the Copyright Page with original signatures)
-	Complete Exit Survey

Some of the above deadlines are set by the Oceanography Office. Important dates from the FGS PhD Thesis And Defence Timeline and Checklist16 are provided above.
C5.3	PhD THESIS DEFENCE OUTCOME

The student opens the examination with a 20-minute presentation of the thesis. Other faculty, students and public may be present, and ask questions, but the critical decisions are made in private and are the responsibility of the examining committee.

Following the defence, the candidate will receive a letter from the Faculty of Graduate Studies indicating the nature of any corrections to be made and the time frame within which they are to be completed.  In the event of an unsuccessful defence, an explanation of the negative outcome is provided.
C5.4	PhD SEMINAR PRESENTATION

A departmental seminar on the thesis should be presented after successful completion of the thesis examination.

D.	FINANCIAL  SUPPORT
Dalhousie Graduate Fellowships will normally be provided for two years for an MSc student and four years for a PhD student.

Scholarship support is provided directly to some students by the Natural Sciences and Engineering Research Council (NSERC), the Killam Foundation through Dalhousie, by companies, and by other agencies, both national and international. Students are encouraged to apply for external support whenever possible.


15 DalSpace \elink{http://dalspace.library.dal.ca/}.
16 PhD Candidate Thesis And Defence: Timeline \& Checklist \elink{http://dalgrad.dal.ca/currentstudents/thesesanddefences/checklists/}.
 

Full-time students without such special scholarship support are generally awarded Dalhousie Graduate Fellowships from funds that come partly to the Department from the University and mainly from individual faculty members’ research grants.

Students may also receive extra support by taking on a teaching assistantship, or by demonstrating in undergraduate laboratories.

Students should discuss any additional jobs with their advisory committee to ensure that the time commitments outside the Department are not excessive, and do not interfere with program expectations.
D1	FGS CONFERENCE TRAVEL GRANT

Applications must be submitted through the Graduate Secretary at least one month prior to the date of travel.

Graduate students are eligible to apply for one travel grant per degree at Dalhousie (Master’s or Doctoral) through the Faculty of Graduate Studies17. Students must be registered in a graduate program at the time of application and at the time of the conference. In order to be eligible, students must present a poster of paper based on the results of their graduate thesis research at a national or international scholarly meeting or conference. Departmental approval must be given to these applications.


E.	INTELLECTUAL  PROPERTY

The Faculty of Graduate Studies is developing a policy on Intellectual Property, which will be available on the Faculty of Graduate Studies website.
If students and/or faculty have concerns or doubts about any issue pertaining to any part of Section 6, consult with your Chair, Graduate Coordinator, or Supervisor, or contact the Faculty of Graduate Studies for advice. If you feel uncomfortable with approaching your immediate supervisor, then go to the next level and ask to be heard in confidence18.
E1	ACADEMIC INTEGRITY

Any material submitted by a student at Dalhousie University may be checked for originality to confirm that the student has not plagiarized from other sources. Plagiarism is a serious academic offence that may lead to a failing grade, suspension or expulsion from the University, or even the revocation of a degree.  It is essential that there be correct attribution of authorities from which facts and opinions have




17 FGS Conference Travel Grant Application Form and Guidelines \elink{http://dalgrad.dal.ca/currentstudents/forms/ - travel}.
18 FGS Regulations, section 6.4 Policy on Intellectual Property \elink{http://dalgrad.dal.ca/regulations/vi/ - 6.4}.
 

been derived. Prior to submitting any paper in a course, or material for a thesis, students should read the Policy on Intellectual Honesty19.





-










































19 Policy on Intellectual Honesty \elink{http://www.dal.ca/dept/university_secretariat/academic- integrity.html}.
 
\printindex

\end{document}
