\documentclass[12pt]{article}
\usepackage{fullpage}
%\usepackage{mathptmx}
\usepackage{times}
\usepackage{ifthen}
\usepackage{microtype}
\usepackage{color}
\usepackage{xspace}
\usepackage{makeidx}
\usepackage{enumitem}
\usepackage{graphicx}
\usepackage[normalem]{ulem}
\usepackage{titlesec}
\usepackage{tikz}
\usepackage[plainpages=false,debug]{hyperref} % should be last package

%\usepackage{fancyhdr}
%
%\fancyfoot{}
%\fancyhead[RO,LE]{\thepage}
%\fancyhead[LO]{}
%\fancyhead[RE]{}

\setlist{nosep} % tighten itemized lists

\definecolor{diColor}{rgb}{0,0,0.8}
\definecolor{deleteColor}{rgb}{1,0,0}
\definecolor{addColor}{rgb}{0.0,0.5,0}

\definecolor{fixmeColor}{rgb}{0.8,0,0} % hide them as black
\definecolor{voteColor}{rgb}{0.5,0,1} % hide them as black

\newcommand\ie{i.e.\xspace}
\newcommand\eg{e.g.\xspace}
\newcommand\qe{PhD Qualifying Examination\index{PhD Qualifying Examination}\xspace}
\newcommand\QE{PhD Qualifying Examination\index{PhD Qualifying Examination}\xspace}

\newcommand\delete[1]{\color{deleteColor}\sout{\textbf{#1}}\color{black}\xspace%
\marginpar{\color{deleteColor}$\Delta$\color{black}}}

\newcommand\add[1]{\color{addColor}\textbf{#1}\color{black}\xspace%
\marginpar{\color{addColor}$\Delta$\color{black}}}

%\newcommand\edited{\marginpar{\color{red}$\Delta$\color{black}}}

\newcommand{\course}[2][EMPTY]%
    {\ifthenelse%
        {\equal{#1}{EMPTY}}%
        {OCEA #2}%
        {#1 #2}%
    }

\newcommand{\di}[2][EMPTY]%
    {\ifthenelse%
        {\equal{#1}{EMPTY}}%
        {\color{diColor}#2\color{black}\index{#2}}%
        {\color{diColor}#2\color{black}\index{#1}}%
    } 

\newcommand{\fixme}[1]{\color{fixmeColor}$\{$#1$\}$\color{black}\index{$>>>>$FIXME$<<<<$}}

\newcommand{\vote}[1]{\color{voteColor}$\ll$#1$\gg$\color{black}\marginpar[$\gg$ vote]{$\ll$ vote}\index{$>>>>$VOTE$<<<<$}}

\newcommand{\discuss}[1]{\footnote{\color{fixmeColor}#1\color{black}}\index{$>>>>$DISCUSS$<<<<$}}

% sm = supplemental material
%\newcommand{\sm}[2]{\noindent \texttt{#1}}



%\newcommand{\npara}[1]{\noindent\textbf{\thesubsection.#1}} % for numbered paragraphs of appeals section

\setlength{\parindent}{0em}
% number paragraphs
\newcommand{\parnum}{\arabic{parcount}}
\newcounter{parcount}
%\newcommand\p{\stepcounter{parcount}\leavevmode[\parnum]\hspace{0.2em}\marginpar[\hfill\parnum]{\parnum}}
\newcommand\p{\stepcounter{parcount}\leavevmode{\raisebox{0.2ex}{\scriptsize[\parnum]}}\hspace{0.2em}}
\newcommand\cp{\setcounter{parcount}{0}}

\newcommand{\furl}[1]{\footnote{\footnotesize{\url{#1}}}}

% Use macros to get indexing, and ensure uniform representation
\newcommand{\supervisor}{\index{supervisor}supervisor\xspace}
\newcommand{\curcom}{\index{Committees!Curriculum}Curriculum Committee\xspace}
\newcommand{\advcom}{\index{Committees!Advisory}Advisory Committee\xspace}
\newcommand{\gocom}{\index{Committees!Graduate Oversight}Graduate Oversight Committee\xspace}
%\newcommand{\GC}{\index{Graduate Coordinator}Graduate Coordinator\xspace}
\newcommand{\GC}{Graduate Coordinator\xspace}
%\newcommand{\GS}{\index{Graduate Secretary}Graduate Secretary\xspace}
\newcommand{\GS}{Graduate Secretary\xspace}
\newcommand{\formAPR}{\index{forms!Annual Progress Report}Annual Progress Report Form\xspace}
\newcommand{\formGSP}{\index{forms!Graduate Student Program}Graduate Student Program Form\xspace}
\newcommand{\formGSPU}{\index{forms!Graduate Student Program}Graduate Student Program Update Form\xspace}
\newcommand{\formSPRF}{\index{forms!Supplementary Program Report}Supplementary Program Report Form\xspace}
\newcommand{\formST}{\index{forms!Sea Time}Sea Time Form\xspace}
\newcommand{\FGS}{\index{Faculty of Graduate Studies (\textsc{fgs})}\textsc{fgs}\xspace}
\newcommand{\NSERC}{\index{Natural Sciences and Engineering Research Canada (\textsc{nserc})}\textsc{nserc}\xspace}
\newcommand{\GSIS}{\index{Graduate Student Information System (\textsc{gsis})}\textsc{gsis}\xspace}
\newcommand{\dalonline}{\index{Dal Online}DalOnline\xspace}


\setlength{\parskip}{0.7ex plus 0.5ex minus 0.2ex}

\setcounter{secnumdepth}{5}
\titleformat*{\section}{\large\bfseries}
\titlespacing*{\section}{0pt}{*0}{0pt}
\titleformat*{\subsection}{\bfseries\slshape}
\titlespacing*{\subsection}{0pt}{1.5ex plus 0.5ex minus 0.2ex}{0pt}
\titleformat*{\subsubsection}{\slshape}
\titlespacing*{\subsubsection}{0pt}{*0}{0pt}
%\titleformat*{\subsubsubsection}{}
%\titlespacing*{\subsubsubsection}{0pt}{*0}{0pt}
\titleformat*{\paragraph}{\slshape}
\titlespacing*{\paragraph}{0pt}{*0}{0pt}

\newcommand{\ExternalLink}{%
    \tikz[x=1.2ex, y=1.2ex, baseline=-0.05ex]{% 
        \begin{scope}[x=1ex, y=1ex]
            \clip (-0.1,-0.1) 
                --++ (-0, 1.2) 
                --++ (0.6, 0) 
                --++ (0, -0.6) 
                --++ (0.6, 0) 
                --++ (0, -1);
            \path[draw, 
                line width = 0.5, 
                rounded corners=0.5] 
                (0,0) rectangle (1,1);
        \end{scope}
        \path[draw, line width = 0.5] (0.5, 0.5) 
            -- (1, 1);
        \path[draw, line width = 0.5] (0.6, 1) 
            -- (1, 1) -- (1, 0.6);
        }
    }

%\renewcommand*\contentsname{}

\makeindex

\title{Oceanography Graduate Handbook}
\author{}
\date{for GOC/GAC discussion March 10, 2017}


\begin{document}
\clearpage
\maketitle
\thispagestyle{empty}

\textbf{Disclaimer.} The guidelines in this handbook supplement, but do not,
and cannot, reduce the requirements listed in the rules and regulations
published in the Graduate Studies Calendar of Dalhousie University.  As such,
this document adds further requirements that students must accomplish in order
to complete a degree program in Oceanography.  Such additional prerequisites
are permitted by Graduate Studies.  In the event of a real or perceived
conflict in such stipulations, Graduate Studies (\FGS) requirements and rules
shall prevail.

\newpage

\renewcommand{\baselinestretch}{0.5}\normalsize

\pagenumbering{roman}
{\footnotesize
\setcounter{tocdepth}{2}
\tableofcontents
}
\thispagestyle{empty}
\renewcommand{\baselinestretch}{1.0}\normalsize

\pagenumbering{arabic}
%\pagestyle{fancy}
\setcounter{page}{0}

\section{\label{sec:glossary}Glossary of acronyms and terms}

The following list explains some acronyms and special terms used in this
document.

\begin{itemize}

    \item ``CRN'' refers to a Course Registration Number. For example, the
        Physical Oceanography core class, listed as \course{5120} in the University
        Calendar, has CRN 12070. Consult the Registrar's website\furl{https://dalonline.dal.ca/PROD/fysktime.P_DisplaySchedule?s_term=201710,201720&s_subj=OCEA&s_district=100} for more information.

    \item \dalonline is the gateway online system (at
        \url{https://dalonline.dal.ca}) for student-university interaction.

    \item ``External'' \fixme{write a few lines here, with the 3 types
        discussed in the meeting of 20160527}.

    \item \FGS stands for Faculty of Graduate Studies.

    \item \NSERC stands for Natural Science and Engineering Research Council.

\end{itemize}


% A.	GENERAL  PROGRAM REQUIREMENTS
\section{\label{sec:general_program_requirements}General Program Requirements}

%A1	REGISTRATION
\subsection{Registration}

\p Graduate students must maintain their registration in all three terms of
each year in their program, except in cases where a formal \di[leave of
absence]{Leave of Absence} has been approved by the Faculty of Graduate
Studies. This involves registering for a ``course'' named ``Registration
Course--Graduate'' (designated \course[REGN]{9999}, and also called a
``fee-generating course''), as well as either ``Masters Thesis''
(\course{9000}) or ``PhD thesis'' (\course{9530}); see the Registrar's
website\footnote{\url{http://www.dal.ca/faculty/gradstudies/currentstudents/registration.html}}.
for more information, including the relevant Course Registration Numbers. It is
critical to register for these courses in time, since failure to register at
least 1 month prior to the beginning of a term will result in non-payment of
scholarships and stipends.


%\p Registration consists of
%\begin{enumerate}
%    \item REGN 9999
%    \item Thesis Code
%    \item Courses (if applicable)
%\end{enumerate}

\fixme{DK: somewhere hereabouts, insert the diagram I made of the timeline for
MSc and PhD}

\p If \course{REGN~9999} is not added for each term, graduate students are not
considered to be registered. Failure to register at least one month prior to
the beginning of each term will result in non-payment of scholarships and
stipends.

\p Students who fail to register within the approved deadlines will be
considered to have \di{lapsed registration}. Such students will not be
permitted to submit a thesis, nor will they receive any services from the
University during that academic term. Students who allow their registration to
lapse will be considered to have withdrawn and will be required to apply for
readmission\furl{http://academiccalendar.dal.ca/Catalog/ViewCatalog.aspx?pageid=viewcatalog&catalogid=58&chapterid=2678&topicgroupid=10848}.

%%A1.1	REGN 9999
%\subsection{REGN 9999}
%
%\cp
%
%\p Students must register at least one month prior to the beginning of each term.
%
%\p Students must register for the Fee Generating
%Course\index{courses!fee-generating} (REGN 9999) in all three terms for the
%duration of their program. If REGN 9999 is not added for each term, graduate
%students are not considered to be registered. Failure to register at least one
%month prior to the beginning of each term will result in non- payment of
%scholarships and stipends.

%%A1.2	THESIS CODE
%\subsection{Thesis Code}
%
%\cp
%
%\p Students must register at least one month prior to the beginning of each term.
%
%\p Students must register for the Thesis Code in all three terms for the duration
%of their program. Failure to do so in a term during which formal classes are
%not being taken will result in a blank term on the student's transcript, which
%means that there will be no documentation demonstrating that work was done on
%the thesis.
%
%\p The CRN for each term can be found in the Academic Timetable under the
%Oceanography course listings (MSc – OCEA 9000, PhD – OCEA 9530).



% A2	COURSES
\subsection{\label{sec:courses}Courses}

\cp

\p In addition to the artificial courses listed in the previous section,
students take conventional courses, chosen in consultation with their
supervisors, and pursuant to the requirements set out in
section~\ref{sec:msc_course_requirements} or
section~\ref{sec:phd_course_requirements} for MSc and PhD students,
respectively.

\p Within the Department of Oceanography, courses are designated as either
``core'' or ``non-core.'' The \di{core classes} are
Biological Oceanography (\course{5140}),
Chemical Oceanography (\course{5130}),
Geological Oceanography (\course{5110}),
and
Physical Oceanography (\course{5120}). These 
are broadly based courses that are intended to
help students gain a wide understanding of the major areas of oceanography.
They are offered every year, typically with Physical and Chemical in the 
September term, to set the stage for Geological and Biological in the January term.
Note that these courses are cross-listed to 4th year classes at the
Undergraduate level, and there are circumstances in which they may be waived
for Dalhousie students who have taken them at this lower level (see
Section~\ref{sec:msc_course_requirements} for MSc and
Section~\ref{sec:phd_course_requirements} for PhD).

\p The non-core classes tend to be more advanced and specialized. They include
Fluid Dynamics (\course{5311}),
Time Series Analysis (\course{5210}),
Ocean Dynamics (\course{5221}),
Estuary, Coast and Shelf Dynamics (\course{5222}),
Numerical Modelling (\course{5220}),
Ocean Waves (\course{5223}),
Introduction to Acoustical Oceanography (\course{5250}),
Advanced Chemical Oceanography (\course{5290}),
and
Advanced Marine Particles (\course{5293}).  Not all of these are offered every
year. Details of offerings are available in the departmental office and at the
Registrar's
website\furl{https://dalonline.dal.ca/PROD/fysktime.P_DisplaySchedule?s_term=201710,201720&s_subj=OCEA&s_district=100}.


\p Students select courses in consultation with their {\supervisor}s, with the
approval of members of their \advcom. Students must register for the 5000
stream of courses that are cross-listed at the 4000 and 500 levels (see
Section~\ref{sec:msc_course_requirements} or \ref{sec:phd_course_requirements}
for notes on the case in which 4000-level classes were taken previously) prior
to the registration deadline.  See the Graduate Studies Academic
Calendar\furl{http://academiccalendar.dal.ca/Catalog/ViewCatalog.aspx?pageid=viewcatalog&catalogid=58&chapterid=2677}
for deadlines.


\p A \formGSP must be submitted to the \GC within one month of the start of
your program.  Students must submit a \formGSPU to the \GS when a decision to
make changes to course requirements is made.  Graduate Student Program
Forms can be downloaded from the FGS
website\furl{http://www.dal.ca/faculty/gradstudies/currentstudents/forms.html}.
Failure to submit a \formGSP within the stated timetable can result in the
assessment of additional fees for courses not included in the program
requirements, or affect the ability to graduate.

%\p In addition to the core classes as described above, there are several
%classes designed to provide students with the tools and background knowledge
%appropriate to their particular area of research. As explained in sections B
%and C, the detailed requirements differ for MSc and PhD and between the
%sub-disciplines of Oceanography. The choice of classes is made by the student
%in consultation with the thesis supervisor, with the approval of the advisory
%committee. Students must register for the 5000 level courses (4000 level is for
%undergraduates only) prior to the registration deadline\footnote{Course
%registration deadlines can be found in the Graduate Studies Academic Calendar
%\href{http://academiccalendar.dal.ca/Catalog/ViewCatalog.aspx?pageid=viewcatalog&catalogid=58&chapterid=2677}.}.

%A2.3	GRADING
%\subsubsection{Grading}

\p FGS
Regulations\furl{http://academiccalendar.dal.ca/Catalog/ViewCatalog.aspx?pageid=viewcatalog&catalogid=58&chapterid=2678&topicgroupid=10850}
stipulate that graduate students must achieve a grade of B- or higher in all
classes required as part of their degree program. The only grade that can be
assigned below a ``B-'' is an ``F''. Any student who receives an F will be withdrawn
from the Graduate Program and must apply for re-admission. Such a student may
apply, in writing, to the department for immediate reinstatement; see \FGS
Regulations, section~5.4.1: Readmission of
Students\furl{http://academiccalendar.dal.ca/Catalog/ViewCatalog.aspx?pageid=viewcatalog&catalogid=58&chapterid=2678&topicgroupid=10848}.
Reinstatement to a program after a failing grade must be supported by the \GC,
and must be approved in writing by the Faculty of Graduate Studies. If
readmitted, any subsequent ``F'' will result in a final program dismissal. Any
academic withdrawal and reinstatement will be recorded on the student's
official transcript.
% above was 5.2.6 but now seems to be 5.4

%A2.4	REGISTERING FOR COURSES AT ANOTHER INSTITUTION
%\subsection{Registering for courses at another institution}

%\cp

\p Classes approved by the Department and \FGS (after
examination of class descriptions\fixme{Q: who approves these? I've done that
as the professor for my own course. Is that the rule, or the habit?}) can be
taken at other universities as part of the graduate degree program, provided
the class is not available at Dalhousie; see \FGS Regulations, section 7.6.6:
Letters of Permission
\furl{http://academiccalendar.dal.ca/Catalog/ViewCatalog.aspx?pageid=viewcatalog&catalogid=58&chapterid=2678&topicgroupid=10850}.
The Letter of Permission, available at the \FGS
website\furl{http://www.dal.ca/faculty/gradstudies/currentstudents/forms.html},
must be completed and submitted to the \GC in advance. Such approval will not
be given retroactively.\fixme{NB: need to check with Lori on what we have done
in the past, and what FGS or the registrar says we must do now.}


%A3	THESIS PROPOSAL
\subsection{Thesis proposal}

\cp

%\p When a particular research area has been identified and an advisory committee
%appointed, students are required to provide their advisory committees with
%detailed thesis proposals. This document, typically developed in collaboration
%with the supervisor, should demonstrate the student's
%
%\begin{itemize}
%\item Background of appropriate literature
%\item Awareness of current research activity
%\item Ability to formulate pertinent scientific hypotheses
%\item Appreciation of the time, effort, and resources necessary to achieve the thesis objectives
%\end{itemize}

%\p The acceptance of a thesis proposal is a critical step in a student's program.

\p See sections~\ref{sec:msc_thesis_proposal} and~\ref{sec:phd_thesis_proposal}
for requirements in the MSc and PhD programs, respectively.

%A4	SEA TIME
\subsection{Sea time}

% discussed Oct 2 2015 by the curr. committ.
% discussed in a departmental meeting in Oct 2014

\cp

% \p Graduate students in Oceanography are required to spend time at sea on
% oceanographic ships to familiarize themselves with oceanographic techniques,
% even if their research does not require measurements at sea. Students are
% required to submit a \di[sea time]{Sea Time} Form to the Graduate Secretary.
% The \curcom must approve completed forms. Requests for special consideration by
% the \curcom for \di{fieldwork} in lieu of sea time may be made in extenuating
% circumstances.

\p The Department of Oceanography has a longstanding tradition of requiring its
graduate students to go to sea\index{sea time} on research cruises as part of
their degree requirements. This tradition meets several objectives:

\begin{itemize}

    \item It exposes students of all backgrounds and research interests,
        and to the rewards and challenges of gathering data at sea.

    \item It offers the opportunity to network with other scientists in a
        unique and stimulating environment.

    \item It provides practical education about how to solve problems that
        arise when outside assistance is not readily at hand.

    \item It emphasizes the importance of practicing expeditionary behaviour
        whereby adherence to safe procedures and maintenance of clear and open
        communication are paramount.

\end{itemize}

\p Arrangements to meet the \di{sea time} requirement are the responsibility of
the student in consultation with the \supervisor. Financial costs associated
with meeting the \di{sea time} requirement should be borne either by the
supervisor, a travel grant, or an outside investigator funded to participate in
the research cruise. By convention, ``sea time'' means spending 3 or more
nights at sea on a research vessel that is conducting active research. Students
are required to submit a \formST to the \GS after completion of this
requirement, and the \curcom must approve completed forms.

\p In a limited set of extenuating circumstances, students may have the
sea-time requirement waived and replaced with commensurate experience. For
example, students facing physical or mental challenges to working at sea,
students with extensive experience at sea prior to coming to Dalhousie, or
students with demanding nearshore/intertidal field research projects that limit
time available to go to sea may be granted a waiver of the sea-time
requirement. To apply for a waiver, the student composes a letter to the
\curcom stating the reason for the request and proposing commensurate
replacement experience. An accompanying letter supporting the student's request
must be composed by the \supervisor and signed by the \advcom. The
decision to grant or deny the request for a sea time waiver is the
responsibility of the \curcom.


% A5	SEMINARS
\subsection{Seminars}

\p Students are required to attend the general \di{departmental
seminars}\index{seminars} and to attend, and participate in, the specialty
seminars in their field of interest.  Students who are unable to attend
seminars regularly must have the specific agreement of their \advcom that this
requirement is waived and this must be communicated to the departmental office
in a written memo signed by the \supervisor. It is important to note that
materials presented in these seminars may form part of the questions for the
\QE (Section~\ref{sec:phd_qualifying_examination})
and the \di{PhD Thesis Proposal Defence} (Section~\ref{sec:phd_thesis_proposal}).



 

%A6	CHANGE IN STATUS
\subsection{Status}

\cp

\p Any change in status, such as transfer from MSc to PhD (see
Section~\ref{sec:transfer_to_phd}), a leave of absence, and entrance to another
degree program must be recommended by the \advcom to the \gocom.  Final
approval for a change in status will be made by the \gocom, which will meet as
required to review these requests.  Students should consult the Graduate
Secretary regarding the required documentation.


%A7	ANNUAL PROGRESS REPORT
\subsection{Annual progress report}

\cp

\p According to \FGS regulations, students must submit an \formAPR each year,
one month prior to the anniversary of their admission date. Failure to submit
this report will result in delays in registration and funding.  This report is
submitted by the student through \dalonline, with that system later prompting
the \supervisor and then the \GC for approval.  The system relies on emails, so
students are advised 

\p In addition to the above, students must submit a \formSPRF each year. A
blank version of this document is available from the \GS. This form is a way to
record accomplishments, in more detail than would be used in a CV, and students
are advised to keep this up to date.

%A8	TIME LIMITS FOR COMPLETION OF DEGREES
\subsection{Time limits for completion of degrees}

\cp

\p Graduate students have a maximum period of time within which to complete
their graduate program, 4 years for MSc and 6 years for PhD.  Extensions may be
granted by FGS on the recommendation of the department, along with a
satisfactory Progress
Report\furl{http://academiccalendar.dal.ca/Catalog/ViewCatalog.aspx?pageid=viewcatalog&catalogid=58&chapterid=2678&topicgroupid=10850}.

\section{Supervision}

%A9.3	SUPERVISOR
\subsection{Supervisor}

\cp

\p The Supervisor provides direct supervision of the student's research and is expected to

\begin{itemize}

    \item Advise the student on the choice of a research topic, on the possible
        directions to emphasize during the work, and the point at which it
        should be concluded.

    \item Supervise the research, as well as the preparation of the proposal,
        progress reports and thesis.

    \item Ensure that the resources necessary to the thesis project are made
        available and that any necessary skills are acquired by the student.

    \item Assess the student's progress, evaluating strengths and weaknesses.

    \item Provide current updates of progress to the student's file.

    \item Ensure that the student is funded

    \item Select thesis examining committee members (following the guidelines
        for composition of the various examining committees).

\end{itemize}

\p A \supervisor is changed only with the approval of the Department. This may
arise by mutual agreement within the advisory committee or at the request of
the student (to either the \GC or the Chair of the Department).

\p In order to take advantage of the special expertise and facilities
available, the role of \supervisor is not restricted to regular faculty members
of the Department; see the \FGS
website\furl{http://academiccalendar.dal.ca/Catalog/ViewCatalog.aspx?pageid=viewcatalog&catalogid=58&chapterid=2678&topicgroupid=10852}
for details. However, students who have a \supervisor who is not a full time
faculty member in the Department of Oceanography, must also find a regular
member of the department who can act as a \di{internal supervisor}. The
responsibilities of an \di{internal supervisor} include:

\begin{itemize}

\item Advising on academic requirements, course load, waiver of courses,
    general departmental affairs, etc.

\item Ensuring that the student has suitable office and laboratory space.

\item In consultation with the student, selecting a \supervisor and \advcom
    appropriate to the student's research interests (In many cases the chair
    will also be the supervisor.)\fixme{who is the chair?}

\item Ensuring that adequate reports and records of committee meetings are kept
    and submitted to the appropriate departmental files, and that necessary
    decisions as to the academic program, (e.g. thesis proposal), are made on
    an appropriate time-scale.

\item Assessing the progress of the student both in academic work and research.

\end {itemize}
 

%B.	MSc  PROGRAM REQUIREMENTS




%A9	THESIS ADVISORY COMMITTEE
\subsection{\label{sec:advisory_committee_structure}Advisory Committee structure}

\cp

\p Advisory committees are formally sub-committees of the Department appointed
both to provide expert advice to students and to evaluate and report on student
progress.  Members of \advcom must either hold faculty positions at Dalhousie
University, or have an adjunct status registered with \FGS.  The \advcom
structure differs for MSc and PhD, as explained in
Sections~\ref{sec:msc_advisory_committee_structure} and
\ref{sec:phd_advisory_committee_structure}, respectively.

\p The \advcom must be formed within one month of the start of the student's
program, and it may be changed through the progress of a student's program.
The \supervisor chooses committee members, in consultation with the student.

\p Students must inform the department of \advcom structure (and any changes to
that structure) by filling out a \formGSP and submitting it to the \GS. 


% above was: sections B and C


% \p Committees are changed by mutual consent within the committee or by the
% Chair of the committee and student in consultation with the \GC or Chair of the
% Department. See the \GS for procedures on reporting these changes.

%A9.1	ADVISORY COMMITTEE MEETINGS
%\subsection{Advisory committee meetings}
%
%\cp

\subsection{Advisory committee meeting schedule}
\cp \p There must be no less than two advisory committee meetings held per
academic
year\furl{http://academiccalendar.dal.ca/Catalog/ViewCatalog.aspx?pageid=viewcatalog&catalogid=58&chapterid=2678&topicgroupid=10852},
but meetings may be more frequent, as required by the student and/or
supervisor. It is the responsibility of the student to ensure that committee
meetings are held. The \supervisor will ensure that brief minutes of these
meetings are recorded and are filed with the \GS.

\p \fixme{DK to GOC: my notes suggest deleting this paragraph and the itemized
list below it, but now I think it should be kept.  Do others agree to keep it?}
Following approval of the thesis proposal by the \advcom, the student should
have several meetings in addition to the regular committee meetings as follows:

\begin{itemize}

\item An interim progress meeting to verify that the research is on track.

\item A final progress meeting at which the committee agrees that sufficient
    research has been conducted, and that thesis writing may proceed.

\item Individual meetings with advisory committee members to receive input on
    drafts or chapters.

\item A meeting at which the advisory committee agrees that the thesis is
    defensible and that a defence may be scheduled. Alternatively, the advisory
    committee may request further revision prior to approving a defence.

\end{itemize}


% %A9.2	COMMITTEE APPROVAL
% \subsection{Committee approval}
% 
% \cp

\p At several points in a student's program the approval of the advisory
committee is essential:

\begin{itemize}

    \item Initial discussion of proposed research; this should occur as early
        as possible in the student's program.

    \item Approval of research plan and direction for thesis proposal (see
        sections~\ref{sec:msc_thesis_proposal}
        and~\ref{sec:phd_thesis_proposal}. \vote{DK to GOC: my notes suggest
        deleting this item, but I suggest keeping it. OK?}

    \item Acceptance of a thesis proposal.

    \item Change of status, such as advancement to the PhD program (see
        section~\ref{sec:transfer_to_phd}).

    \item The acceptance of the thesis, in final form, as being ready for defence.

\end{itemize}


%B.	MSc  PROGRAM REQUIREMENTS

\section{\label{sec:msc_program_requirements}MSc program requirements}

% B1	MSc COURSE REQUIREMENTS

\subsection{\label{sec:msc_course_requirements}MSc course requirements}

\cp

\p Students must complete OCEA 5000 and an additional 6 credit hours from core
courses, namely \course{5110}, \course{5120}, \course{5130} and \course{5140}.
(Students who took the core classes at the 4000 level as undergraduates, having
achieved grades of ``A'' or ``A+'' need not to take them again at the 5000
level.) In addition to the above, additional courses may be required to
strengthen a student's background; the \supervisor and \advcom advises on this
matter.

% \p MSc students must complete a minimum of 5 half-credits at the 5000-level or
% higher, at least three of which must be chosen from the introductory core
% courses (refer to section~\ref{sec:courses}).

\p Any student who anticipates a \index{transfer from MSc to PhD} transfer from
the MSc program to the PhD program should complete the course-work and other
requirements as listed in the PhD program requirements (see
Section~\ref{sec:phd_program_requirements}).

%B2	MSc COMMITTEE STRUCTURE
\subsection{\label{sec:msc_advisory_committee_structure}MSc Advisory Committee structure}

\cp
        
\p As noted in Section~\ref{sec:advisory_committee_structure}, the \advcom must
be formed within one month of the start of the student's program.  This
committee consists of at least three members. There must be two full-time
faculty members (with \FGS status) from the Oceanography Department.  If a
regular faculty member serving on the committee leaves the Department, reducing
the number of regular members to below two, then a replacement with another
regular faculty member must be made. At least one of the members of the \advcom
should be from a sub-discipline other than the student's own.


%B3	MSc THESIS PROPOSAL
\subsection{\label{sec:msc_thesis_proposal}MSc thesis proposal}

\cp

\p MSc students are expected to produce an approved \di[thesis
proposal!MSc]{thesis proposal} within one year of enrolling in the program.

\p This document serves primarily as a well-reasoned course of action, and not
as an exhaustive literature review nor as a deeply detailed description of
methods. This should be under fifteen pages of text, double-spaced and
inclusive of diagrams and citations.\vote{DK to GOC: I made the length more
specific. OK?}

\p The \di[thesis proposal!MSc]{proposal} is developed in consultation with the
\supervisor and \advcom. The scope of the research should be such that the
program can be completed within one year.

\p Students may be required to defend their thesis proposals, at the discretion
of their \advcom.  Such defences normally occur during a meeting of the
\advcom, unlike in the PhD case, where they are public events.

\p The \supervisor will notify the Oceanography Office that the \di[thesis
proposal!MSc]{proposal} requirement has been satisfied and a copy of the
approved proposal, signed by both student and \supervisor, must be placed on
file with the \GS.



%B4	MSc THESIS
\subsection{\label{sec:msc_thesis}MSc thesis}

\cp

\p The MSc thesis requirements are subject to detailed regulations of the
Faculty of Graduate
Studies\furl{http://academiccalendar.dal.ca/Catalog/ViewCatalog.aspx?pageid=viewcatalog&catalogid=58&chapterid=2678&topicgroupid=10853}.

\p The MSc thesis should report original research and be in a satisfactory and
consistent literary form. Faculty of Graduate Studies thesis formatting
requirements are available on their
website\furl{http://academiccalendar.dal.ca/Catalog/ViewCatalog.aspx?pageid=viewcatalog&catalogid=2&chapterid=399&topicgroupid=1435}.

\p The oral defence of the thesis is open to all members of the department and
to other interested persons. Notices of the thesis defence, including an
abstract, will be posted and distributed by the \GS to all members of the
Oceanography Department.

%B4.1	MSc EXAMINING COMMITTEE
\subsection{\label{sec:msc_examining_committee}MSc examining committee}

\cp

\p It is the {\supervisor}'s responsibility to select the examining committee
members, following the committee structure guidelines. The defence may
\emph{not} proceed in violation of the below composition of the examining
committee due to short-term absence of committee members.

\p The examining committee will consist of a minimum of 5 members:\fixme{DK:
I'll look up the FGS rules to check that what is written below is OK. I'll also
add a link to the site where FGS gives those rules.}

\begin{itemize}
    \item 3 members of the advisory committee
        \begin{itemize}
            \item Supervisor(s)
            \item Full-time department faculty member
            \item One other member
        \end{itemize}

    \item Departmental Representative, who chairs the meeting (without asking
        questions or voting on the outcome) and writes a report on the outcome.
        This person must be a regular faculty member, and not a member of the
        \advcom.  (The \GS consults with the \GC on the selection of the
        Departmental Representative, who is often a member of the \gocom.)

    \item External examiner from outside the advisory
        committee\footnote{Adjunct professors may fulfill the requirement of an
        external examiner.}

\end{itemize}


%B4.2	MSc THESIS DEFENCE TIMELINE
\subsection{MSc thesis defence timeline}

\cp

\p Prior to proceeding with scheduling the defence, the thesis (in essentially
final form) should be distributed to the \supervisor and committee to be
reviewed for suitability and final approval.\fixme{DK: I will check into
whether FGS has a timeline, and refer to it here, if so.}

\fixme{If the list below is on the \FGS website, then I think we should just
refer to that. It is difficult having to keep a lot of text synchronized with
\FGS, and it is very problematic if the rules drift apart.}

\begin{itemize}
    \item Six weeks prior to defence:

        \begin{itemize}

            \item Have a format review done by FGS before the thesis is
                distributed to your examining committee\fixme{why ``before''?
                I don't think committee members care, and they might appreciate
                having more time to read.}

            \item Submit an MSc Examination Information Form to the Graduate
                Secretary

            \item Submit a Thesis Binding Submission Form and payment to the
                \GS

            \item Submit an electronic copy of the abstract to the Graduate
                Secretary

            \item Provide copies of the thesis to the \supervisor for
                distribution to the examining committee\fixme{DK: this seems
                crazy to me. There are similar rules for PhD, with the \GC
                being told to distribute; I've not seen that done once in 30
                years}

        \end{itemize}

    \item Four weeks prior to defence:

        \begin{itemize}

            \item Notify the \GS of any Audiovisual requirements

        \end{itemize}

    \item Ten working days prior to defence:
        \begin{itemize}

            \item Submit a printed copy of the thesis to the \GS

        \end{itemize}

    \item Following a successful Defence:

        \begin{itemize}

            \item Have Examining Committee Members sign the signature
                page.\fixme{Does such a form still exist?}

            \item Submit required changes to the \supervisor within the
                specified timeframe.

            \item Have a final format check of the thesis done by \FGS.

            \item Electronically submit the final version of the thesis via
                DalSpace\furl{http://dalspace.library.dal.ca/}.

            \item Submit original completed forms to FGS (National Library of
                Canada Form, Title Page of Thesis, Signature page with original
                signatures, Copyright Page, Ethics Pages (if applicable),
                Student Contribution to Manuscripts (if applicable).

            \item Submit final copies of thesis to the \GS for binding (include
                copies of the signed signature page, and the Copyright Page
                with original signatures).\fixme{Do we still need to require
                this? Why should the student have to pay for binding? Maybe the
                student should submit in PDF and the department should print,
                if it wants to. Students submit PDF to \FGS and a nation-wide
                agency, anyway, so that is now the official copy.}

            \item Complete the \FGS Exit Survey

        \end{itemize}

\end{itemize}


%B4.3	MSc THESIS DEFENCE OUTCOME
\subsection{MSc thesis defence outcome}

\fixme{DK to GOC: my notes say to delete this whole section. But I wrote
``probably chop'' instead of ``chop'' so we should revisit this.  My vote today
would be to delete this, and to refer to \FGS rules, to avoid problems that
arise when rules drift apart.}

\cp

\p Outcomes: All theses are either approved or not approved. The categories are:

\begin{itemize}

    \item Approve as submitted

    \item Approved pending corrections and a clear timetable for completion
        (normally within one month)

    \item Rejected but with permission to re-submit a revised thesis for
        re-examination with a clear timetable for completion (within one year)

    \item Rejected outright

\end{itemize}

\p A simple majority determines the outcome, based on all the examiners except
the Departmental Representative, who will vote only in the event of a tie.

\p Following the defence, the candidate will receive a letter from the Chair of
the examining committee indicating the nature of any corrections to be made and
the time frame within which they are to be completed. In the event of an
unsuccessful defence, an explanation of the negative outcome is provided. A
copy of the report will be provided to the \GS.

\subsection{\label{sec:transfer_to_phd}Transfer to PhD}

\cp

% \p A request to \index{transfer from MSc to PhD} transfer from the MSc program to the PhD program should be
% completed no later than one month prior to the anniversary of the student's
% admission date.

\p An MSc student may transfer to a PhD program without completing a MSc.
Students wishing to transfer should provide a letter to their advisory
committee stating the reason for the transfer, and justification that they are
qualified for the PhD program. A one-page explanation is sufficient. The
student's \advcom is required to meet, and the \advcom may recommend
advancement to the PhD program.  A letter from the \supervisor, noting the date
when the \advcom met and justifying the advancement of the student to the PhD
program, should then be submitted to the \GS for discussion by the \gocom,
which makes the decision on whether the student can transfer. Once this
approval is granted, the students must fill in a \di{Graduate Student Program
Update
Form}\furl{https://www.dal.ca/content/dam/dalhousie/pdf/fgs/currentstudents/Graduate_Student_Program_Update_Form.pdf}
to indicate the change of program.

%\p The decision categories for the transfer to the PhD are:
%\begin{itemize}
%    \item Approved
%     \item Not approved
% \end{itemize}

%\p The decision categories for the transfer to the PhD are: ``Approved,'' ``Not
%approved,'' and ``Deferred.''
% \begin{itemize} \item Approved \item Not approved \item Deferred
% \end{itemize}

\p Students anticipating a transfer from MSc to PhD should take the required
core courses and take the qualifying examinations (see
Section~\ref{sec:phd_qualifying_examination}) in the first year following
completion of those required core classes.\fixme{DK to GOC: I may have mangled
the timing here, in my edits. Let's discuss, please.}

% above referred to C3 for QE

%C.	PhD  PROGRAM REQUIREMENTS
\section{\label{sec:phd_program_requirements}PhD program requirements}

%C1	PhD COURSE REQUIREMENTS
\subsection{\label{sec:phd_course_requirements}PhD course requirements}

\cp

%\p PhD students must complete a minimum of 6 half-credits at the 5000-level or
%higher, at least two of which must be chosen from the introductory core courses
%outside the student's sub-discipline (refer to section~\ref{sec:courses}).


\p Students must complete at least 12 credit hours at 5000-level or higher of
OCEA courses (OCEA \course{5000} does not count towards this requirement), 9 of
which from core courses, \ie OCEA \course{5110}, OCEA \course{5120}, OCEA
\course{5130}, and OCEA \course{5140}. For students with an MSc from Dalhousie
Oceanography, core courses taken at MSc level can be waived\fixme{Q for LL: is
the word ``waive'' correct? And what is the procedure?} for the PhD
requirement. For students wit an MSc from other departments or institutions:
core course equivalencies will be assessed on a case-by-case basis\fixme{Q:
assessed by whom?}). Additional courses may be required to strengthen a
student’s background.\vote{DK to GOC: I removed ``in basic science'' from AM's
text, since it doesn't mean much and is incorrect, since the courses are often
advanced. OK?}.

% above referred to A2.1 originally

\p Courses are chosen in consultation with the \advcom. The various
sub-disciplines have individual course requirements, some of which exceed the
minimum requirements, as explained below.

%C1.1	BIOLOGICAL OCEANOGRAPHY GUIDELINES
%\subsubsection{Biological Oceanography guidelines}

\p The normal course requirements for a PhD in Biological Oceanography are:

\begin{itemize}

    \item Core Classes: Biological Oceanography (\course{5140}), Chemical
        Oceanography (\course{5130}), and Physical Oceanography (\course{5120}).

    \item A minimum of one other half-credit class drawn from Geological
        Oceanography (\course{5110}) or a suitable Atmospheric Science
        course.\vote{DK to GOC: I propose deleting AtmSci. OK?}

    \item Other classes, as required by the \advcom.

\end{itemize}

%C1.2	CHEMICAL OCEANOGRAPHY GUIDELINES
%\subsubsection{Chemical Oceanography guidelines}

\p The normal course requirements for a PhD in Chemical Oceanography are:

\begin{itemize}

    \item Core Classes: Chemical Oceanography (\course{5130}), Geological
        Oceanography (\course{5110}), Physical Oceanography (\course{5120}),
        and Biological Oceanography (\course{5140}).

    \item Advanced Chemical Oceanography (modular) (\course{5290}). \fixme{5290
        not offered since 2007/9}

    \item Other classes, as required by the \advcom.

\end{itemize}

%C1.3	GEOLOGICAL OCEANOGRAPHY GUIDELINES
%\subsubsection{Geological Oceanography guidelines}

\p The normal course requirements for a PhD in Geological Oceanography are:

\begin{itemize}

    \item Core Classes: Geological Oceanography (\course{5110}) and two other Core Class.

    \item Other classes, as required by the \advcom.

\end{itemize}

%C1.4	PHYSICAL OCEANOGRAPHY GUIDELINES
%\subsubsection{Physical Oceanography guidelines}

\p The normal course requirements for a PhD in Physical Oceanography are:

\begin{itemize}

    \item Core Classes: Physical Oceanography (\course{5120}) and two other
        Core Class.

    \item Advanced Classes: Fluid Dynamics (\course{5311}), Time Series
        Analysis (\course{5210}), Ocean Dynamics (\course{5221}), Estuary,
        Coast and Shelf Dynamics (\course{5222}).

    \item Other Classes, as required by the \advcom, \eg Numerical
        Modelling (\course{5220}), Ocean Waves (\course{5223}), Introduction to
        Acoustical Oceanography (\course{5250}), Advanced Marine Particles
        (\course{5293}).

\end{itemize}

%C2	PhD COMMITTEE STRUCTURE
\subsection{\label{sec:phd_advisory_committee_structure}PhD committee structure}

\p The \advcom must be formed by October 1st of the first year of
study.\fixme{Need to restate in terms of months in program.}

\p The PhD \advcom consists of at least four members. There must be two
full-time faculty members from the Oceanography Department (not Adjunct
Professors)\fixme{Need new wording on adj-prof} on the \advcom. If a regular
faculty member serving on the committee leaves the department, a replacement
with another faculty member must be made.

\p One member will be from another sub-discipline, and it is desirable that at
least one member of the committee be from outside the Department.


%C3	PhD QUALIFYING EXAMINATION
\subsection{\label{sec:phd_qualifying_examination}PhD Qualifying Examination}

%C3	PhD QUALIFYING EXAMINATION
%\subsection{\label{sec:phd_qualifying_examination}PhD Qualifying Examination}


%  \texttt{meetings/20160617/20160617\_meeting\_notes.jpg}
%  
%  \texttt{meetings/20160708/20160708\_meeting\_notes.jpg}
%  
%  \texttt{marked\_up\_versions/bo\_qe\_20160708.pdf}

\cp

\p The \QE (\qe, henceforth) is an oral examination based on a reading list
tailored to the PhD candidate's general research area. It is a Department
process, not a Supervisory Committee process. The Department delegates its
administration to the regular Professors in the student's subdiscipline(s).
(If the student is co-supervised by an Adjunct Professor or a cross-listed
Professor, then that individual also takes part in the \qe.)\discuss{DK: do I
have this exception correct?}

\p The \qe should be completed between months 9 and 12 of the PhD program.
Students who have transferred from the MSc program should take the QE within 12
months of their transfer.\discuss{DK: did we say that exceptions could be
granted by the CurrComm?}

\p The details and timing are as follows.

\begin{enumerate}

    \item The Candidate prepares a 1 to 3 page\vote{DK: there has been some
        discussion (by BB and DK) of shortening this further, even to just
        keywords or a few senetences. I propose to put a flag in the document
        to invite discussion. OK?} summary of their general research interests
        and forwards it to their Supervisor for approval.  The summary is
        intended to guide faculty members in compiling a reading list.

    \item The Supervisor forwards the summary to faculty members in the
        student's subdiscipline(s), along with a request that they, and the
        Supervisor, each provide one or two papers (book chapters are also
        acceptable) to make up a reading list that is relevant to the summary.
        The list should not contain anything written by a member of the Examining
        Committee.\vote{DK to GOC: if we don't agree on this, there should be an
        indication of this in the document, to invite wider opinions.}
        At least three faculty members, including the supervisor, must
        contribute to the reading list. The list must contain between 4 and 6
        papers\discuss{DK: I wrote in ``6'' here \ldots I don't recall if we
        set a limit, and I've seen more than 6, but I don't want us sliding
        into a system where every prof gives 2 papers so we can have 10 papers
        in total, potentially}. The suggested papers must be forwarded to the
        Supervisor within 2 weeks of the request.\discuss{I don't understand
        this forwarding.}

    \item The \supervisor vets the papers to ensure appropriateness, and then
        distributes the list to the Candidate and the relevent
        faculty.\discuss{DK to GOC: we need to discuss whether the \supervisor vets
        the papers, and also who organizes the whole procedure. I could see
        and argument that papers be sent to the \GS for organization, and that
        seems to be the practice in ChemOce. But of course the \GS cannot vet
        papers. And I don't think we can ask the \GC to vet them, either. So
        there is an open question of whether, or how, to decide on
        apprriateness. In the days of ``old comps'' there was a committee, but
        the work was 5X less in scope since students wrote common exams.}
            
    \item The qualifying exam must be held within 6 weeks of the reading list
        being distributed. 

    \item It is the responsibility of the \supervisor to form the Examining
        Committee and schedule the exam. The committee must include a Chair
        drawn from the student's subdiscipline(s), at least three members of
        the subdisicpline, including the \supervisor, and an External Examiner
        drawn from the Oceanography faculty outside the sub-discipline(s) and
        preferably in a sub-discipline closely related to the candidate’s
        research interests. \emph{In exceptional circumstances an External from
        outside the department may be nominated by the \supervisor. The
        nomination must be approved by at least three BO faculty members. The
        nominee must have no conflict of interest with the Candidate’s research
        programme.}\discuss{I think the highlighted text should be deleted.}

    \item The form of the Examining Committee must be approved by at least
        three members of the subdisciplinary faculty no later than 3 weeks
        prior to the date of the exam. Once approved, the \supervisor forwards
        the research summary, the reading list, and these guidelines to the
        External Examiner and to the Chair. The External is not required to
        read the listed papers.\discuss{Is anyone ``required''? My impression,
        having been at a lot of these, is that examiners often have not read
        the other papers, especially if they are specialized (which I think
        is a mistaken choice of paper, but that's just my own view).}
        \vote{Does the external provide a paper? I think there is a lot of
        merit in that, and I vote ``yes''.} At the same time, the Candidate is
        advised of the committee membership. 

    \item The Chair moderates the exam (see guidelines below) and does not
        participate in questioning. Ideally, the exam will last no longer than
        1.5~hours. The exam begins with a 20-minute presentation by the
        Candidate who will attempt to demonstrate an understanding of the
        concepts and contents of the papers, perhaps highlighting common
        threads or links to the rest of the literature or the student's plans,
        as appropriate.

    \item The committee then questions the Candidate about oceanography and
        topics related to the Candidate’s research interests where the entry
        points into questions and discussions are to be based on the reading
        material that was provided to the Candidate and Candidate’s
        presentation. The exam is not about the proposed research.

    \item Upon the completion of questioning, the committee meets \emph{in
        camera} to agree on the exam outcome and recommendations to the
        Candidate and their \advcom, as appropriate.

    \item The possible outcomes of the Qualifying Examination are:

        \begin{enumerate}
            \item The candidate passes, without extra conditions.

            \item The candidate passes, but is informed of weaknesses that
                should be addressed during the PhD work, in the form of
                courses, audits, or directed studies.

            \item The candidate is required to take a written examination where
                in the format and time line for the exam will determined by the
                examining committee while meeting \emph{in camera}. This
                examination will be based on the topics that arose during the
                oral examination and will not exceed three hours. 
                

            \item The candidate is transferred to, or continued in, the MSc
                programme.  \vote{DK to GOC: I added this, based on discussions
                with BB and others about the fact that this is a formal
                examination, and we had no ``failure'' mode.  OK?}

        \end{enumerate}

    \item The possible outcomes of a followup written examination are:

        \begin{enumerate}

            \item The candidate passes without extra conditions.

            \item The candidate passes, but is informed of weaknesses that
                should be addressed during the PhD work, in the form of
                courses, audits, or directed studies.

            \item The candidate is transferred to, or continued in, the MSc programme.

        \end{enumerate}

    \item The Chair shall report in writing to the Department Chair and the \GC
        to summarise the procedure and outcome of the exam.  Copies of the
        report will be forwarded to the Candidate and the Examining Committee.

\end{enumerate}

\p Guidelines for the Chair of {\QE}s.

\begin{itemize}

    \item The Exam:

        \begin{itemize}

            \item Introduce the Candidate, the External and the Examining
                Committee as needed.

            \item Summarise exam procedure: presentation, questioning, in
                camera meeting, decision and recommendation.

            \item Keep presentation by the Candidate to 20 minutes; at 25 warn a cut off.

            \item Announce the questioning order, beginning with External and
                ending with \supervisor.

            \item Limit the examination to two rounds of questioning, the
                second shorter than the first.

            \item Keep questioners on time, with 10 minutes (15 minutes for the
                External Examiner) on the first round, and less on the second
                round.

            \item Keep notes on the time used by questioners.

            \item Do not let questioners go astray--keep everyone on
                track.\discuss{I don't know what this means, or how to
                accomplish it}

            \item Discourage too much discussion back and forth between members
                of the Examining Committee; questions should be directed at the
                student and the student should be permitted to respond.

            \item Do not allow the \supervisor to answer questions directed at student.

            \item Keep the process just, calm and professional.

            \item The entire exam should last no longer than 1.5 hours.\vote{DK
                to GOC: we should use history to guide this.  And a nominal
                calculation is: 20 min + 10min/examinar*5examiner=2 hours just
                for one round, without in-camear.  I think stating 2.5 hours
                for the non-in-camera part is more sensible. OK?}

            \item Keep notes on significant issues during questioning.

        \end{itemize}

    \item The \emph{in camera} meeting:

        \begin{itemize}

            \item Remind the Examining Committee of the Defence outcome options
                (see above).

            \item Ask for recommendation and comments from the External first.

            \item Ask remainder of the Committee, in order of questioning, for
                their recommendation and comments.

            \item Encourage either a unanimous decision, or at least a
                consensus one. In the case of a split decision, the Chair casts
                the deciding vote.\discuss{DK to GOC: note that I say the chair
                can cast a deciding vote. The point is we cannot have a
                non-outcome. This is an examination; after all, and students
                cannot be told that the professors cannot decide whether they
                pass or fail.}

            \item Keep notes on significant issues that should be included in
                the report sent to the Chair and the departmental \GC.

        \end{itemize}

    \item The Report:

        \begin{itemize}


            \item The examination Chair shall submit a written report to the
                Chair of the Department and the departmental \GC.  The report
                shall include a written description of the procedure and
                outcome of the exam and associated recommendations. The report
                shall be included in the Candidate’s departmental file.

        \end{itemize}

\end{itemize}


% C4	PhD THESIS PROPOSAL AND DEFENCE
\subsection{\label{sec:phd_thesis_proposal}PhD thesis proposal and defence}

\cp

\p PhD students are expected to complete the \di[thesis proposal!PhD]{thesis
proposal} and examination of the proposal within 20 months of enrolling in the
graduate program.

\p The advisory committee sets the exact timing.

\p In order to proceed with the \di[thesis proposal!PhD]{proposal}, the student
will have completed the required first year courses, have a named \supervisor
and \advcom, and received approval from the supervisor.

\p The requirement consists of a written \di[thesis proposal!PhD]{thesis
proposal} and a public oral defence.  The written proposal will include an
overview of the research topic and relevant background material, and a plan for
carrying out the PhD research.  The oral portion is a defence of the proposal,
including other relevant areas of Oceanography. The topics covered in the
defence may relate not just to the planned research, but also to wider
oceanographic issues, as deemed appropriate by the defence committee.

\p The defence is open to attendance by all interested persons, except for the
in-camera portion in which the decision is made by the examining committee.
Notices of the thesis proposal defence, including an abstract will be
distributed to all members of the Oceanography Department and will be posted.


%C4.1	EXAMINING COMMITTEE
\subsubsection{Examining committee}

\cp

\p The examining committee is selected by the \advcom, and consists of members
from that committee, external members, and a departmental representative. The
\GS will arrange for the Departmental Representative (normally the \GC or a
designate).

\p The Chair of the PhD proposal defence is independent of the examining
committee. The Departmental Representative normally performs this duty.

\p Students must make the proposal available in the Department of Oceanography
office at least 10 working days prior to the oral defence.  The proposal
typically receives input from the \supervisor or others prior to its
submission. Approval from the \advcom is required to proceed with the oral
defence.

%C4.2	PROPOSAL TIMELINE
\subsubsection{Proposal timeline}

\cp

\p It is the student's responsibility to initiate the proposal process and
develop the proposal in coordination with the \supervisor and \advcom. Two
terms are more than adequate to complete this requirement. The scheduling of
the proposal defence should be planned well in advance so that it can be
carried out prior to the 20th month of the program.

\p There is no restriction on early completion of the proposal and oral exam (e.g.
in the summer between first and second years or fall term of second year).

\p Because the proposal is a formal requirement of the Department, it carries the
same weight as courses, thesis defence, etc. Failure to complete the proposal
within the required time frame reflects poorly on the student and can endanger
standing in the program.


%C4.3	PROPOSAL OUTCOME
\subsubsection{Proposal outcome}

\cp

\p Based on both the written proposal and oral defence, the \supervisor will
notify the \GS, \GC, student, and examining committee in writing of one of the
four possible outcomes:

\begin{itemize}
\item Continuation in the PhD program
\item Permission to re-defend
\item Transfer to the MSc program
\item Withdrawal from the program
\end{itemize}

\p Revisions, and/or a re-defence must be completed prior to the end of the exam
period in the winter term. Following the defence, a copy of the approved
proposal must be placed on file in the Oceanography office.


%C5	PhD THESIS
\subsection{\label{sec:phd_thesis}PhD thesis}

\cp

\p The PhD thesis should report original research of high caliber carried out by
the student. It should be of such value as to merit publication and be in
satisfactory literary form, with clearly presented figures and other supporting
material. The thesis should be presented first, in draft form, to the
supervisor who will provide initial comments and approve distribution to the
rest of the committee. Typically, the writing phase is very intensive, and
involves working closely with the \supervisor and \advcom. In many cases, the
student writes journal articles that link closely with the thesis chapters, and
the Faculty of Graduate Studies has conventions on how this is to be handled.
As the thesis reaches completion, the advisory committee will meet to discuss
any final changes to the thesis and sign the PhD Thesis Submission Form; this
is just one of a series of steps that the student must keep in mind (see
section~\ref{sec:phd_thesis_defence_timeline}). 


%C5.1	PhD EXAMINING COMMITTEE
\subsubsection{PhD examining committee}
 
\cp

\p It is the {\supervisor}'s responsibility to select the examining committee
members, following the committee structure guidelines.

\p The examining committee will consist of a minimum of 5 members:

\begin{itemize}

\item 3 members of the advisory committee

    \begin{itemize}

        \item Supervisor(s)

        \item Full-time department faculty member

        \item One other member

    \end{itemize}

\item A Departmental Representative, from outside the committee.  The Graduate
    Secretary will arrange for the Departmental Representative, normally the
    Graduate Coordinator or a designate.

\item External examiner (from outside the University)

    \item Chair\footnote{The Chair of the PhD thesis defence is independent of
        the examining committee. The Faculty of Graduate Studies will appoint
        the Chair.}

\end{itemize}


%C5.2	PhD THESIS DEFENCE TIMELINE
\subsubsection{\label{sec:phd_thesis_defence_timeline}PhD thesis defence timeline}

\cp

\p Students should be aware of the time scales required by FGS and Departmental
regulations, which stretch over a period of more than eight months before the
defence.

\p Prior to proceeding with scheduling the defence, the thesis (in essentially
final form) should be distributed to the \supervisor and committee to be
reviewed for suitability and final approval.


\begin{itemize}
    \item Twelve weeks prior to defence:

        \begin{itemize}

            \item Submit Request to Arrange an Oral Defence Form, listing 3
                choices for external examiners along with CV for external
                examiner of first choice to the \GS.

        \end{itemize}

    \item Six weeks prior to defence:

        \begin{itemize}

            \item Have a format review of the thesis done by FGS before it is
                distributed to the examining committee

            \item Submit a PhD Thesis Submission From and copy of the thesis
                for the External Examiner to the \GS 

            \item Submit a PhD Examination Information Form to the Graduate
                Secretary

            \item Submit a copy of the student's CV to the \GS 

            \item Submit a Thesis Binding Submission Form and payment to the
                \GS

            \item Submit an electronic copy of the abstract to the Graduate
                Secretary

            \item Provide copies of the thesis to the \supervisor for
                distribution to the internal examining committee members

        \end{itemize}

    \item Four weeks prior to defence:

        \begin{itemize}

            \item Notify the \GS of any Audiovisual requirements

        \end{itemize}

    \item Two weeks prior to defence:
        
        \begin{itemize}

            \item Submit a printed copy of the thesis to the \GS 

        \end{itemize}


    \item Following Defence:

        \begin{itemize}

            \item Have Examining Committee Members sign the signature page

            \item Submit required changes to the thesis within the specified
                timeframe

            \item Have a final format check done by FGS

            \item Electronically submit the final version of the thesis via
                DalSpace\footnote{DalSpace
                \href{http://dalspace.library.dal.ca/}.}.

            \item Submit original completed forms to \FGS (National Library of
                Canada Form, Title Page of Thesis, Signature page with original
                signatures, Copyright Page, Ethics Pages (if applicable),
                Student Contribution to Manuscripts (if applicable)

            \item Submit final copies of thesis to the \GS for binding (include
                copies of the signed signature page, and the Copyright Page
                with original signatures)

            \item Complete Exit Survey

        \end{itemize}

\end{itemize}

\p Some of the above deadlines are set by the Oceanography Office. See the FGS
website for advice on preparing for a Doctoral defence\footnote{FGS Theses and
Defences
\href{http://www.dal.ca/faculty/gradstudies/currentstudents/thesesanddefences/defense.html}.}.


%C5.3	PhD THESIS DEFENCE OUTCOME
\subsubsection{PhD thesis defence outcome}

\cp

\p The student opens the examination with a 20-minute presentation of the thesis.
Other faculty, students and public may be present, and ask questions, but the
critical decisions are made in private and are the responsibility of the
examining committee.

\p Following the defence, the candidate will receive a letter from the Faculty of
Graduate Studies indicating the nature of any corrections to be made and the
time frame within which they are to be completed.  In the event of an
unsuccessful defence, an explanation of the negative outcome is provided.  

%C5.4 PhD SEMINAR PRESENTATION
\subsubsection{PhD seminar presentation}

\cp

\p A departmental seminar on the thesis should be presented after successful
completion of the thesis examination.

%D.	FINANCIAL  SUPPORT
\section{Financial  support}

\cp

\p Dalhousie Graduate Fellowships will normally be provided for two years for an
MSc student and four years for a PhD student.

\p Scholarship support is provided directly to some students by the Natural
Sciences and Engineering Research Council (\di{NSERC}), the \di{Killam
Foundation} through Dalhousie, by companies, and by other agencies, both
national and international. Students are encouraged to apply for external
support whenever possible.

\p Full-time students without such special scholarship support are generally
awarded Dalhousie Graduate Fellowships from funds that come partly to the
Department from the University and mainly from individual faculty members'
research grants.

\p Students may also receive extra support by taking on a teaching assistantship,
or by demonstrating in undergraduate laboratories.

\p Students should discuss any additional jobs with their advisory committee to
ensure that the time commitments outside the Department are not excessive, and
do not interfere with program expectations.


%D1	FGS CONFERENCE TRAVEL GRANT
%\subsection{FGS conference travel grant}


\p
Graduate students are eligible to apply for one \di{travel grant} per degree at
Dalhousie (Master's or Doctoral) through the \FGS\footnote{FGS Conference
Travel Grant Application Form and Guidelines
\href{http://www.dal.ca/faculty/gradstudies/funding/grants.html}.}.
Applications must be submitted through the \GS at least one month prior to the
date of travel.  Students must be registered in a graduate program at the time
of application and at the time of the conference. In order to be eligible,
students must present a poster of paper based on the results of their graduate
thesis research at a national or international scholarly meeting or conference.
Departmental approval must be given to these applications.


%E.	INTELLECTUAL  PROPERTY
\section{Intellectual  property}

\cp

\p The Faculty of Graduate Studies is developing\fixme{probably already exists
-- check into it}  a policy on Intellectual Property, which will be available
on the Faculty of Graduate Studies website.  If students and/or faculty have
concerns or doubts about any issue pertaining to any part of Section 6, consult
with your Chair, \GC, or Supervisor, or contact the Faculty of
Graduate Studies for advice. If you feel uncomfortable with approaching your
immediate \supervisor, then go to the next level and ask to be heard in
confidence\footnote{FGS Regulations, section 6.4: Policy on Intellectual
Property
\href{http://academiccalendar.dal.ca/Catalog/ViewCatalog.aspx?pageid=viewcatalog&catalogid=58&chapterid=2678&topicgroupid=10849}.}.

%\elink{http://academiccalendar.dal.ca/Catalog/ViewCatalog.aspx?pageid=viewcatalog&catalogid=58&chapterid=2678&topicgroupid=10849}, section 6.4.}.

%E1	ACADEMIC INTEGRITY
\section{Academic integrity}

\cp

\p Any material submitted by a student at Dalhousie University may be checked
for originality to confirm that the student has not plagiarized from other
sources.  \di[plagiarism]{Plagiarism} is a serious academic offence that may
lead to a failing grade, suspension or expulsion from the University, or even
the revocation of a degree.  It is essential that there be correct attribution
of authorities from which facts and opinions have been derived. Prior to
submitting any paper in a course, or material for a thesis, students should
read the Policy on Integrity in Scholarly Activity\footnote{FGS Regulations,
section 6.3: Policy on Integrity in Scholarly Activity
\href{http://academiccalendar.dal.ca/Catalog/ViewCatalog.aspx?pageid=viewcatalog&catalogid=58&chapterid=2678&topicgroupid=10849}.}.

%\section{Procedures for Student Academic Appeals}
%and Governance of the Ad Hoc Appeal Committee}

\subsection{Introduction}

\cp
\p The procedures detailed below will be followed in cases where a
student wishes to appeal an academic procedure, other than a grade, on the set
of criteria detailed in section 2 below. (Grade appeals are handled by the
University's Registrar.)

\p The regulations and procedures given below cover appeals launched by
both undergraduate and graduate students in the Department of Oceanography at
Dalhousie University.  

\p These procedures do not apply to labor disputes between an employee
(be that person a student, post-doc, etc.) and an employer (faculty, staff) in
regard to working conditions.  Persons wishing to dispute a labor condition
should consult either the Chair of the Department and/or their union agreement,
if applicable, or the laws governing labour relations in Nova Scotia and the
procedures detailed therein. 

\p These regulations do not apply if the cause of the dispute centers
on discrimination based on gender, race, nationality, sexual orientation,
gender identity, official language, or disability.  The complainant in such
cases must contact the Dalhousie Office of Human Rights, Equity \& Harassment
Prevention for procedures in such cases.  Nor are these regulations applicable
to cases of sexual harassment, assault, criminality, etc., where Dalhousie
Security or Law Enforcement should be involved.  

\p There are no appeals on admission decisions, including transfers to
the PhD program.  

\subsection{Basis for an Appeal}
\cp

\p The procedures below apply to the administration of qualifying and
preliminary examinations, comprehensive examinations, thesis proposal defenses,
and Master's thesis defenses.  Results (grades) of such examinations cannot be
appealed as oral portions cannot be regraded.  

\p Appeals related to Doctoral defenses must be directed to Faculty of
Graduate Studies in the first instance. 

\p Appeals of course examination grades (re-grading) are direct through
the University Registrar's Office.
%, and not the Department of Oceanography.
%Some clarification on the method of reassessment is given below in
%section~\ref{sec:grade_reassessment}; otherwise the appellant must follow the
%procedures set by the Registrar. \fixme{should we just leave the rules to the
%registrar, thus axiomatically avoiding contradiction?}
 
\p The grounds for appeal are limited to the following:
\begin{enumerate} \item procedural unfairness \item bias \item irregularity in
        procedure \item inappropriate or unfair expectation(s) \end{enumerate}

``Procedural Unfairness'' means that the method used to administer the academic
process, e.g., an exam, was unreasonably and inherently stacked against any
person attempting the process.  Note: exams and their content, in and of
themselves, are not procedurally unfair.  Differences in how an exam is
administered between the sub-disciplines in the Department also does not
constitute unfairness, as long as the method is applied consistently within
that sub-discipline.  

``Bias'' means that the process was conducted in such a way as to disadvantage the
particular appellant, relative to other persons undertaking the same process.    

``Irregularity in procedure'' means that the process did not follow the
procedures set in the regulations and guidelines governing the actions of the
Department and its faculty.  

``Inappropriate or unfair expectation'' means that the deliverable of the
process could not reasonably be expected of a person who undertakes
assiduously, genuinely, and accurately the steps leading to completion of that
process.   

For all these grounds, the onus is on the appellant to prove, through factual
documentation, that such conditions existed. 

\p A written appeal must be submitted to the Chair of the Department of
Oceanography within 30 days following the event or circumstances being
appealed. 

\p The appeal submission must include:
\begin{enumerate}

\item A description of the exact nature of the appeal, including a summary of
    events and chronology

\item Specific details of the alleged unfairness, bias or irregularity, and any
    other relevant consideration or information

\item The requested resolution of the appeal, which is limited to a reasonable
    academic action(s). 

\end{enumerate}

\p The submission of an appeal will engender the following actions by
the Chair of the Department:

\begin{enumerate}

\item The Chair will contact both the appellant and the person responsible for
the process being appealed to see if an informal resolution is first possible, and

\item Failing an informal resolution, the Chair will constitute an Ad Hoc
Appeal Panel for that specific appeal.  The nature of and procedures for that
Panel are detailed below in section~\ref{sec:the_ad_hoc_appeal_panel}.

\end{enumerate}

\p Decisions of the Ad Hoc Appeal Panel are forwarded to the Chair, who
will communicate the decision in writing to the appellant.

\p Decisions of the Ad Hoc Appeal Panel are subject to further appeal to
the Faculty of Science (undergraduates) or the Faculty of Graduate Studies
(graduate students), or Senate, as regulations in those administrative units
specify. 


\subsection{\label{sec:the_ad_hoc_appeal_panel}The Ad Hoc Appeal Panel}

\cp

\p Decisions on appeals are made by an Ad Hoc Appeal Panel.

\p The Panel does not have fixed membership.  Instead, it will consist
of three faculty members, chosen on a fixed rotating basis from all the faculty
in the Department.  The professor(s) responsible for the process being appealed
will be excluded.  For graduate students, the student's supervisor or
co-supervisors, or for undergraduates, the student's honors supervisor (if
applicable) will also be excluded from the list.

\p An appropriate graduate student (for undergraduate appeals) or
Post-doctoral fellow (for graduate appeals) will constitute the fourth and
final member of the Panel.  Any student on the Panel will not have a personal
or academic conflict of interest with respect to the appellant.  For
undergraduates, the student will be chosen from the graduate students.  For
graduate students, the student member will be chosen from the Post-doctoral
Fellows, without conflict of interest.  

\p The Panel will examine the appeal and determine if a \emph{prima
facie} case exists for the appeal.  (\emph{Prima facie} in this context means
the appeal document contains sufficient evidence to support the stated claim.)
If such a case exists, then the Panel will convene a formal Appeal Hearing at
the first opportunity when all involved, i.e., appellant, the faculty member
who owns the process, and the Panel members can meet to hear verbal arguments
on the appeal, but preferably within three weeks of constitution of the Panel.
If there is no \emph{prima facie} case, the Panel will inform the Chair and the
appeal will be dismissed at the Departmental level.

\p If a Hearing is convened, the Panel will chose a Chair who will
control and direct all communication.

\p The Chair of the Panel will make the faculty owner of the process
aware of the appeal and provide, in a timely manner, a copy of the appeal
document to the process-owner (faculty member).  

\p Before the Hearing, the process owner will provide the Panel with a
written rebuttal to the appeal within 5 working days of the appeal being
received.  (Exceptions will be made to accommodate persons being at sea or
subject to other serious limitations in availability.)  The Panel will forward
that rebuttal to the appellant.


\subsection{The Appeal Hearing}
\cp

\p The Hearing will adhere to the following steps:

\begin{enumerate}

\item The Chair of the Panel will direct the flow of the Hearing, making sure
    both parties are properly heard, and directing questioning;

\item The Chair of the Panel will first invite the appellant to present and
    explain their case; this will be followed by an invitation to the faculty
    member owner for a rebuttal;

\item Neither the appellant nor the owner can ask questions during these
    presentations; however, any Panel member may interrupt the presentation to
    ask questions;

\item At the end of the presentations, the appellant will be invited to direct
    questions to the faculty owner for the purpose of seeking clarifications
    (only); after that is completed, the faculty owner will be invited to do
    the same;  the Panel members may interrupt these questioning sessions at
    any time to ask their own questions; the Chair of the panel will ensure
    that this questioning remains civil and on topic;

\item At the end of the questioning, the Chair of the Panel will ask the
    appellant and faculty owner to leave and convene an \emph{in-camera}
    discussion of the presentations and Appeal Document, the Panel will decide
    on the validity of the appeal and of requested resolution.

\end{enumerate}

\p The Chair of the Panel will communicate, in writing, the decision of
the Panel within 3 working days (except for serious time conflicts such as
noted those mentioned at \ref{sec:the_ad_hoc_appeal_panel}.7 above).

\p In an appeal process, the student has the right to representation.
The student is required to inform the Chair of the Panel, in writing, if s/he
will have a representative at the appeal, or intends to call witnesses. 

\p Witnesses may be called by either party, but only if those witnesses
can testify about direct knowledge of the matter.  No character or indirect
testimony is permitted.  The Panel Chair must be informed 72-hours before a
Hearing if a witness(es) will be called, along with that person's identity and
significance to the Hearing.  The Chair of the Panel will then inform the other
party of the witness(es).  

\p No persons providing solely moral support can be present at a
Hearing.  A person providing aid to someone with a disability is permitted, as
is a translator, but such persons cannot testify at a Hearing.


\subsection{Supplementary Points}

\cp

\p The Ad Hoc Appeal Panel has no jurisdiction to hear student appeals
on a matter involving a requested exemption from the application of
Departmental, Faculty or University regulations or procedures, except when
irregularities or unfairness in the application thereof is alleged.  This means
that only procedural issues, and not the merits of the regulations, are subject
to appeal.

\p The Ad Hoc Appeal Panel may not render decisions counter to
Departmental, Faculty or University regulations, nor can it make decisions that
go beyond strictly academic matters, e.g., financial or administrative.  If the
requested resolution contains such points, these must be ignored and dismissed.   

\p Matters involving allegations of ``failure to supervise'' by a
graduate student will be referred directly to the Faculty of Graduate Studies
for resolution.  


% \subsection{\label{sec:grade_reassessment}Grade Reassessment (Re-grading)}
% 
% \fixme{Earlier, it says we just follow university rules. It seems best to do
% that, rather than to write something here that could be wrong. I confess I
% don't really understand the system of the two levels. What I read here is
% consistent with what I've seen done in the past (e.g. I regraded an exam for
% Barry Ruddick's class, once).}
% 
% \npara{1} A request to re-grade a written exam or test follows the procedures
% set by the University's Registrar.
% 
% \npara{2} A grade cannot be both reassessed via the Registrar's Office and
% appealed to the Department, under the rules in this document.  A student must
% choose one procedure or the other, as they are mutually exclusive.  Under
% normal circumstances a student should ask for a re-grading, unless there is
% evidence of bias, irregularity, unfairness in the administration of the exam,
% or inappropriate expectation.   
% 
% \npara{3} When the Department receives a re-grading request, the instructor of
% the course will prepare the following materials:
% 
% \begin{enumerate}
% 
% \item  A list of at least two independent other professors (including contact
% information), either within or outside of the University, who are sufficiently
% knowledgeable of the topic in question, so as to be suitable to re-grade the
% exam.  (Real and apparent conflicts of interest must be avoided in choosing
% these re-assessors.) 
% 
% \item A copy of the exam in question, along with a template of ``optimal''
%     answers for each question.
% 
% \item Copies of at least one student-completed exam at both the A and B grade
% levels, fully de-personalized.  
% 
% \item A fully de-personalized copy of the completed exam being re-graded.
% 
% \end{enumerate}
% 
% \npara{4} The above materials will be given to the Chair of the Department who
% will then contact the re-assessors and forward the materials if they accept the
% re-grading task.  The re-grading should be complete within two weeks of receipt
% of the material by the re-assessors.  
% 
% \npara{5} Communication of the results of the re-assessment will follow
% Registrar procedures.   
% 





\section{Procedures for Student Academic Appeals}
%and Governance of the Ad Hoc Appeal Committee}

\subsection{Introduction}

\cp

\p The procedures detailed below will be followed in cases where a student
wishes to appeal an academic procedure, other than a grade, on the set of
criteria detailed in section~\ref{sec:basis_for_an_appeal}. (Grade appeals are
handled by the University's Registrar.)

\p The regulations and procedures given below cover appeals launched by
both undergraduate and graduate students in the Department of Oceanography at
Dalhousie University.  

\p These procedures do not apply to labor disputes between an employee
(be that person a student, post-doc, etc.) and an employer (faculty, staff) in
regard to working conditions.  Persons wishing to dispute a labor condition
should consult either the Chair of the Department and/or their union agreement,
if applicable, or the laws governing labour relations in Nova Scotia and the
procedures detailed therein. 

\p These regulations do not apply if the cause of the dispute centers
on discrimination based on gender, race, nationality, sexual orientation,
gender identity, official language, or disability.  The complainant in such
cases must contact the Dalhousie Office of Human Rights, Equity \& Harassment
Prevention for procedures in such cases.  Nor are these regulations applicable
to cases of sexual harassment, assault, criminality, etc., where Dalhousie
Security or Law Enforcement should be involved.  

\p There are no appeals on admission decisions, including transfers to
the PhD program.  

%\end{document}

\subsection{\label{sec:basis_for_an_appeal} Basis for an Appeal}

\cp

\p The procedures below apply to the administration of qualifying and
preliminary examinations, comprehensive examinations, thesis proposal defenses,
and Master's thesis defenses.  Results (grades) of such examinations cannot be
appealed as oral portions cannot be regraded.  

\p Appeals related to Doctoral defenses must be directed to Faculty of
Graduate Studies in the first instance. 

\p Appeals of course examination grades (re-grading) are direct through
the University Registrar's Office.
%, and not the Department of Oceanography.
%Some clarification on the method of reassessment is given below in
%section~\ref{sec:grade_reassessment}; otherwise the appellant must follow the
%procedures set by the Registrar. \fixme{should we just leave the rules to the
%registrar, thus axiomatically avoiding contradiction?}
 
\p The grounds for appeal are limited to the following.  For all these grounds,
the onus is on the appellant to prove, through factual documentation, that such
conditions existed. 

%\end{document}

\begin{enumerate}

    \item Procedural Unfairness.  (This means that the method used to
        administer the academic process, e.g., an exam, was unreasonably and
        inherently stacked against any person attempting the process.  Note:
        exams and their content, in and of themselves, are not procedurally
        unfair.  Differences in how an exam is administered between the
        sub-disciplines in the Department also does not constitute unfairness,
        as long as the method is applied consistently within that
        sub-discipline.)

    \item Bias. (This means that the process was conducted in such a way as to
        disadvantage the particular appellant, relative to other persons
        undertaking the same process.)

    \item Irregularity in Procedure. (This means that the process did not
        follow the procedures set in the regulations and guidelines governing
        the actions of the Department and its faculty.)

    \item Inappropriate or Unfair Expectation(s). This means that the
        deliverable of the process could not reasonably be expected of a person
        who undertakes assiduously, genuinely, and accurately the steps leading
        to completion of that process.) 

\end{enumerate}

% ``Procedural Unfairness'' means that the method used to administer the academic
% process, e.g., an exam, was unreasonably and inherently stacked against any
% person attempting the process.  Note: exams and their content, in and of
% themselves, are not procedurally unfair.  Differences in how an exam is
% administered between the sub-disciplines in the Department also does not
% constitute unfairness, as long as the method is applied consistently within
% that sub-discipline.  
%
% ``Bias'' means that the process was conducted in such a way as to disadvantage the
% particular appellant, relative to other persons undertaking the same process.    
%
%``Irregularity in procedure'' means that the process did not follow the
%procedures set in the regulations and guidelines governing the actions of the
%Department and its faculty.  
%
%``Inappropriate or unfair expectation'' means that the deliverable of the
%process could not reasonably be expected of a person who undertakes
%assiduously, genuinely, and accurately the steps leading to completion of that
%process.   


\p A written appeal must be submitted to the Chair of the Department of
Oceanography within 30 days following the event or circumstances being
appealed. 

\p The appeal submission must include:
\begin{enumerate}

\item A description of the exact nature of the appeal, including a summary of
    events and chronology

\item Specific details of the alleged unfairness, bias or irregularity, and any
    other relevant consideration or information

\item The requested resolution of the appeal, which is limited to a reasonable
    academic action(s). 

\end{enumerate}

\p The submission of an appeal will engender the following actions by
the Chair of the Department:

\begin{enumerate}

\item The Chair will contact both the appellant and the person responsible for
the process being appealed to see if an informal resolution is first possible, and

\item Failing an informal resolution, the Chair will constitute an Ad Hoc
Appeal Panel for that specific appeal.  The nature of and procedures for that
Panel are detailed below in section~\ref{sec:the_ad_hoc_appeal_panel}.

\end{enumerate}

\p Decisions of the Ad Hoc Appeal Panel are forwarded to the Chair, who
will communicate the decision in writing to the appellant.

\p Decisions of the Ad Hoc Appeal Panel are subject to further appeal to
the Faculty of Science (undergraduates) or the Faculty of Graduate Studies
(graduate students), or Senate, as regulations in those administrative units
specify. 

%\end{document}


\subsection{\label{sec:the_ad_hoc_appeal_panel}The Ad Hoc Appeal Panel}

\cp

\p Decisions on appeals are made by an Ad Hoc Appeal Panel.

\p The Panel does not have fixed membership.  Instead, it will consist
of three faculty members, chosen on a fixed rotating basis from all the faculty
in the Department.  The professor(s) responsible for the process being appealed
will be excluded.  For graduate students, the student's \supervisor or
co-supervisors, or for undergraduates, the student's honors supervisor (if
applicable) will also be excluded from the list.

\p An appropriate graduate student (for undergraduate appeals) or
Post-doctoral fellow (for graduate appeals) will constitute the fourth and
final member of the Panel.  Any student on the Panel will not have a personal
or academic conflict of interest with respect to the appellant.  For
undergraduates, the student will be chosen from the graduate students.  For
graduate students, the student member will be chosen from the Post-doctoral
Fellows, without conflict of interest.  

\p The Panel will examine the appeal and determine if a \emph{prima
facie} case exists for the appeal.  (\emph{Prima facie} in this context means
the appeal document contains sufficient evidence to support the stated claim.)
If such a case exists, then the Panel will convene a formal Appeal Hearing at
the first opportunity when all involved, i.e., appellant, the faculty member
who owns the process, and the Panel members can meet to hear verbal arguments
on the appeal, but preferably within three weeks of constitution of the Panel.
If there is no \emph{prima facie} case, the Panel will inform the Chair and the
appeal will be dismissed at the Departmental level.

\p If a Hearing is convened, the Panel will chose a Chair who will
control and direct all communication.

\p The Chair of the Panel will make the faculty owner of the process
aware of the appeal and provide, in a timely manner, a copy of the appeal
document to the process-owner (faculty member).  

\p Before the Hearing, the process owner will provide the Panel with a
written rebuttal to the appeal within 5 working days of the appeal being
received.  (Exceptions will be made to accommodate persons being at sea or
subject to other serious limitations in availability.)  The Panel will forward
that rebuttal to the appellant.


\subsection{The Appeal Hearing}
\cp

\p The Hearing will adhere to the following steps:

\begin{enumerate}

\item The Chair of the Panel will direct the flow of the Hearing, making sure
    both parties are properly heard, and directing questioning;

\item The Chair of the Panel will first invite the appellant to present and
    explain their case; this will be followed by an invitation to the faculty
    member owner for a rebuttal;

\item Neither the appellant nor the owner can ask questions during these
    presentations; however, any Panel member may interrupt the presentation to
    ask questions;

\item At the end of the presentations, the appellant will be invited to direct
    questions to the faculty owner for the purpose of seeking clarifications
    (only); after that is completed, the faculty owner will be invited to do
    the same;  the Panel members may interrupt these questioning sessions at
    any time to ask their own questions; the Chair of the panel will ensure
    that this questioning remains civil and on topic;

\item At the end of the questioning, the Chair of the Panel will ask the
    appellant and faculty owner to leave and convene an \emph{in-camera}
    discussion of the presentations and Appeal Document, the Panel will decide
    on the validity of the appeal and of requested resolution.

\end{enumerate}

\p The Chair of the Panel will communicate, in writing, the decision of the
Panel within 3 working days (except for serious time conflicts such as noted
those mentioned at \ref{sec:the_ad_hoc_appeal_panel}.7 above).\fixme{cross ref}

\p In an appeal process, the student has the right to representation.
The student is required to inform the Chair of the Panel, in writing, if s/he
will have a representative at the appeal, or intends to call witnesses. 

\p Witnesses may be called by either party, but only if those witnesses
can testify about direct knowledge of the matter.  No character or indirect
testimony is permitted.  The Panel Chair must be informed 72-hours before a
Hearing if a witness(es) will be called, along with that person's identity and
significance to the Hearing.  The Chair of the Panel will then inform the other
party of the witness(es).  

\p No persons providing solely moral support can be present at a
Hearing.  A person providing aid to someone with a disability is permitted, as
is a translator, but such persons cannot testify at a Hearing.

%\end{document}

\subsection{Supplementary Points}

\cp

\p The Ad Hoc Appeal Panel has no jurisdiction to hear student appeals
on a matter involving a requested exemption from the application of
Departmental, Faculty or University regulations or procedures, except when
irregularities or unfairness in the application thereof is alleged.  This means
that only procedural issues, and not the merits of the regulations, are subject
to appeal.

\p The Ad Hoc Appeal Panel may not render decisions counter to
Departmental, Faculty or University regulations, nor can it make decisions that
go beyond strictly academic matters, e.g., financial or administrative.  If the
requested resolution contains such points, these must be ignored and dismissed.   

\p Matters involving allegations of ``failure to supervise'' by a
graduate student will be referred directly to the Faculty of Graduate Studies
for resolution.  


\addcontentsline{toc}{section}{Index}
\printindex


\newpage
\textbf{\large Notes from GOC meeting on 2016-06-12} \label{mn:20160612}

\includegraphics[height=8in]{meetings/20160612/20160612_meeting_notes.jpg}

\newpage
\textbf{\large Notes from GOC meeting on 2016-06-17} \label{mn:20160617}

\includegraphics[height=8in]{meetings/20160617/20160617_meeting_notes.jpg}

\newpage
\textbf{\large Notes from GOC meeting on 2016-07-08} \label{mn:20160708}

\includegraphics[height=8in]{meetings/20160708/20160708_meeting_notes.jpg}

\newpage
\textbf{\large BioOce QE rules, marked up} \label{boqe}


%\includegraphics[height=8in]{marked_up_versions/bo_qe_20160708.pdf}


\end{document}
