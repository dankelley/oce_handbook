\documentclass[12pt]{article}
\usepackage{fullpage}
%\usepackage{mathptmx}
\usepackage{times}
\usepackage{ifthen}
\usepackage{microtype}
\usepackage{color}
\usepackage{xspace}
\usepackage{makeidx}
\usepackage{enumitem}
\usepackage{graphicx}
\usepackage[normalem]{ulem}
\usepackage{titlesec}
\usepackage{tikz}
\usepackage[plainpages=false,debug]{hyperref} % should be last package


\setlist{nosep} % tighten itemized lists

\definecolor{diColor}{rgb}{0,0,0.8}
\definecolor{deleteColor}{rgb}{1,0,0}
\definecolor{addColor}{rgb}{0.0,0.5,0}

\definecolor{fixmeColor}{rgb}{0.8,0,0} % hide them as black

\newcommand\delete[1]{\color{deleteColor}\sout{\textbf{#1}}\color{black}\xspace%
\marginpar{\color{deleteColor}$\Delta$\color{black}}}

\newcommand\add[1]{\color{addColor}\textbf{#1}\color{black}\xspace%
\marginpar{\color{addColor}$\Delta$\color{black}}}

%\newcommand\edited{\marginpar{\color{red}$\Delta$\color{black}}}


\newcommand{\di}[2][EMPTY]{\ifthenelse{\equal{#1}{EMPTY}}{\color{diColor}#2\color{black}\index{#2}}{\color{diColor}#2\color{black}\index{#1}}}
\newcommand{\fixme}[1]{\color{fixmeColor}$<$#1$>$\color{black}\index{$>>>>$FIXME$<<<<$}}

\newcommand{\npara}[1]{\noindent\textbf{\thesubsection.#1}} % for numbered paragraphs of appeals section

\setlength{\parindent}{0em}
% number paragraphs
\newcommand{\parnum}{\arabic{parcount}}
\newcounter{parcount}
%\newcommand\p{\stepcounter{parcount}\leavevmode[\parnum]\hspace{0.2em}\marginpar[\hfill\parnum]{\parnum}}
\newcommand\p{\stepcounter{parcount}\leavevmode{\small[\parnum]}\hspace{0.2em}}
\newcommand\cp{\setcounter{parcount}{0}}

\newcommand{\curcom}{\index{Committees!Curriculum}\color{diColor}Curriculum Committee\color{black}\xspace}

\newcommand{\advcom}{\index{Committees!Advisory}\color{diColor}Advisory Committee\color{black}\xspace}

\newcommand{\goc}{\index{Committees!Graduate Oversight}\color{diColor}Graduate Oversight Committee\color{black}\xspace}

\setlength{\parskip}{0.7ex plus 0.5ex minus 0.2ex}

\setcounter{secnumdepth}{5}
\titleformat*{\section}{\large\bfseries}
\titlespacing*{\section}{0pt}{*0}{0pt}
\titleformat*{\subsection}{\slshape}
\titlespacing*{\subsection}{0pt}{*0}{0pt}
\titleformat*{\subsubsection}{\slshape}
\titlespacing*{\subsubsection}{0pt}{*0}{0pt}
%\titleformat*{\subsubsubsection}{}
%\titlespacing*{\subsubsubsection}{0pt}{*0}{0pt}
\titleformat*{\paragraph}{\slshape}
\titlespacing*{\paragraph}{0pt}{*0}{0pt}

\newcommand{\ExternalLink}{%
    \tikz[x=1.2ex, y=1.2ex, baseline=-0.05ex]{% 
        \begin{scope}[x=1ex, y=1ex]
            \clip (-0.1,-0.1) 
                --++ (-0, 1.2) 
                --++ (0.6, 0) 
                --++ (0, -0.6) 
                --++ (0.6, 0) 
                --++ (0, -1);
            \path[draw, 
                line width = 0.5, 
                rounded corners=0.5] 
                (0,0) rectangle (1,1);
        \end{scope}
        \path[draw, line width = 0.5] (0.5, 0.5) 
            -- (1, 1);
        \path[draw, line width = 0.5] (0.6, 1) 
            -- (1, 1) -- (1, 0.6);
        }
    }


\makeindex

\title{Oceanography Graduate Handbook}
\author{}
\date{September, 2016}


\begin{document}
\maketitle
\newpage

\renewcommand{\baselinestretch}{0.5}\normalsize
{\footnotesize
\setcounter{tocdepth}{2}
\tableofcontents
}
\renewcommand{\baselinestretch}{1.0}\normalsize


\section{\label{sec:glossary}Glossary of acronyms and terms}

The following list explains some acronyms and special terms used in this
document.

\begin{itemize}

    \item ``CRN'' refers to a Course Registration Number. For example, the
        Physical Oceanography core class, listed as OCEA~5120 in the University
        Calendar, has CRN 12070. Consult the Registrar's website\footnote{Registrars website of Oceanography classes \href{https://dalonline.dal.ca/PROD/fysktime.P_DisplaySchedule?s_term=201710,201720&s_subj=OCEA&s_district=100}{\ExternalLink}.}

    \item ``External'' \fixme{write a few lines here, with the 3 types
        discussed in the meeting of 20160527}.

    \item ``FGS''  stands for Faculty of Graduate Studies.

    \item ``NSERC'' stands for Natural Science and Engineering Research
        Council, a funding agency.

\end{itemize}


% A.	GENERAL  PROGRAM REQUIREMENTS
\section{\label{sec:general_program_requirements}General Program Requirements}

%A1	REGISTRATION
\subsection{Registration}

\p Graduate students must maintain their registration in all three terms of
each year in their program, except in cases where a formal \di[leave of
absence]{Leave of Absence} has been approved by the Faculty of Graduate
Studies.

\fixme{I don't understand what REGN etc even means. I'll just rewrite this
junk after I ask a student what they do.}

\p Registration consists of
\begin{enumerate}
    \item REGN 9999
    \item Thesis Code
    \item Courses (if applicable)
\end{enumerate}

\p Students who fail to register within the approved deadlines will be considered
to have \di{lapsed registration}. Such students will not be permitted to submit
a thesis, nor will they receive any services from the University during that
academic term. Students who allow their registration to lapse will be
considered to have withdrawn and will be required to apply for
readmission\footnote{FGS Regulations, section 5.2.3: Failure to Register
\href{http://academiccalendar.dal.ca/Catalog/ViewCatalog.aspx?pageid=viewcatalog&catalogid=58&chapterid=2678&topicgroupid=10848&loaduseredits=False}{\ExternalLink}.}.

%A1.1	REGN 9999
\subsection{REGN 9999}

\cp

\p Students must register at least one month prior to the beginning of each term.

\p Students must register for the Fee Generating
Course\index{courses!fee-generating} (REGN 9999) in all three terms for the
duration of their program. If REGN 9999 is not added for each term, graduate
students are not considered to be registered. Failure to register at least one
month prior to the beginning of each term will result in non- payment of
scholarships and stipends.

\p The \di{Course Registration Number} (\di{CRN}) for each term can be found
in the Academic Timetable on the Registrar's Office website\footnote{Academic
Timetable for Oceanography
\href{https://dalonline.dal.ca/PROD/fysktime.P_DisplaySchedule?s_term=201710,201720&s_subj=OCEA&s_district=100}{\ExternalLink}.}.

%A1.2	THESIS CODE
\subsection{Thesis Code}

\cp

\p Students must register at least one month prior to the beginning of each term.

\p Students must register for the Thesis Code in all three terms for the duration
of their program. Failure to do so in a term during which formal classes are
not being taken will result in a blank term on the student's transcript, which
means that there will be no documentation demonstrating that work was done on
the thesis.

\p The CRN for each term can be found in the Academic Timetable under the
Oceanography course listings (MSc – OCEA 9000, PhD – OCEA 9530).



% A2	COURSES
\subsection{Courses}
 

%A2.1	CORE COURSES
\subsubsection{\label{sec:core_courses}Core courses}

\cp

\p Core courses are broadly based, advanced-level courses that are intended to
help students gain a wide understanding of the major areas of oceanography, and
they are to be taken outside, as well as within, the student's research area.
They are offered every year.

\p Depending on the degree program, students usually take some combination of the
core courses\index{courses!core}, OCEA 5110, 5120, 5130, and 5140, which cover
Marine Geology and Geophysics, Physical Oceanography, Chemical Oceanography and
Biological Oceanography, respectively. One or more of these sometimes may be
waived with the consent of the thesis advisory committee, if the student has
sufficient background.


%A2.2	COURSE REQUIREMENTS
\subsubsection{Course requirements}

\cp

\p A Graduate Student Program form must be submitted to the Graduate Secretary
within one month of the start of your program.

\p Students must submit a Graduate Student Program Update Form to the Graduate
Secretary when a decision to make changes to course requirements is
made\footnote{Graduate Student Program Forms can be downloaded from the FGS
website
\href{http://www.dal.ca/faculty/gradstudies/currentstudents/forms.html}{\ExternalLink}.}.
Failure to do so can result in the assessment of additional fees for courses
not included in your program requirements, or affect the student's ability to
graduate.

\p In addition to the core classes as described above, there are several classes
designed to provide students with the tools and background knowledge
appropriate to their particular area of research. As explained in sections B
and C, the detailed requirements differ for MSc and PhD and between the
sub-disciplines of Oceanography. The choice of classes is made by the student
in consultation with the thesis supervisor, with the approval of the advisory
committee. Students must register for the 5000 level courses (4000 level is for
undergraduates only) prior to the registration deadline\footnote{Course
registration deadlines can be found in the Graduate Studies Academic Calendar
\href{http://academiccalendar.dal.ca/Catalog/ViewCatalog.aspx?pageid=viewcatalog&catalogid=58&chapterid=2677&loaduseredits=False}{\ExternalLink}.}.

%A2.3	GRADING
\subsubsection{Grading}

\cp

\p FGS Regulations\footnote{FGS Regulations, section 7.6: Classes and Grades
\href{http://academiccalendar.dal.ca/Catalog/ViewCatalog.aspx?pageid=viewcatalog&catalogid=58&chapterid=2678&topicgroupid=10850&loaduseredits=False}{\ExternalLink}.}
stipulate that graduate students must achieve a grade of B- or higher in all
classes required as part of their degree program.  The grading scheme follows
the FGS regulations.

\p Upon failure of any course (grade less than B-), a student will be withdrawn
from the Graduate Program and must apply for re-admission. Such a student may
apply, in writing, to the department for immediate reinstatement\footnote{FGS
Regulations, section~5.4.1: Readmission of Students
\href{http://academiccalendar.dal.ca/Catalog/ViewCatalog.aspx?pageid=viewcatalog&catalogid=58&chapterid=2678&topicgroupid=10848&loaduseredits=False}{\ExternalLink}.}.
Reinstatement to a program after a failing grade must be supported by the
Graduate Coordinator, and must be approved in writing by the Faculty of
Graduate Studies. If readmitted, any subsequent 'F' will result in a final
program dismissal. Any academic withdrawal and reinstatement will be recorded
on the student's official transcript.
% above was 5.2.6 but now seems to be 5.4

%A2.4	REGISTERING FOR COURSES AT ANOTHER INSTITUTION
\subsection{Registering for courses at another institution}

\cp

\p Classes approved by the Department and Faculty of Graduate Studies (after
examination of class descriptions) can be taken at other universities as part
of the graduate degree program, provided the class is not available at
Dalhousie\footnote{FGS Regulations, section 7.6.6: Letters of Permission
\href{http://academiccalendar.dal.ca/Catalog/ViewCatalog.aspx?pageid=viewcatalog&catalogid=58&chapterid=2678&topicgroupid=10850&loaduseredits=False}{\ExternalLink}.}.
The Letter of Permission Form\footnote{FGS forms and documents: Letter of Permission--Request Form 
\href{http://www.dal.ca/faculty/gradstudies/currentstudents/forms.html}{\ExternalLink}.}
must be completed and submitted to the Graduate Secretary in advance. Such
approval will not be given retroactively.


%A3	THESIS PROPOSAL
\subsection{Thesis proposal}

\cp

\p When a particular research area has been identified and an advisory committee
appointed, students are required to provide their advisory committees with
detailed thesis proposals. This document, typically developed in collaboration
with the supervisor, should demonstrate the student's

\begin{itemize}
\item Background of appropriate literature
\item Awareness of current research activity
\item Ability to formulate pertinent scientific hypotheses
\item Appreciation of the time, effort, and resources necessary to achieve the thesis objectives
\end{itemize}

\p The acceptance of a thesis proposal is a critical step in a student's program.
The detailed requirements differ for MSc and PhD proposals (see
sections~\ref{sec:msc_program_requirements}
and~\ref{sec:phd_program_requirements} respectively).

%A4	SEA TIME
\subsection{Sea time}

\cp

\p Graduate students in Oceanography are required to spend time at sea on
oceanographic ships to familiarize themselves with oceanographic techniques,
even if their research does not require measurements at sea. Students are
required to submit a \di[sea time]{Sea Time} Form to the Graduate Secretary.
The \curcom must approve completed forms. Requests for special consideration by
the \curcom for \di{fieldwork} in lieu of sea time may be made in extenuating
circumstances.


% A5	SEMINARS
\subsection{Seminars}

\p Students are required to attend the general \di{departmental
seminars}\index{seminars} and to attend, and participate in, the specialty
seminars in their field of interest.  Students who are unable to attend
seminars regularly must have the specific agreement of their \advcom that this
requirement is waived and this must be communicated to the departmental office
in a written memo signed by the supervisor. It is important to note that
materials presented in these seminars may form part of the questions for the
PhD Qualifying Exam and the PhD Thesis Proposal Defence.



 

%A6	CHANGE IN STATUS
\subsection{Change in status}

\cp

\p Any change in status, such as transfer from MSc to PhD (see
section~\ref{sec:transfer_to_phd}), a leave of absence, and entrance to another
degree program must be recommended by the advisory committee to the Graduate
Oversight Committee. Final approval for a change in status will be made by the
Graduate Oversight Committee, which will meet as required to review these
requests.  Students should consult the Graduate Secretary regarding the
required documentation.


%A7	ANNUAL PROGRESS REPORT
\subsection{Annual progress report}

\cp

\p Students must submit an \di{Annual Progress Report} each year, one month prior
to the anniversary of their admission date. Failure to submit this report will
result in delays in registration and funding.

\p This report is made with the Graduate Student Information System (GSIS)
through DalOnline\footnote{DalOnline
\href{https://dalonline.dal.ca}{\ExternalLink}.}.  The supervisor and the
Graduate Coordinator must approve this report electronically, and students
should bear this in mind so that the deadline can be met.

\p A hard copy of the Oceanography ``\di{supplementary program report form}''
(available from the Graduate Secretary) must also be approved by your
supervisor and the Graduate Coordinator, and submitted to the Graduate
Secretary annually, following the same deadline as the Annual Progress Report.

\p The \di{supplementary program report form} is intended to be saved
electronically and added to on a regular basis, similar to a CV. An up-to-date
copy of the \di{supplementary program report form} is to be brought to all
meetings with the Graduate Coordinator.


%A8	TIME LIMITS FOR COMPLETION OF DEGREES
\subsection{Time limits for completion of degrees}

\cp

\p Graduate students have a maximum period of time within which to complete their
graduate program, 4 years for MSc and 6 years for PhD.

\p Extensions may be granted by FGS on the recommendation of the department, along
with a satisfactory Progress Report\footnote{FGS Regulations, section 7.1:
Maximum Time for Degree Completion and Extensions
\href{http://academiccalendar.dal.ca/Catalog/ViewCatalog.aspx?pageid=viewcatalog&catalogid=58&chapterid=2678&topicgroupid=10850&loaduseredits=False}{\ExternalLink}.}
Under no circumstances can a student be registered in a program beyond 10 years
from their initial registration in the program.


%A9	THESIS ADVISORY COMMITTEE
\subsection{Thesis advisory committee}

\cp

\p The Thesis \advcom must be formed within one month of the start of the
student's program.

\p The supervisor will select a provisional committee, in
consultation with the student. A \di{Graduate Student
Program Form} listing the \advcom members must be
submitted to the Graduate Secretary.

\p Membership on an \advcom is flexible and may change during the course of a
student's program. Students must complete a \di{Graduate Student Program Update
Form} when a decision to make changes to committee membership is made. The
\advcom structure differs for MSc and PhD, as explained in
sections~\ref{sec:msc_program_requirements} and
\ref{sec:phd_program_requirements}.

% above was: sections B and C

\p Advisory committees are formally sub-committees of the Department appointed
both to provide expert advice to students and to evaluate and report on student
progress.

\p All committee members must have Faculty of Graduate Studies membership (most
commonly as an \di{External Scholar}). Memberships should be applied for through the
Graduate Secretary.

\p Committees are changed by mutual consent within the committee or by the Chair
of the committee and student in consultation with the Graduate Coordinator or
Chair of the Department. Limits to changes are set by the availability of
particular people as supervisors and of research facilities and funds (noting
that most research grants are awarded for specific projects).

%A9.1	ADVISORY COMMITTEE MEETINGS
\subsection{Advisory committee meetings}

\cp

\p There must be no less than two advisory committee meetings held per academic
year\footnote{FGS Regulations, section 9.3: Supervisory Committees
\href{http://academiccalendar.dal.ca/Catalog/ViewCatalog.aspx?pageid=viewcatalog&catalogid=58&chapterid=2678&topicgroupid=10852&loaduseredits=False}{\ExternalLink}.}

\p Advisory committee meetings may be held as often as required by the student
and/or supervisor. It is the responsibility of the student to ensure that
committee meetings are held. The supervisor will ensure that brief minutes of
these meetings are recorded and are filed with the Graduate Secretary.

\p Following approval of the research proposal by the advisory committee, the
student should have several meetings in addition to the regular committee
meetings as follows:

\begin{itemize}

\item An interim progress meeting to verify that the research is on track.

\item A final progress meeting at which the committee agrees that sufficient
research has been conducted, and that thesis writing may proceed.

\item Individual meetings with advisory committee members to receive input on drafts or chapters.

\item A meeting at which the advisory committee agrees that the thesis is
defensible and that a defence may be scheduled. Alternatively, the advisory
committee may request further revision prior to approving a defence.

\end{itemize}





%A9.2	COMMITTEE APPROVAL
\subsection{Committee approval}

\cp

\p At several points in a student's program the approval of the advisory committee is essential:
\begin{itemize}
\item Initial discussion of research plan and direction; this should occur as early as possible in the student's program.
\item Approval of research plan and direction for thesis proposal.
\item Acceptance of a thesis proposal.
\item Change of status, such as advancement to the PhD program.
\item Approval to write up sections of a thesis.
\item Approval of draft of thesis.
\item The acceptance of the thesis, in final form, as being ready for defence.
\end{itemize}


%A9.3	SUPERVISOR
\subsection{Supervisor}

\cp

\p The Supervisor provides direct supervision of the student's research and is expected to
\begin{itemize}
\item Advise the student on the choice of a research topic, on the possible directions to emphasize during the work, and the point at which it should be concluded.
\item Supervise the research, as well as the preparation of the proposal, progress reports and thesis.
\item Ensure that the resources necessary to the thesis project are made available and that any necessary skills are acquired by the student.
\item Assess the student's progress, evaluating strengths and weaknesses.
\item Provide current updates of progress to the student's file.
\item Ensure that the student is funded
\item Select thesis examining committee members (following the guidelines for composition of the various examining committees).
\end{itemize}

\p A supervisor is changed only with the approval of the Department. This may
arise by mutual agreement within the advisory committee or at the request of
the student (to either the Graduate Coordinator or the Chair of the
Department).

\p In order to take advantage of the special expertise and facilities available,
the role of supervisor is not restricted to permanent faculty members of the
Department. However, supervisors must be Adjunct Professors or Full Professors
with Faculty of Graduate Studies membership in Oceanography.

\p In the case where a student has a supervisor who is not a full time faculty
member in the Department of Oceanography, there must be an internal supervisor
who is a faculty member in the Department.

%A9.4	INTERNAL SUPERVISOR
\subsection{Internal supervisor}

\cp

\p The responsibilities of the internal supervisor in part are
\begin{itemize}
\item To advise on academic requirements, course load, waiver of courses, general departmental affairs, etc.
\item To ensure that the student has suitable office and laboratory space.
\item In consultation with the student, to select a supervisor and advisory committee appropriate to the student's research interests (In many cases the chair will also be the supervisor).
\item To ensure that adequate reports and records of committee meetings are kept and submitted to the appropriate departmental files, and that necessary decisions as to the academic program, (e.g.. thesis proposal), are made on an appropriate time-scale.
\item To assess the progress of the student both in academic work and research.
\end {itemize}
 

%B.	MSc  PROGRAM REQUIREMENTS

\section{\label{sec:msc_program_requirements}MSc program requirements}

% B1	MSc COURSE REQUIREMENTS

\subsection{MSc course requirements}

\cp

\p MSc students must complete a minimum of 5 half-credits at the 5000-level or
higher, at least three of which must be chosen from the introductory core
courses (refer to section~\ref{sec:core_courses}).

\p Any student who anticipates a \index{transfer from MSc to PhD} transfer from
the MSc program to the PhD program should complete the course-work and
other requirements as listed in the PhD program requirements.

%B2	MSc COMMITTEE STRUCTURE
\subsection{MSc committee structure}

\cp
        
\p The Thesis \advcom must be formed within one month of the start of the
student's program.

\p The MSc \advcom consists of at least three members. There must be two full-time
faculty members from the Oceanography Department (not Adjunct Professors). If a
regular faculty member serving on the committee leaves the department, a
replacement with another faculty member must be made. One member will be from
another sub-discipline, and it is desirable that at least one member of the
committee be from outside the Department.


%B3	MSc THESIS PROPOSAL
\subsection{MSc thesis proposal}

\cp

\p MSc students are expected to produce an approved proposal within one year of
enrolling in the program.

\p The thesis proposal should be developed in consultation with the supervisor and
advisory committee. The scope of the research should be such that it can be
accomplished and the thesis written within one year. Students should consult
with their supervisors on the content of a proposal.

\p The Supervisor will notify the Oceanography Office that the proposal
requirement has been satisfied and a copy of the approved proposal must be
placed on file with the Graduate Secretary. MSc students may be required to
defend the research proposal at the discretion of their committee.

\p Because a large portion of a student's first year is consumed by course work,
the thesis proposal serves primarily as a well-reasoned course of action, and
not as an exhaustive literature review nor as a deeply detailed description of
methods. Fifteen pages of text should provide ample space.


%B4	MSc THESIS
\subsection{MSc thesis}

\cp

\p Graduate students should pay particular attention to the Faculty of Graduate
Studies deadlines for submission of a thesis for graduation and fee payment
schedules.
 
\p The examination for the degree of Master of Science is subject to detailed
regulations of the Faculty of Graduate Studies\footnote{FGS Regulations,
section 10.3: Master's Theses \href{http://academiccalendar.dal.ca/Catalog/ViewCatalog.aspx?pageid=viewcatalog&catalogid=58&chapterid=2678&topicgroupid=10853&loaduseredits=False}{\ExternalLink}.}.

\p The MSc thesis should report original research of such value as to merit
publication and be in a satisfactory and consistent literary form. Faculty of
Graduate Studies thesis formatting requirements are available on their
website\footnote{FGS Regulations, section 10.2: Preparation of Manuscript and
Submission of Theses
\href{http://academiccalendar.dal.ca/Catalog/ViewCatalog.aspx?pageid=viewcatalog&catalogid=2&chapterid=399&topicgroupid=1435&loaduseredits=False}{\ExternalLink}.}.

\p The oral defence of the thesis is open to all members of the department and to
other interested persons. Notices of the thesis defence, including an abstract,
will be distributed to all members of the Oceanography Department and will be
posted.

%B4.1	MSc EXAMINING COMMITTEE
\subsection{MSc examining committee}

\cp

\p It is the supervisor's responsibility to select the examining committee
members, following the committee structure guidelines. The defence may
\emph{not} proceed in violation of the below composition of the examining
committee due to short-term absence of committee members.

\p The examining committee will consist of a minimum of 5 members:

\begin{itemize}
    \item 3 members of the advisory committee
        \begin{itemize}
            \item Supervisor(s)
            \item Full-time department faculty member
            \item One other member
        \end{itemize}

    \item Departmental Representative/Chair (from outside the
        committee)\footnote{The Graduate Secretary will arrange for the
        Departmental Representative (normally the Graduate Coordinator or a
        designate). The Chair of the MSc thesis defence is independent of the
        examining committee. The Departmental Representative normally performs
        this duty.}

    \item External examiner from outside the advisory
        committee\footnote{Adjunct professors may fulfill the requirement of an
        external examiner.}

\end{itemize}


%B4.2	MSc THESIS DEFENCE TIMELINE
\subsection{MSc thesis defence timeline}

\cp

\p Students with immediate commitments subsequent to the defence must anticipate
the deadlines, rather than attempting to rush the defence process in a
compressed time frame.

\p Prior to proceeding with scheduling the defence, the thesis (in essentially
final form) should be distributed to the supervisor and committee to be
reviewed for suitability and final approval.

\begin{itemize}
\item Six weeks prior to defence:

        \begin{itemize}

            \item Have a format review done by FGS before the thesis is
                distributed to your examining committee

            \item Submit an MSc Examination Information Form to the Graduate
                Secretary

            \item Submit a Thesis Binding Submission Form and payment to the
                Graduate Secretary

            \item Submit an electronic copy of the abstract to the Graduate
                Secretary

            \item Provide copies of the thesis to the supervisor for
                distribution to the examining committee

        \end{itemize}

    \item Four weeks prior to defence:

        \begin{itemize}

            \item Notify the Graduate Secretary of any Audiovisual requirements

        \end{itemize}

    \item Two weeks prior to defence:
        \begin{itemize}

            \item Submit a printed copy of the thesis to the Graduate Secretary

        \end{itemize}

    \item Following a successful Defence:

        \begin{itemize}

            \item Have Examining Committee Members sign the signature page

            \item Submit required changes to the supervisor within the
                specified timeframe

            \item Have a final format check of the thesis done by FGS

            \item Electronically submit the final version of the thesis via
                DalSpace\footnote{DalSpace
                \href{http://dalspace.library.dal.ca/}{\ExternalLink}.}

            \item Submit original completed forms to FGS (National Library of
                Canada Form, Title Page of Thesis, Signature page with original
                signatures, Copyright Page, Ethics Pages (if applicable),
                Student Contribution to Manuscripts (if applicable)

            \item Submit final copies of thesis to the Graduate Secretary for
                binding (include copies of the signed signature page, and the
                Copyright Page with original signatures)

            \item Complete Exit Survey

        \end{itemize}

\end{itemize}

%B4.3	MSc THESIS DEFENCE OUTCOME
\subsection{MSc thesis defence outcome}

\cp

\p Outcomes: All theses are either approved or not approved. The categories are:
\begin{itemize}
\item Approve as submitted
\item Approved pending corrections and a clear timetable for completion (normally within one month)
\item Rejected but with permission to re-submit a revised thesis for re-examination with a clear timetable for completion (within one year)
\item Rejected outright
\end{itemize}

\p A simple majority determines the outcome, based on all the examiners except the
Departmental Representative, who will vote only in the event of a tie.

\p Following the defence, the candidate will receive a letter from the Chair of
the examining committee indicating the nature of any corrections to be made and
the time frame within which they are to be completed. In the event of an
unsuccessful defence, an explanation of the negative outcome is provided. A
copy of the report will be provided to the Graduate Secretary.

\subsection{\label{sec:transfer_to_phd}Transfer to PhD}

\cp

\p A request to \index{transfer from MSc to PhD} transfer from the MSc program to the PhD program should be
completed no later than one month prior to the anniversary of the student's
admission date.

\p An MSc student may transfer to a PhD program without completing a MSc. Students
wishing to transfer should provide a letter to their advisory committee stating
the reason for the transfer, and justification that they are qualified for the
PhD program. A one-page explanation is sufficient. The student's \advcom is
required to meet, and the committee may recommend advancement to the PhD
program. A letter from the supervisor, justifying the advancement of the
student to the PhD program, should be submitted to the Graduate Coordinator
along with the student's letter no later than one month prior to the
anniversary of the student's admission date. The \goc will then review the
request.

\p The decision categories for the transfer to the PhD are:
\begin{itemize}
\item Approved
\item Not approved
\item Deferred
\end{itemize}

\p Students anticipating this change should take the required core courses in the
first year and take the qualifying examinations (see
section~\ref{sec:phd_qualifying_examination}) in the first year following
completion of the required core classes.

% above referred to C3 for QE

%C.	PhD  PROGRAM REQUIREMENTS
\section{\label{sec:phd_program_requirements}PhD program requirements}

%C1	PhD COURSE REQUIREMENTS
\subsection{PhD course requirements}

\cp

\p PhD students must complete a minimum of 6 half-credits at the 5000-level or
higher, at least two of which must be chosen from the introductory core courses
outside the student's sub-discipline (refer to section~\ref{sec:core_courses}).

% above referred to A2.1 originally

\p Courses are to be chosen in consultation with the advisory committee. The
various sub-disciplines have individual course requirements, some of which
exceed the minimum requirements, as explained below.

%C1.1	BIOLOGICAL OCEANOGRAPHY GUIDELINES
%\subsubsection{Biological Oceanography guidelines}

\p The normal course requirements for a PhD in Biological Oceanography are:
\begin{itemize}
\item Core Classes: Biological Oceanography (5140), Chemical Oceanography (5130), and Physical Oceanography (5120)
\item A minimum of one other half credit drawn from Geological Oceanography (5110) or a suitable Atmospheric Science course
\item Other Classes, as required by the Advisory Committee
\end{itemize}

%C1.2	CHEMICAL OCEANOGRAPHY GUIDELINES
%\subsubsection{Chemical Oceanography guidelines}

\p The normal course requirements for a PhD in Chemical Oceanography are:

\begin{itemize}
\item Core Classes: Chemical Oceanography (5130), Geological Oceanography (5110), Physical Oceanography (5120), and Biological Oceanography (5140)
\item Advanced Chemical Oceanography (modular) (5290)\fixme{5290 not offered since 2007/9}
\item Other Classes, as required by the Advisory Committee
\end{itemize}

%C1.3	GEOLOGICAL OCEANOGRAPHY GUIDELINES
%\subsubsection{Geological Oceanography guidelines}

\p The normal course requirements for a PhD in Geological Oceanography are:

\begin{itemize}
\item Core Classes: Geological Oceanography (5110) and two other Introductory Classes.
\item Other Classes, as required by the Advisory Committee
\end{itemize}

%C1.4	PHYSICAL OCEANOGRAPHY GUIDELINES
%\subsubsection{Physical Oceanography guidelines}

\p The normal course requirements for a PhD in Physical Oceanography are:

\begin{itemize}
\item Core Classes: Physical Oceanography (5120) and two other Introductory Classes.
\item Advanced Classes: Fluid Dynamics (5311), Time Series Analysis (5210),
Ocean Dynamics (5221), Estuary, Coast and Shelf Dynamics (5222).
\item Other Classes, as required by the Advisory Committee e.g. Numerical
Modelling (5220), Ocean Waves (5223), Introduction to Acoustical Oceanography
(5250), Advanced Marine Particles (5293).
\end{itemize}

%C2	PhD COMMITTEE STRUCTURE
%\subsection{PhD committee structure}

\p The Thesis Advisory Committee must be formed by October 1st of the first year
of study.

\p The PhD Advisory Committee consists of at least four members. There must be two
full-time faculty members from the Oceanography Department (not Adjunct
Professors) on the advisory committee. If a regular faculty member serving on
the committee leaves the department, a replacement with another faculty member
must be made.

\p One member will be from another sub-discipline, and it is desirable that at
least one member of the committee be from outside the Department.


%C3	PhD QUALIFYING EXAMINATION
\subsection{\label{sec:phd_qualifying_examination}PhD Qualifying Examination}

\cp

\p The \di{qualifying examination} should be completed between months 9 and 12 of
the program.  Students transferring from the MSc program should take the
qualifying examinations within 12 months of their transfer.

\p This examination is designed to assess the student's background knowledge, with
the primary aim of identifying weaknesses that need to be addressed in order
for a student to undertake research at the Ph.D. level in the chosen field. The
format varies with sub-discipline. In each case there is an oral component, but
in some cases there can also be a written assignment before or after the oral
examination.

\p The \curcom  may grant extensions up to month 15 if the student and supervisor
document a compelling conflict, e.g. \di{fieldwork}.
 

%C3.1	QUALIFYING EXAMINATION COMMITTEE
\subsubsection{Qualifying Examination committee}

\cp

\p The examination committee has at least 4 members, including a faculty member
from outside the sub- discipline, and a departmental representative who acts as
chair. The Graduate Secretary will arrange for the Departmental Representative
(normally the Graduate Coordinator or a designate). The composition of the rest
of the examining committee is set by the sub-discipline.

\p The topics of the examination are student-specific, and may include items of a
general oceanographic nature (tailored to the courses taken) as well as items
that are more tightly focused on the sub-discipline and the intended area of
research.

%C3.2	FORMAT AND GUIDELINES
\subsubsection{Format and guidelines}

%C3.2.1  BIOLOGICAL OCEANOGRAPHY QE GUIDELINES
\paragraph{Biological Oceanography QE guidelines}\hfill

\cp
\p The core of the qualification process is an oral examination based on a reading
list tailored to the student's research area. In some cases this may be
followed by a written examination. The details and the timing are as follows.

\begin{enumerate}

\item The PhD candidate prepares a brief (2-3 page) summary of his or her
research interests and sends it to the BO faculty. The student's summary of
research interests guides faculty members in the selection of papers; it will
not be used as source material during the exam, and neither the student nor the
committee members bring copies of the research summary to the exam.

\item Based on this document, at least 3 of the available BO faculty members
(including the supervisor) each contribute one or two papers or book chapters
to a reading list. These papers are intended to serve as guides for the exam
and to provide entry points for discussions of oceanographic topics during
questioning. The process is organized by the research supervisor, who vets the
papers to ensure appropriateness, and then communicates the list to the student
and the BO faculty within 2 weeks of receipt of the student's research summary.

\item It is the responsibility of the candidate's supervisor to: a) form the
Examining Committee; and b) schedule the exam. The committee will include a
Chair, at least three members of the BO faculty, and one Oceanography faculty
member from out of the BO sub-discipline, preferably in the sub-discipline most
closely related to the student's research. The Chair (a member of the
Oceanography faculty not in BO) will serve as moderator of the exam and will
not ask questions. 3 members of the BO group must approve the form of the
examining committee no later than 3 weeks prior to the date of the exam. The
composition of the committee will be revealed to the student and the student
will see that each member has copies of the readings.

\item Within 6 weeks after the reading list is given to the student, an oral
examination is held.

\item The exam begins with a 20-minute presentation by the student,
highlighting aspects of the papers or chapters that are relevant to his or her
research interests.

\item During the next 1-1.5~h, the committee asks questions about oceanography
and topics related to the students research interests. Entry points into
questions and discussions are to be based on the reading material that was
provided to the student and by the student's presentation.

\item After the exam, the committee meets in camera to agree on comments and
recommendations to the student and to the students committee, as appropriate.

\item The possible outcomes of the oral examination are:
\begin{itemize}
\item The candidate passes without extra conditions.

\item The candidate passes, but is informed of weaknesses that should be
addressed during the PhD work, e.g. in courses or in directed studies.

\item The candidate is required to take a written examination, at a date
determined during the oral examination meeting. This examination will be based
on the topics that arose during the oral examination and will not exceed three
hours. The committee can then pass the student with no extra conditions, or be
informed of weaknesses that should be addressed during the PhD work, e.g. in
courses or in directed studies.

\end{itemize}

\item The Chair is to report in writing to the Chair of the Curriculum
Committee on the outcome of the exam, with copies to candidate and examining
committee.

\end{enumerate}


%C3.2.2  CHEMICAL OCEANOGRAPHY QE GUIDELINES
\paragraph{Chemical Oceanography QE guidelines}\hfill

\cp

\p No discipline-specific guidelines--follow general Departmental guidelines.

%C3.2.3  GEOLOGICAL OCEANOGRAPHY QE GUIDELINES
\paragraph{Geological Oceanography QE guidelines}\hfill

\cp

\p No discipline-specific guidelines--follow general Departmental guidelines.

%C3.2.4  PHYSICAL OCEANOGRAPHY QE GUIDELINES
\paragraph{Physical Oceanography QE guidelines}\hfill
\cp

\p The core of the qualification process is an oral examination based on a reading
list tailored to the student's research area. In some cases this may be
followed by a written examination. The details and the timing are as follows.

\p The PhD candidate prepares a brief (1 page) description of the intended area of
research and sends it to the physical oceanography faculty members 5 weeks
prior to the oral examination.

\p Based on this document, each available faculty member in the sub-discipline
contributes a paper to a reading list. This process is organized by the
research supervisor, who then communicates the list to the student and the
Physical Oceanography faculty members within 1 week of receipt of the student's
research summary.

\p Within 4 to 6 weeks, an oral examination is held, to test the student's
comprehension of the reading list and its relevance to the intended research.
The examination committee comprises all available faculty in the
sub-discipline, a faculty member from another sub-discipline, and a
Departmental representative.

\p The possible outcomes of the oral examination are:
\begin{itemize}
\item The candidate passes without extra conditions

\item The candidate passes, but is informed of weaknesses that should be
addressed during the PhD work, e.g. to be addressed in courses or in directed
studies

\item The candidate is required to sit written examination, at a date
determined during the oral examination meeting. This examination will be based
on the topics arising during the oral defence.

\end{itemize}



%C3.3	QUALIFYING EXAMINATION OUTCOME
\subsubsection{Qualifying Examination outcome}

\cp
\p There are several possible outcomes of the qualifying examination:

\begin{itemize}

\item Continuation in the PhD program with additional course work or directed studies
\item Continuation in the PhD program without additional course work or directed studies
\item Transfer to the MSc program
\item Transfer to the PhD program with additional course work or directed studies
\item Transfer to the PhD program without additional course work or directed studies
\item Continuation in the MSc program

\end{itemize}

% C4	PhD THESIS PROPOSAL AND DEFENCE
\subsection{PhD thesis proposal and defence}

\cp

\p PhD students are expected to complete the proposal and examination of the
proposal within 20 months of enrolling in the graduate program.

\p The advisory committee sets the exact timing.

\p In order to proceed with the proposal, the student will have completed the
required first year courses, have a named supervisor and advisory committee,
and received approval from the supervisor.

\p The requirement consists of a written thesis proposal and a public oral
defence.  The written proposal will include an overview of the research topic
and relevant background material, and a plan for carrying out the PhD research.
The oral portion is a defence of the proposal, including other relevant areas
of Oceanography. The topics covered in the defence may relate not just to the
planned research, but also to wider oceanographic issues, as deemed appropriate
by the defence committee.

\p The defence is open to attendance by all interested persons, except for the
in-camera portion in which the decision is made by the examining committee.
Notices of the thesis proposal defence, including an abstract will be
distributed to all members of the Oceanography Department and will be posted.


%C4.1	EXAMINING COMMITTEE
\subsubsection{Examining committee}

\cp

\p The examining committee is selected by the supervisory committee, and consists
of members from that committee, external members, and a departmental
representative. The Graduate Secretary will arrange for the Departmental
Representative (normally the Graduate Coordinator or a designate).

\p The Chair of the PhD proposal defence is independent of the examining
committee. The Departmental Representative normally performs this duty.

\p Students must make the proposal available in the Department of Oceanography
office at least 10 working days prior to the oral defence.  The proposal
typically receives input from the research supervisor or others prior to its
submission. Approval from the advisory committee is required to proceed with
the oral defence.

%C4.2	PROPOSAL TIMELINE
\subsubsection{Proposal timeline}

\cp

\p It is the student's responsibility to initiate the proposal process and develop
the proposal in coordination with the supervisor and committee. Two terms are
more than adequate to complete this requirement. The scheduling of the proposal
defence should be planned well in advance so that it can be carried out prior
to the 20th month of the program.

\p There is no restriction on early completion of the proposal and oral exam (e.g.
in the summer between first and second years or fall term of second year).

\p Because the proposal is a formal requirement of the Department, it carries the
same weight as courses, thesis defence, etc. Failure to complete the proposal
within the required time frame reflects poorly on the student and can endanger
standing in the program.


%C4.3	PROPOSAL OUTCOME
\subsubsection{Proposal outcome}

\cp

\p Based on both the written proposal and oral defence, the supervisor will notify
the Graduate Secretary, Graduate Coordinator, student, and examining committee
in writing of one of the four possible outcomes:

\begin{itemize}
\item Continuation in the PhD program
\item Permission to re-defend
\item Transfer to the MSc program
\item Withdrawal from the program
\end{itemize}

\p Revisions, and/or a re-defence must be completed prior to the end of the exam
period in the winter term. Following the defence, a copy of the approved
proposal must be placed on file in the Oceanography office.


%C5	PhD THESIS
\subsection{PhD thesis}

\cp

\p The PhD thesis should report original research of high caliber carried out by
the student. It should be of such value as to merit publication and be in
satisfactory literary form, with clearly presented figures and other supporting
material. The thesis should be presented first, in draft form, to the
supervisor who will provide initial comments and approve distribution to the
rest of the committee. Typically, the writing phase is very intensive, and
involves working closely with the supervisor and committee members. In many
cases, the student writes journal articles that link closely with the thesis
chapters, and the Faculty of Graduate Studies has conventions on how this is to
be handled. As the thesis reaches completion, the advisory committee will meet
to discuss any final changes to the thesis and sign the PhD Thesis Submission
Form; this is just one of a series of steps that the student must keep in mind
(see section~\ref{sec:phd_thesis_defence_timeline}). 


%C5.1	PhD EXAMINING COMMITTEE
\subsubsection{PhD examining committee}
 
\cp

\p It is the supervisor's responsibility to select the examining committee
members, following the committee structure guidelines.

\p The examining committee will consist of a minimum of 5 members:

\begin{itemize}

\item 3 members of the advisory committee

    \begin{itemize}

        \item Supervisor(s)

        \item Full-time department faculty member

        \item One other member

    \end{itemize}

\item Departmental Representative (from outside the committee)\footnote{The
    Graduate Secretary will arrange for the Departmental Representative
        (normally the Graduate Coordinator or a designate).}

\item External examiner (from outside the University)

    \item Chair\footnote{The Chair of the PhD thesis defence is independent of the
        examining committee. The Faculty of Graduate Studies will appoint the Chair.}

\end{itemize}


%C5.2	PhD THESIS DEFENCE TIMELINE
\subsubsection{\label{sec:phd_thesis_defence_timeline}PhD thesis defence timeline}

\cp

\p Students should be aware of the time scales required by FGS and Departmental
regulations, which stretch over a period of more than eight months before the
defence.

\p Prior to proceeding with scheduling the defence, the thesis (in essentially
final form) should be distributed to the supervisor and committee to be
reviewed for suitability and final approval.

\begin{itemize}
    \item Twelve weeks prior to defence:

        \begin{itemize}

            \item Submit Request to Arrange an Oral Defence Form, listing 3
                choices for external examiners along with CV for external
                examiner of first choice to the Graduate Secretary.

        \end{itemize}

    \item Six weeks prior to defence:

        \begin{itemize}

            \item Have a format review of the thesis done by FGS before it is
                distributed to the examining committee

            \item Submit a PhD Thesis Submission From and copy of the thesis
                for the External Examiner to the Graduate Secretary

            \item Submit a PhD Examination Information Form to the Graduate
                Secretary

            \item Submit a copy of the student's CV to the Graduate Secretary

            \item Submit a Thesis Binding Submission Form and payment to the
                Graduate Secretary

            \item Submit an electronic copy of the abstract to the Graduate
                Secretary

            \item Provide copies of the thesis to the supervisor for
                distribution to the internal examining committee members

        \end{itemize}

    \item Four weeks prior to defence:

        \begin{itemize}

            \item Notify the Graduate Secretary of any Audiovisual requirements

        \end{itemize}

    \item Two weeks prior to defence:
        
        \begin{itemize}

            \item Submit a printed copy of the thesis to the Graduate Secretary

        \end{itemize}


    \item Following Defence:

        \begin{itemize}

            \item Have Examining Committee Members sign the signature page

            \item Submit required changes to the thesis within the specified
                timeframe

            \item Have a final format check done by FGS

            \item Electronically submit the final version of the thesis via
                DalSpace\footnote{DalSpace
                \href{http://dalspace.library.dal.ca/}{\ExternalLink}.}.

            \item Submit original completed forms to FGS (National Library of
                Canada Form, Title Page of Thesis, Signature page with original
                signatures, Copyright Page, Ethics Pages (if applicable),
                Student Contribution to Manuscripts (if applicable)

            \item Submit final copies of thesis to the Graduate Secretary for
                binding (include copies of the signed signature page, and the
                Copyright Page with original signatures)

            \item Complete Exit Survey

        \end{itemize}

\end{itemize}

\p Some of the above deadlines are set by the Oceanography Office. See the FGS
website for advice on preparing for a Doctoral defence\footnote{FGS Theses and
Defences
\href{http://www.dal.ca/faculty/gradstudies/currentstudents/thesesanddefences/defense.html}{\ExternalLink}.}.


%C5.3	PhD THESIS DEFENCE OUTCOME
\subsubsection{PhD thesis defence outcome}

\cp

\p The student opens the examination with a 20-minute presentation of the thesis.
Other faculty, students and public may be present, and ask questions, but the
critical decisions are made in private and are the responsibility of the
examining committee.

\p Following the defence, the candidate will receive a letter from the Faculty of
Graduate Studies indicating the nature of any corrections to be made and the
time frame within which they are to be completed.  In the event of an
unsuccessful defence, an explanation of the negative outcome is provided.  

%C5.4 PhD SEMINAR PRESENTATION
\subsubsection{PhD seminar presentation}

\cp

\p A departmental seminar on the thesis should be presented after successful
completion of the thesis examination.

%D.	FINANCIAL  SUPPORT
\section{Financial  support}

\cp

\p Dalhousie Graduate Fellowships will normally be provided for two years for an
MSc student and four years for a PhD student.

\p Scholarship support is provided directly to some students by the Natural
Sciences and Engineering Research Council (\di{NSERC}), the \di{Killam
Foundation} through Dalhousie, by companies, and by other agencies, both
national and international. Students are encouraged to apply for external
support whenever possible.

\p Full-time students without such special scholarship support are generally
awarded Dalhousie Graduate Fellowships from funds that come partly to the
Department from the University and mainly from individual faculty members'
research grants.

\p Students may also receive extra support by taking on a teaching assistantship,
or by demonstrating in undergraduate laboratories.

\p Students should discuss any additional jobs with their advisory committee to
ensure that the time commitments outside the Department are not excessive, and
do not interfere with program expectations.


%D1	FGS CONFERENCE TRAVEL GRANT
\subsection{FGS conference travel grant}

\cp

\p Applications must be submitted through the Graduate Secretary at least one
month prior to the date of travel.

\p Graduate students are eligible to apply for one travel grant per degree at
Dalhousie (Master's or Doctoral) through the Faculty of Graduate
Studies\footnote{FGS Conference Travel Grant Application Form and Guidelines
\href{http://www.dal.ca/faculty/gradstudies/funding/grants.html}{\ExternalLink}.}.

\p Students must be registered in a graduate program at the time of application
and at the time of the conference. In order to be eligible, students must
present a poster of paper based on the results of their graduate thesis
research at a national or international scholarly meeting or conference.
Departmental approval must be given to these applications.


%E.	INTELLECTUAL  PROPERTY
\section{Intellectual  property}

\cp

\p The Faculty of Graduate Studies is developing a policy on Intellectual
Property, which will be available on the Faculty of Graduate Studies website.
If students and/or faculty have concerns or doubts about any issue pertaining
to any part of Section 6, consult with your Chair, Graduate Coordinator, or
Supervisor, or contact the Faculty of Graduate Studies for advice. If you feel
uncomfortable with approaching your immediate supervisor, then go to the next
level and ask to be heard in confidence\footnote{FGS Regulations, section 6.4:
Policy on Intellectual Property
\href{http://academiccalendar.dal.ca/Catalog/ViewCatalog.aspx?pageid=viewcatalog&catalogid=58&chapterid=2678&topicgroupid=10849&loaduseredits=False}{\ExternalLink}.}.

%\elink{http://academiccalendar.dal.ca/Catalog/ViewCatalog.aspx?pageid=viewcatalog&catalogid=58&chapterid=2678&topicgroupid=10849&loaduseredits=False}, section 6.4.}.

%E1	ACADEMIC INTEGRITY
\section{Academic integrity}

\cp

\p Any material submitted by a student at Dalhousie University may be checked
for originality to confirm that the student has not plagiarized from other
sources.  \di[plagiarism]{Plagiarism} is a serious academic offence that may
lead to a failing grade, suspension or expulsion from the University, or even
the revocation of a degree.  It is essential that there be correct attribution
of authorities from which facts and opinions have been derived. Prior to
submitting any paper in a course, or material for a thesis, students should
read the Policy on Integrity in Scholarly Activity\footnote{FGS Regulations,
section 6.3: Policy on Integrity in Scholarly Activity
\href{http://academiccalendar.dal.ca/Catalog/ViewCatalog.aspx?pageid=viewcatalog&catalogid=58&chapterid=2678&topicgroupid=10849&loaduseredits=False}{\ExternalLink}.}.

\section{Procedures for Student Academic Appeals and Governance of the Ad Hoc
Appeal Committee}

\subsection{Introduction}

\npara{1} The procedures detailed below will be followed in cases where a
student wishes to appeal an academic procedure, other than a grade, on the set
of criteria detailed in section 2 below.  \fixme{contradiction: near the end we
see appeals of grades} 

\npara{2} The regulations and procedures given below cover appeals launched by
both undergraduate and graduate students in the Department of Oceanography at
Dalhousie University.  

\npara{3} These procedures do not apply to labor disputes between an employee
(be that person a student, post-doc, etc.) and an employer (faculty, staff) in
regard to working conditions.  Persons wishing to dispute a labor condition
should consult either the Chair of the Department and/or their union agreement,
if applicable, or the laws governing labour relations in Nova Scotia and the
procedures detailed therein. 

\npara{4} These regulations do not apply if the cause of the dispute centers
on discrimination based on gender, race, nationality, sexual orientation,
gender identity, official language, or disability.  The complainant in such
cases must contact the Dalhousie Office of Human Rights, Equity \& Harassment
Prevention for procedures in such cases.  Nor are these regulations applicable
to cases of sexual harassment, assault, criminality, etc., where Dalhousie
Security or Law Enforcement should be involved.  

\npara{5} There are no appeals on admission decisions, including transfers to
the PhD program.  

\subsection{Basis for an Appeal}

\npara{1} The procedures below apply to the administration of qualifying and
preliminary examinations, comprehensive examinations, thesis proposal defenses,
and Master's thesis defenses.  Results (grades) of such examinations cannot be
appealed as oral portions cannot be regraded.  

\npara{2} Appeals related to Doctoral defenses must be directed to Faculty of
Graduate Studies in the first instance. 

\npara{3} Appeals of course examination grades (re-grading) are direct through
the University Registrar's Office, and not the Department of Oceanography.
Some clarification on the method of reassessment is given below in
section~\ref{sec:grade_reassessment}; otherwise the appellant must follow the
procedures set by the Registrar. \fixme{should we just leave the rules to the
registrar, thus axiomatically avoiding contradiction?}
 
\npara{4} The grounds for appeal are limited to the following:
\begin{enumerate} \item procedural unfairness \item bias \item irregularity in
        procedure \item inappropriate or unfair expectation(s) \end{enumerate}

``Procedural Unfairness'' means that the method used to administer the academic
process, e.g., an exam, was unreasonably and inherently stacked against any
person attempting the process.  Note: exams and their content, in and of
themselves, are not procedurally unfair.  Differences in how an exam is
administered between the sub-disciplines in the Department also does not
constitute unfairness, as long as the method is applied consistently within
that sub-discipline.  

``Bias'' means that the process was conducted in such a way as to disadvantage the
particular appellant, relative to other persons undertaking the same process.    

``Irregularity in procedure'' means that the process did not follow the
procedures set in the regulations and guidelines governing the actions of the
Department and its faculty.  

``Inappropriate or unfair expectation'' means that the deliverable of the
process could not reasonably be expected of a person who undertakes
assiduously, genuinely, and accurately the steps leading to completion of that
process.   

For all these grounds, the onus is on the appellant to prove, through factual
documentation, that such conditions existed. 

\npara{5} A written appeal must be submitted to the Chair of the Department of
Oceanography within 30 days following the event or circumstances being
appealed. 

\npara{6} The appeal submission must include:
\begin{enumerate}

\item A description of the exact nature of the appeal, including a summary of
events and chronology; 

\item Specific details of the alleged unfairness, bias or irregularity, and any
other relevant consideration or information; and 

\item The requested resolution of the appeal, which is limited to a reasonable
academic action(s). 

\end{enumerate}

\npara{7} The submission of an appeal will engender the following actions by
the Chair of the Department:
\begin{enumerate}

\item The Chair will contact both the appellant and the person responsible for
the process being appealed to see if an informal resolution is first possible, and

\item Failing an informal resolution, the Chair will constitute an Ad Hoc
Appeal Panel for that specific appeal.  The nature of and procedures for that
Panel are detailed below in section~\ref{sec:the_ad_hoc_appeal_panel}.

\end{enumerate}

\npara{7} Decisions of the Ad Hoc Appeal Panel are forwarded to the Chair, who
will communicate the decision in writing to the appellant.

\npara{8} Decisions of the Ad Hoc Appeal Panel are subject to further appeal to
the Faculty of Science (undergraduates) or the Faculty of Graduate Studies
(graduate students), or Senate, as regulations in those administrative units
specify. 


\subsection{\label{sec:the_ad_hoc_appeal_panel}The Ad Hoc Appeal Panel}

\npara{1} Decisions on appeals are made by an Ad Hoc Appeal Panel.

\npara{2} The Panel does not have fixed membership.  Instead, it will consist
of three faculty members, chosen on a fixed rotating basis from all the faculty
in the Department.  The professor(s) responsible for the process being appealed
will be excluded.  For graduate students, the student’s supervisor or
co-supervisors, or for undergraduates, the student’s honors supervisor will
also be excluded from the list. \fixme{not all undergrads are in honours} 

\npara{3} An appropriate graduate student (for undergraduate appeals) or
Post-doctoral fellow (for graduate appeals) will constitute the fourth and
final member of the Panel.  Any student on the Panel will not have a personal
or academic conflict of interest with respect to the appellant.  For
undergraduates, the student will be chosen from the graduate students.  For
graduate students, the student member will be chosen from the Post-doctoral
Fellows, without conflict of interest.  

\npara{4} The Panel will examine the appeal and determine if a \emph{prima
facie} case exists for the appeal.  (\emph{Prima facie} in this context means
the appeal document contains sufficient evidence to support the stated claim.)
If such a case exists, then the Panel will convene a formal Appeal Hearing at
the first opportunity when all involved, i.e., appellant, the faculty member
who owns the process, and the Panel members can meet to hear verbal arguments
on the appeal, but preferably within three weeks of constitution of the Panel.
If there is no \emph{prima facie} case, the Panel will inform the Chair and the
appeal will be dismissed at the Departmental level. \fixme{edit: italic for
Latin}

\npara{6} If a Hearing is convened, the Panel will chose a Chair who will
control and direct all communication.  \fixme{part 5 is missing}

\npara{7} The Chair of the Panel will make the faculty owner of the process
aware of the appeal and provide, in a timely manner, a copy of the appeal
document to the process-owner (faculty member).  

\npara{8} Before the Hearing, the \add{process} owner will provide \delete{to}
the Panel with a written rebuttal to the appeal within 5 working days of the
appeal being received.  (An exception will be made to accommodate persons at
sea.)  The Panel will forward that rebuttal to the appellant.\fixme{Is sea time
the only exception? What about serious medical issues, etc? I imagine there is
some standard wording for such things.} 

\npara{8} The procedure for an Appeal Hearing is detailed in section 4 below.
\fixme{duplicate heading}


\subsection{The Appeal Hearing}

\npara{1} The Hearing will adhere to the following steps:

\begin{enumerate}

\item The Chair of the Panel will direct the flow of the Hearing, making sure
    both parties are properly heard\add{,} and directing questioning;

\item The Chair of the Panel will first invite the appellant to present and
    explain their case; this will be followed by an invitation \delete{for}
    \add{to} the faculty member owner for a rebuttal;

\item Neither the appellant nor the owner can ask questions during these
    presentations; however, any Panel member may interrupt the presentation to
    ask questions;

\item At the end of the presentations, the appellant will be invited to direct
questions to the faculty owner for the purpose of seeking clarifications
(only); after that is completed, the faculty owner will be invited to do the
same;  the Panel members may interrupt these questioning sessions at any
time to ask their own questions; the Chair of the panel will ensure that this
\delete{questing}\add{questioning} remains civil and on topic;

\item At the end of the questioning, the Chair of the Panel will ask the
    appellant and faculty owner to leave and convene an \delete{in camera}
    \add{\emph{in-camera}} discussion of the presentations and Appeal Document,
    the Panel will decide on the validity of the appeal and of requested
    resolution.

\end{enumerate}

\npara{2} The Chair of the Panel will communicate, in writing, the decision of
the Panel within 3 working days (except if sea-time interferes).  \fixme{see a
previous note about sea time}

\npara{3} In an appeal process, the student has the right to representation.
The student is required to inform the Chair of the Panel, in writing, if s/he
will have a representative at the appeal\add{,} or \add{intends to} call
witnesses. 

\npara{4} Witnesses may be called by either party, but only if those witnesses
can testify about direct knowledge of the matter.  No character or indirect
testimony is permitted.  The Panel Chair must be informed 72-hours before a
Hearing if a witness(es) will be called, along with that person's identity and
significance to the Hearing.  The Chair of the Panel will then inform the other
party of the witness(es).  

\npara{5} No persons providing solely moral support can be present at a
Hearing.  A person providing aid to someone with a disability is permitted,
which may include translation.  Such a person cannot testify at a Hearing.
\fixme{I'm confused on 'translation'. Does this mean e.g. someone who would
sign, to aid someone with difficulty hearing? If it means translation for
language, does this make sense? Isn't English the stated language of business
at Dalhousie?}


\subsection{Supplementary Points}

\npara{1} The Ad Hoc Appeal Panel has no jurisdiction to hear student appeals
on a matter involving a requested exemption from the application of
Departmental, Faculty or University regulations or procedures, except when
irregularities or unfairness in the application thereof is alleged.  This means
that only procedural issues, and not the merits of the regulations, are subject
to appeal.

\npara{2} The Ad Hoc Appeal Panel may not render decisions counter to
Departmental, Faculty or University regulations, nor can it make decisions that
go beyond strictly academic matters, e.g., financial or administrative.  If the
requested resolution contains such points, these must be ignored and dismissed.   

\npara{3} Matters involving allegations of ``failure to supervise'' by a
graduate student will be referred directly to the Faculty of Graduate Studies
for resolution.  


\subsection{\label{sec:grade_reassessment}Grade Reassessment (Re-grading)}

\fixme{Earlier, it says we just follow university rules. It seems best to do
that, rather than to write something here that could be wrong. I confess I
don't really understand the system of the two levels. What I read here is
consistent with what I've seen done in the past (e.g. I regraded an exam for
Barry Ruddick's class, once).}

\npara{1} A request to re-grade a written exam or test follows the procedures
set by the University's Registrar.

\npara{2} A grade cannot be both reassessed via the Registrar's Office and
appealed to the Department, under the rules in this document.  A student must
choose one procedure or the other, as they are mutually exclusive.  Under
normal circumstances a student should ask for a re-grading, unless there is
evidence of bias, irregularity, unfairness in the administration of the exam,
or inappropriate expectation.   

\npara{3} When the Department receives a re-grading request, the instructor of
the course will prepare the following materials:

\begin{enumerate}

\item  A list of at least two independent other professors (including contact
information), either within or outside of the University, who are sufficiently
knowledgeable of the topic in question, so as to be suitable to re-grade the
exam.  (Real and apparent conflicts of interest must be avoided in choosing
these re-assessors.) 

\item A copy of the exam in question, along with a template of ``optimal''
    answers for each question.

\item Copies of at least one student-completed exam at both the A and B grade
levels, fully de-personalized.  

\item A fully de-personalized copy of the completed exam being re-graded.

\end{enumerate}

\npara{4} The above materials will be given to the Chair of the Department who
will then contact the re-assessors and forward the materials if they accept the
re-grading task.  The re-grading should be complete within two weeks of receipt
of the material by the re-assessors.  

\npara{5} Communication of the results of the re-assessment will follow
Registrar procedures.   





\printindex

\end{document}
