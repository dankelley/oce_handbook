%%% this whole document is commented-out. It was just a draft, and the material is no
%%% included in the main document.

%%% %C3	PhD QUALIFYING EXAMINATION
%%% %\subsection{\label{sec:phd_qualifying_examination}PhD Qualifying Examination}
%%% 
%%% 
%%% %  \texttt{meetings/20160617/20160617\_meeting\_notes.jpg}
%%% %  
%%% %  \texttt{meetings/20160708/20160708\_meeting\_notes.jpg}
%%% %  
%%% %  \texttt{marked\_up\_versions/bo\_qe\_20160708.pdf}
%%% 
%%% \cp
%%% 
%%% \p The \QE (\qe, henceforth) is an oral examination based on a reading list
%%% tailored to the PhD candidate's general research area. It is a Department
%%% process, not a Supervisory Committee process. The Department delegates its
%%% administration to the regular Professors in the student's subdiscipline(s).
%%% (If the student is co-supervised by an Adjunct Professor or a cross-listed
%%% Professor, then that individual also takes part in the \qe.)\discuss{DK: do I
%%% have this exception correct?}
%%% 
%%% \p The \qe should be completed between months 9 and 12 of the PhD program.
%%% Students who have transferred from the MSc program should take the QE within 12
%%% months of their transfer.\discuss{DK: did we say that exceptions could be
%%% granted by the CurrComm?}
%%% 
%%% \p The details and timing are as follows.
%%% 
%%% \begin{enumerate}
%%% 
%%%     \item The Candidate prepares a 1 to 3 page\vote{DK: there has been some
%%%         discussion (by BB and DK) of shortening this further, even to just
%%%         keywords or a few senetences. I propose to put a flag in the document
%%%         to invite discussion. OK?} summary of their general research interests
%%%         and forwards it to their Supervisor for approval.  The summary is
%%%         intended to guide faculty members in compiling a reading list.
%%% 
%%%     \item The Supervisor forwards the summary to faculty members in the
%%%         student's subdiscipline(s), along with a request that they, and the
%%%         Supervisor, each provide one or two papers (book chapters are also
%%%         acceptable) to make up a reading list that is relevant to the summary.
%%%         The list should not contain anything written by a member of the Examining
%%%         Committee.\vote{DK to GOC: if we don't agree on this, there should be an
%%%         indication of this in the document, to invite wider opinions.}
%%%         At least three faculty members, including the supervisor, must
%%%         contribute to the reading list. The list must contain between 4 and 6
%%%         papers\discuss{DK: I wrote in ``6'' here \ldots I don't recall if we
%%%         set a limit, and I've seen more than 6, but I don't want us sliding
%%%         into a system where every prof gives 2 papers so we can have 10 papers
%%%         in total, potentially}. The suggested papers must be forwarded to the
%%%         Supervisor within 2 weeks of the request.\discuss{I don't understand
%%%         this forwarding.}
%%% 
%%%     \item The \supervisor vets the papers to ensure appropriateness, and then
%%%         distributes the list to the Candidate and the relevent
%%%         faculty.\discuss{DK to GOC: we need to discuss whether the \supervisor vets
%%%         the papers, and also who organizes the whole procedure. I could see
%%%         and argument that papers be sent to the \GS for organization, and that
%%%         seems to be the practice in ChemOce. But of course the \GS cannot vet
%%%         papers. And I don't think we can ask the \GC to vet them, either. So
%%%         there is an open question of whether, or how, to decide on
%%%         apprriateness. In the days of ``old comps'' there was a committee, but
%%%         the work was 5X less in scope since students wrote common exams.}
%%%             
%%%     \item The qualifying exam must be held within 6 weeks of the reading list
%%%         being distributed. 
%%% 
%%%     \item It is the responsibility of the \supervisor to form the Examining
%%%         Committee and schedule the exam. The committee must include a Chair
%%%         drawn from the student's subdiscipline(s), at least three members of
%%%         the subdisicpline, including the \supervisor, and an External Examiner
%%%         drawn from the Oceanography faculty outside the sub-discipline(s) and
%%%         preferably in a sub-discipline closely related to the candidate’s
%%%         research interests. \emph{In exceptional circumstances an External from
%%%         outside the department may be nominated by the \supervisor. The
%%%         nomination must be approved by at least three BO faculty members. The
%%%         nominee must have no conflict of interest with the Candidate’s research
%%%         programme.}\discuss{I think the highlighted text should be deleted.}
%%% 
%%%     \item The form of the Examining Committee must be approved by at least
%%%         three members of the subdisciplinary faculty no later than 3 weeks
%%%         prior to the date of the exam. Once approved, the \supervisor forwards
%%%         the research summary, the reading list, and these guidelines to the
%%%         External Examiner and to the Chair. The External is not required to
%%%         read the listed papers.\discuss{Is anyone ``required''? My impression,
%%%         having been at a lot of these, is that examiners often have not read
%%%         the other papers, especially if they are specialized (which I think
%%%         is a mistaken choice of paper, but that's just my own view).}
%%%         \vote{Does the external provide a paper? I think there is a lot of
%%%         merit in that, and I vote ``yes''.} At the same time, the Candidate is
%%%         advised of the committee membership. 
%%% 
%%%     \item The Chair moderates the exam (see guidelines below) and does not
%%%         participate in questioning. Ideally, the exam will last no longer than
%%%         1.5~hours. The exam begins with a 20-minute presentation by the
%%%         Candidate who will attempt to demonstrate an understanding of the
%%%         concepts and contents of the papers, perhaps highlighting common
%%%         threads or links to the rest of the literature or the student's plans,
%%%         as appropriate.
%%% 
%%%     \item The committee then questions the Candidate about oceanography and
%%%         topics related to the Candidate’s research interests where the entry
%%%         points into questions and discussions are to be based on the reading
%%%         material that was provided to the Candidate and Candidate’s
%%%         presentation. The exam is not about the proposed research.
%%% 
%%%     \item Upon the completion of questioning, the committee meets \emph{in
%%%         camera} to agree on the exam outcome and recommendations to the
%%%         Candidate and their \advcom, as appropriate.
%%% 
%%%     \item The possible outcomes of the Qualifying Examination are:
%%% 
%%%         \begin{enumerate}
%%%             \item The candidate passes, without extra conditions.
%%% 
%%%             \item The candidate passes, but is informed of weaknesses that
%%%                 should be addressed during the PhD work, in the form of
%%%                 courses, audits, or directed studies.
%%% 
%%%             \item The candidate is required to take a written examination where
%%%                 in the format and time line for the exam will determined by the
%%%                 examining committee while meeting \emph{in camera}. This
%%%                 examination will be based on the topics that arose during the
%%%                 oral examination and will not exceed three hours. 
%%%                 
%%% 
%%%             \item The candidate is transferred to, or continued in, the MSc
%%%                 programme.  \vote{DK to GOC: I added this, based on discussions
%%%                 with BB and others about the fact that this is a formal
%%%                 examination, and we had no ``failure'' mode.  OK?}
%%% 
%%%         \end{enumerate}
%%% 
%%%     \item The possible outcomes of a followup written examination are:
%%% 
%%%         \begin{enumerate}
%%% 
%%%             \item The candidate passes without extra conditions.
%%% 
%%%             \item The candidate passes, but is informed of weaknesses that
%%%                 should be addressed during the PhD work, in the form of
%%%                 courses, audits, or directed studies.
%%% 
%%%             \item The candidate is transferred to, or continued in, the MSc programme.
%%% 
%%%         \end{enumerate}
%%% 
%%%     \item The Chair shall report in writing to the Department Chair and the \GC
%%%         to summarise the procedure and outcome of the exam.  Copies of the
%%%         report will be forwarded to the Candidate and the Examining Committee.
%%% 
%%% \end{enumerate}
%%% 
%%% \p Guidelines for the Chair of {\QE}s.
%%% 
%%% \begin{itemize}
%%% 
%%%     \item The Exam:
%%% 
%%%         \begin{itemize}
%%% 
%%%             \item Introduce the Candidate, the External and the Examining
%%%                 Committee as needed.
%%% 
%%%             \item Summarise exam procedure: presentation, questioning, in
%%%                 camera meeting, decision and recommendation.
%%% 
%%%             \item Keep presentation by the Candidate to 20 minutes; at 25 warn a cut off.
%%% 
%%%             \item Announce the questioning order, beginning with External and
%%%                 ending with \supervisor.
%%% 
%%%             \item Limit the examination to two rounds of questioning, the
%%%                 second shorter than the first.
%%% 
%%%             \item Keep questioners on time, with 10 minutes (15 minutes for the
%%%                 External Examiner) on the first round, and less on the second
%%%                 round.
%%% 
%%%             \item Keep notes on the time used by questioners.
%%% 
%%%             \item Do not let questioners go astray--keep everyone on
%%%                 track.\discuss{I don't know what this means, or how to
%%%                 accomplish it}
%%% 
%%%             \item Discourage too much discussion back and forth between members
%%%                 of the Examining Committee; questions should be directed at the
%%%                 student and the student should be permitted to respond.
%%% 
%%%             \item Do not allow the \supervisor to answer questions directed at student.
%%% 
%%%             \item Keep the process just, calm and professional.
%%% 
%%%             \item The entire exam should last no longer than 1.5 hours.\vote{DK
%%%                 to GOC: we should use history to guide this.  And a nominal
%%%                 calculation is: 20 min + 10min/examinar*5examiner=2 hours just
%%%                 for one round, without in-camear.  I think stating 2.5 hours
%%%                 for the non-in-camera part is more sensible. OK?}
%%% 
%%%             \item Keep notes on significant issues during questioning.
%%% 
%%%         \end{itemize}
%%% 
%%%     \item The \emph{in camera} meeting:
%%% 
%%%         \begin{itemize}
%%% 
%%%             \item Remind the Examining Committee of the Defence outcome options
%%%                 (see above).
%%% 
%%%             \item Ask for recommendation and comments from the External first.
%%% 
%%%             \item Ask remainder of the Committee, in order of questioning, for
%%%                 their recommendation and comments.
%%% 
%%%             \item Encourage either a unanimous decision, or at least a
%%%                 consensus one. In the case of a split decision, the Chair casts
%%%                 the deciding vote.\discuss{DK to GOC: note that I say the chair
%%%                 can cast a deciding vote. The point is we cannot have a
%%%                 non-outcome. This is an examination; after all, and students
%%%                 cannot be told that the professors cannot decide whether they
%%%                 pass or fail.}
%%% 
%%%             \item Keep notes on significant issues that should be included in
%%%                 the report sent to the Chair and the departmental \GC.
%%% 
%%%         \end{itemize}
%%% 
%%%     \item The Report:
%%% 
%%%         \begin{itemize}
%%% 
%%% 
%%%             \item The examination Chair shall submit a written report to the
%%%                 Chair of the Department and the departmental \GC.  The report
%%%                 shall include a written description of the procedure and
%%%                 outcome of the exam and associated recommendations. The report
%%%                 shall be included in the Candidate’s departmental file.
%%% 
%%%         \end{itemize}
%%% 
%%% \end{itemize}
%%% 
