\section{Procedures for Student Academic Appeals and Governance of the Ad Hoc
Appeal Committee}

\subsection{Introduction}

\npara{1} The procedures detailed below will be followed in cases where a
student wishes to appeal an academic procedure, other than a grade, on the set
of criteria detailed in section 2 below.  \fixme{contradiction: near the end we
see appeals of grades} 

\npara{2} The regulations and procedures given below cover appeals launched by
both undergraduate and graduate students in the Department of Oceanography at
Dalhousie University.  

\npara{3} These procedures do not apply to labor disputes between an employee
(be that person a student, post-doc, etc.) and an employer (faculty, staff) in
regard to working conditions.  Persons wishing to dispute a labor condition
should consult either the Chair of the Department and/or their union agreement,
if applicable, or the laws governing labour relations in Nova Scotia and the
procedures detailed therein. 

\npara{4} These regulations do not apply if the cause of the dispute centers
on discrimination based on gender, race, nationality, sexual orientation,
gender identity, official language, or disability.  The complainant in such
cases must contact the Dalhousie Office of Human Rights, Equity \& Harassment
Prevention for procedures in such cases.  Nor are these regulations applicable
to cases of sexual harassment, assault, criminality, etc., where Dalhousie
Security or Law Enforcement should be involved.  

\npara{5} There are no appeals on admission decisions, including transfers to
the PhD program.  

\subsection{Basis for an Appeal}

\npara{1} The procedures below apply to the administration of qualifying and
preliminary examinations, comprehensive examinations, thesis proposal defenses,
and Master's thesis defenses.  Results (grades) of such examinations cannot be
appealed as oral portions cannot be regraded.  

\npara{2} Appeals related to Doctoral defenses must be directed to Faculty of
Graduate Studies in the first instance. 

\npara{3} Appeals of course examination grades (re-grading) are direct through
the University Registrar's Office, and not the Department of Oceanography.
Some clarification on the method of reassessment is given below in
section~\ref{sec:grade_reassessment}; otherwise the appellant must follow the
procedures set by the Registrar. \fixme{should we just leave the rules to the
registrar, thus axiomatically avoiding contradiction?}
 
\npara{4} The grounds for appeal are limited to the following:
\begin{enumerate} \item procedural unfairness \item bias \item irregularity in
        procedure \item inappropriate or unfair expectation(s) \end{enumerate}

``Procedural Unfairness'' means that the method used to administer the academic
process, e.g., an exam, was unreasonably and inherently stacked against any
person attempting the process.  Note: exams and their content, in and of
themselves, are not procedurally unfair.  Differences in how an exam is
administered between the sub-disciplines in the Department also does not
constitute unfairness, as long as the method is applied consistently within
that sub-discipline.  

``Bias'' means that the process was conducted in such a way as to disadvantage the
particular appellant, relative to other persons undertaking the same process.    

``Irregularity in procedure'' means that the process did not follow the
procedures set in the regulations and guidelines governing the actions of the
Department and its faculty.  

``Inappropriate or unfair expectation'' means that the deliverable of the
process could not reasonably be expected of a person who undertakes
assiduously, genuinely, and accurately the steps leading to completion of that
process.   

For all these grounds, the onus is on the appellant to prove, through factual
documentation, that such conditions existed. 

\npara{5} A written appeal must be submitted to the Chair of the Department of
Oceanography within 30 days following the event or circumstances being
appealed. 

\npara{6} The appeal submission must include:
\begin{enumerate}

\item A description of the exact nature of the appeal, including a summary of
events and chronology; 

\item Specific details of the alleged unfairness, bias or irregularity, and any
other relevant consideration or information; and 

\item The requested resolution of the appeal, which is limited to a reasonable
academic action(s). 

\end{enumerate}

\npara{7} The submission of an appeal will engender the following actions by
the Chair of the Department:
\begin{enumerate}

\item The Chair will contact both the appellant and the person responsible for
the process being appealed to see if an informal resolution is first possible, and

\item Failing an informal resolution, the Chair will constitute an Ad Hoc
Appeal Panel for that specific appeal.  The nature of and procedures for that
Panel are detailed below in section~\ref{sec:the_ad_hoc_appeal_panel}.

\end{enumerate}

\npara{7} Decisions of the Ad Hoc Appeal Panel are forwarded to the Chair, who
will communicate the decision in writing to the appellant.

\npara{8} Decisions of the Ad Hoc Appeal Panel are subject to further appeal to
the Faculty of Science (undergraduates) or the Faculty of Graduate Studies
(graduate students), or Senate, as regulations in those administrative units
specify. 


\subsection{\label{sec:the_ad_hoc_appeal_panel}The Ad Hoc Appeal Panel}

\npara{1} Decisions on appeals are made by an Ad Hoc Appeal Panel.

\npara{2} The Panel does not have fixed membership.  Instead, it will consist
of three faculty members, chosen on a fixed rotating basis from all the faculty
in the Department.  The professor(s) responsible for the process being appealed
will be excluded.  For graduate students, the student’s supervisor or
co-supervisors, or for undergraduates, the student’s honors supervisor will
also be excluded from the list. \fixme{not all undergrads are in honours} 

\npara{3} An appropriate graduate student (for undergraduate appeals) or
Post-doctoral fellow (for graduate appeals) will constitute the fourth and
final member of the Panel.  Any student on the Panel will not have a personal
or academic conflict of interest with respect to the appellant.  For
undergraduates, the student will be chosen from the graduate students.  For
graduate students, the student member will be chosen from the Post-doctoral
Fellows, without conflict of interest.  

\npara{4} The Panel will examine the appeal and determine if a \emph{prima
facie} case exists for the appeal.  (\emph{Prima facie} in this context means
the appeal document contains sufficient evidence to support the stated claim.)
If such a case exists, then the Panel will convene a formal Appeal Hearing at
the first opportunity when all involved, i.e., appellant, the faculty member
who owns the process, and the Panel members can meet to hear verbal arguments
on the appeal, but preferably within three weeks of constitution of the Panel.
If there is no \emph{prima facie} case, the Panel will inform the Chair and the
appeal will be dismissed at the Departmental level. \fixme{edit: italic for
Latin}

\npara{6} If a Hearing is convened, the Panel will chose a Chair who will
control and direct all communication.  \fixme{part 5 is missing}

\npara{7} The Chair of the Panel will make the faculty owner of the process
aware of the appeal and provide, in a timely manner, a copy of the appeal
document to the process-owner (faculty member).  

\npara{8} Before the Hearing, the \add{process} owner will provide \delete{to}
the Panel with a written rebuttal to the appeal within 5 working days of the
appeal being received.  (An exception will be made to accommodate persons at
sea.)  The Panel will forward that rebuttal to the appellant.\fixme{Is sea time
the only exception? What about serious medical issues, etc? I imagine there is
some standard wording for such things.} 

\npara{8} The procedure for an Appeal Hearing is detailed in section 4 below.
\fixme{duplicate heading}


\subsection{The Appeal Hearing}

\npara{1} The Hearing will adhere to the following steps:

\begin{enumerate}

\item The Chair of the Panel will direct the flow of the Hearing, making sure
    both parties are properly heard\add{,} and directing questioning;

\item The Chair of the Panel will first invite the appellant to present and
    explain their case; this will be followed by an invitation \delete{for}
    \add{to} the faculty member owner for a rebuttal;

\item Neither the appellant nor the owner can ask questions during these
    presentations; however, any Panel member may interrupt the presentation to
    ask questions;

\item At the end of the presentations, the appellant will be invited to direct
questions to the faculty owner for the purpose of seeking clarifications
(only); after that is completed, the faculty owner will be invited to do the
same;  the Panel members may interrupt these questioning sessions at any
time to ask their own questions; the Chair of the panel will ensure that this
\delete{questing}\add{questioning} remains civil and on topic;

\item At the end of the questioning, the Chair of the Panel will ask the
    appellant and faculty owner to leave and convene an \delete{in camera}
    \add{\emph{in-camera}} discussion of the presentations and Appeal Document,
    the Panel will decide on the validity of the appeal and of requested
    resolution.

\end{enumerate}

\npara{2} The Chair of the Panel will communicate, in writing, the decision of
the Panel within 3 working days (except if sea-time interferes).  \fixme{see a
previous note about sea time}

\npara{3} In an appeal process, the student has the right to representation.
The student is required to inform the Chair of the Panel, in writing, if s/he
will have a representative at the appeal\add{,} or \add{intends to} call
witnesses. 

\npara{4} Witnesses may be called by either party, but only if those witnesses
can testify about direct knowledge of the matter.  No character or indirect
testimony is permitted.  The Panel Chair must be informed 72-hours before a
Hearing if a witness(es) will be called, along with that person's identity and
significance to the Hearing.  The Chair of the Panel will then inform the other
party of the witness(es).  

\npara{5} No persons providing solely moral support can be present at a
Hearing.  A person providing aid to someone with a disability is permitted,
which may include translation.  Such a person cannot testify at a Hearing.
\fixme{I'm confused on 'translation'. Does this mean e.g. someone who would
sign, to aid someone with difficulty hearing? If it means translation for
language, does this make sense? Isn't English the stated language of business
at Dalhousie?}


\subsection{Supplementary Points}

\npara{1} The Ad Hoc Appeal Panel has no jurisdiction to hear student appeals
on a matter involving a requested exemption from the application of
Departmental, Faculty or University regulations or procedures, except when
irregularities or unfairness in the application thereof is alleged.  This means
that only procedural issues, and not the merits of the regulations, are subject
to appeal.

\npara{2} The Ad Hoc Appeal Panel may not render decisions counter to
Departmental, Faculty or University regulations, nor can it make decisions that
go beyond strictly academic matters, e.g., financial or administrative.  If the
requested resolution contains such points, these must be ignored and dismissed.   

\npara{3} Matters involving allegations of ``failure to supervise'' by a
graduate student will be referred directly to the Faculty of Graduate Studies
for resolution.  


\subsection{\label{sec:grade_reassessment}Grade Reassessment (Re-grading)}

\fixme{Earlier, it says we just follow university rules. It seems best to do
that, rather than to write something here that could be wrong. I confess I
don't really understand the system of the two levels. What I read here is
consistent with what I've seen done in the past (e.g. I regraded an exam for
Barry Ruddick's class, once).}

\npara{1} A request to re-grade a written exam or test follows the procedures
set by the University's Registrar.

\npara{2} A grade cannot be both reassessed via the Registrar's Office and
appealed to the Department, under the rules in this document.  A student must
choose one procedure or the other, as they are mutually exclusive.  Under
normal circumstances a student should ask for a re-grading, unless there is
evidence of bias, irregularity, unfairness in the administration of the exam,
or inappropriate expectation.   

\npara{3} When the Department receives a re-grading request, the instructor of
the course will prepare the following materials:

\begin{enumerate}

\item  A list of at least two independent other professors (including contact
information), either within or outside of the University, who are sufficiently
knowledgeable of the topic in question, so as to be suitable to re-grade the
exam.  (Real and apparent conflicts of interest must be avoided in choosing
these re-assessors.) 

\item A copy of the exam in question, along with a template of ``optimal''
    answers for each question.

\item Copies of at least one student-completed exam at both the A and B grade
levels, fully de-personalized.  

\item A fully de-personalized copy of the completed exam being re-graded.

\end{enumerate}

\npara{4} The above materials will be given to the Chair of the Department who
will then contact the re-assessors and forward the materials if they accept the
re-grading task.  The re-grading should be complete within two weeks of receipt
of the material by the re-assessors.  

\npara{5} Communication of the results of the re-assessment will follow
Registrar procedures.   



